\documentclass[float=false, crop=false]{standalone}
\newcommand{\leaf}{\textit{Leaf}}
\newcommand{\mail}[1]{\href{mailto:#1}{\texttt{#1}}}
\newcommand{\mailleaf}{\mail{leaf.gruppo@gmail.com}}
\newcommand{\mailinglist}{\mail{nome@mailing.list}}
\newcommand{\responsabilediprogetto}{\textit{Responsabile di progetto}}
\newcommand{\g}{\tiny \ped{|G|} \normalsize}
\usepackage[T1]{fontenc}				% codifica dei font in uscita
\usepackage[utf8]{inputenc}				% lettere accentate da tastiera
\usepackage[italian]{babel}				% lingua principale del documento
\usepackage{url}						% per scrivere gli indirizzi Internet
\usepackage[subpreambles=true]{standalone}
\usepackage[bookmarks=false,hyperfootnotes=false]{hyperref}
\hypersetup{
			colorlinks=true,
			linkcolor=blue,
			anchorcolor=black,
			citecolor=black,
			urlcolor=blue,
}
\usepackage{import}

\begin{document}
	\section{Comunicazione}
		\subsection{Comunicazioni interne}
		Per le comunicazioni interne verrà utilizzata una mailing list appositamente creata: \mailinglist. \\
		Il componente del gruppo che vuole inviare una email a tutti gli altri componenti deve inviare il messaggio dalla sua casella di posta elettronica all'indirizzo \mailinglist, in questo modo l'email verrà spedita a tutti i membri del team. Tale sistema deve essere utilizzato solo per questioni riguardanti il progetto. \\
		Per facilitare la comunicazione il gruppo può avvalersi degli strumenti di messaggistica istantanea quali Slack e Telegram e per le videochiamate di Skype. \\
		Quando i sistemi sopra elencati venissero utilizzati per decisioni rilevanti per lo sviluppo del progetto è necessario stilare un verbale.\\
		I membri del gruppo sono tenuti a prestare attenzione al numero di messaggi diffusi per non creare difficoltà di comunicazione. 
		\subsection{Comunicazioni esterne}
		Per la comunicazioni esterne è stato creato un apposito indirizzo di posta elettronica: \mailleaf. \\
		Il \responsabilediprogetto ho l'incarico di mantenere i contatti tra il team e le componenti esterne utilizzando tale indirizzo di posta elettronica. Inoltre è suo compito informare i membri del gruppo delle discussioni avvenute con componenti esterne: questo può essere fatto riassumendo la conversazione in una email e inviandola alla mailing list del gruppo.
		\subsection{Composizione delle email}
		Questa sezione tratta le norme da rispettare nella composizione delle email, sia per la comunicazione interna che esterna.
		\paragraph{Mittente}
		Nel caso di comunicazione interna come il mittente dovrà essere l'indirizzo personale di posta elettronica del membro del gruppo che ha scritto l'email, mentre in caso di comunicazione esterna l'unico indirizzo che deve essere utilizzato è \mailleaf. \\ Nel caso in cui la comunicazione debba avvenire tra un gruppo ristretto di persone all'interno del gruppo, questi potranno utilizzare i loro indirizzi personali.
		\paragraph{Destinatario}
		Nel caso di comunicazione interna al gruppo l'unico destinatario deve essere \mailinglist\ per facilitare l'invio a tutti i membri del gruppo stesso, mentre in caso di comunicazione esterna il destinatari possono essere il Prof. Tullio Vardanega, il Prof Riccardo Cardin oppure i committenti del progetto, a secondo dello scopo dell'email.
		\paragraph{Oggetto}
		L'oggetto dell'email deve sintetizzare il contenuto dell'email in modo più chiaro ed esaustivo possibile. Possibilmente l'oggetto delle email deve essere differente rispetto alle email inviate e ricevute in precedenza, in modo tale da rendere facilmente identificabile ogni messaggio. \\ Per comporre un messaggio di risposta è necessario anteporre all'oggetto il prefisso ``Re:'', mentre nel caso di inoltro di un messaggio è obbligatorio aggiungere il prefisso ``I:''. In entrambi i casi, le rimanenti parti dell'email non vanno modificate.
		\paragraph{Corpo}
		Il corpo del messaggio deve essere esaustivo, sintetico e deve essere comprensibile a tutti i destinatari del messaggi. \\ In caso di risposta o inoltro è preferibile aggiungere la nuova parte di testo all'inizio dell'email per permettere una più facile lettura del contenuto. \\ Nell'eventualità che nel contenuto di un'email ci si debba riferire a persone è preferibile utilizzare la sintassi: ``'Nome Cognome'', nel caso invece in cui si debba fare esplicito riferimento ad un ruolo di progetto è consigliabile riportarne per intero il nome.
		\paragraph{Allegati}
		È consigliato di non fare uso di allegati per lo scambio di documenti o file, a meno che non siano strettamente necessario. È preferibile, invece, caricare questi file in una cartella di Google Drive e inviare per email il link al documento o file desiderato. \\ Sia in caso di invio di allegati che di link a documenti o file è buona norma inserire una breve descrizione di presentazione dell'allegato in modo tale che sia possibile in modo semplice capire di cosa si tratta. \\ Infine si chiede di prestare attenzione al formato dei documenti e file.
	\section{Riunioni}
		Il \responsabilediprogetto ha il compito di indire le riunioni sia interne che esterne. Per ogni riunione il \responsabilediprogetto dovrà inviare un'email di convocazione strutturata in questo modo:
		\begin{itemize}
		\item \textbf{oggetto}: convocazione della riunione N per il giorno AAAA-mm-GG
		\item \textbf{corpo:}: 
		\begin{itemize}
		\item data;
		\item ora;
		\item luogo;
		\item tipo;
		\item ordine del giorno.
		\end{itemize}
		\end{itemize}
		Dove N rappresenta il numero della riunione e tipo indica se la riunione sia interna richiesta dal \responsabilediprogetto, interna richiesta da uno o più dai membri del gruppo oppure esterna. \\
		Le informazioni sulle riunioni devono essere presentate con più preavviso possibile, almeno tre giorni prima in modo tale che i membri del gruppo possano organizzare i loro impegni in modo tale da essere presenti alla riunione.
		\subsection{Riunioni interne}
		\subsubsection{Convocazione delle riunioni interne}
		In generale, il compito di convocare le riunioni interne spetta al \responsabilediprogetto, che può indire le riunioni quando più lo ritiene opportuno. Gli altri componenti del gruppo possono richiedere una riunione interna straordinaria, presentando al \responsabilediprogetto le motivazioni per le quali si ritiene necessaria una riunione. In questi casi il \responsabilediprogetto può:
		\begin{itemize}
		\item autorizzare lo svolgimento della riunione;
		\item negare lo svolgimento della riunione nel caso in cui non ritenga le motivazioni presentate valide abbastanza da richiedere una riunione;
		\item suggerire mezzi di comunicazione differenti.
		\end{itemize}
		In ogni caso spetta al \responsabilediprogetto decidere data, ora e luogo dell'incontro contattando i membri del team e chiedendo loro la disponibilità. Questi sono tenuti a rispondere tempestivamente in modo tale da dare la possibilità al \responsabilediprogetto di anticipare o posticipare la data o l'ora della riunione.
		\subsubsection{Gestione della riunioni interne}
		All'inizio di ogni riunione interna il \responsabilediprogetto nomina un Segretario che ha il compito di redigere la minuta della riunione, catturando possibilmente tutti i soli gli aspetti più importanti della riunione stessa. Terminato l'incontro il Segretario ha il compito di redigere il verbale della riunione. Questo verbale verrà archiviato nel repository del gruppo, per la consultazione de parte di tutti i membri. \\ Durante le riunioni i partecipanti devono tenere un comportamento che favorisca la discussione all'interno del gruppo e di tutti gli argomenti previsti. 
\end{document}
