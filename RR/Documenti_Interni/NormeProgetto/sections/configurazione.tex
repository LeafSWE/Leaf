\documentclass[../NormeProgetto.tex]{subfiles}

\begin{document}
	\section{Configurazione}
	\subsection{Controllo di versione}
	Il controllo di versione di documenti e file sorgente viene fatto utilizzando il software Git. Nonostante siano presenti varie alternative, come Mercurial, Subversion e Bazaar, è stato scelto Git in quanto soddisfa appieno le necessita di versionamento dei file per questo progetto e, inoltre, permette di versionare i file localmente, senza bisogno di una connessione internet attiva. Come servizio di hosting per la repository è stato scelto Github. Entrambe le scelte sono state fatte anche perché più membri del gruppo avevano già avuto la possibilità di lavorare con tali servizi. \\ Ogni qualvolta viene eseguita una modifica sostanziale ad un documento o ad un file sorgente deve essere assegnato un nuovo numero di versione a questo, per tener traccia delle modifiche fatte. \\ Per quanto riguarda le norme di versionamento dei file inerenti alla documentazione si rimanda alla sezione \hyperref[sec:Versionamento dei documenti]{Versionamento dei documenti} del presente documento
\subsection{Richieste di modifica}
\paragraph{Procedura di richiesta di modifica}
\subsection{Struttura del repository}
\subsection{Nomi dei file}
\subsection{Commit}
\subsection{Codifica dei file}
\subsection{Aggiornamento del repository}
\subsection{Visibilità del repository}
\end{document}