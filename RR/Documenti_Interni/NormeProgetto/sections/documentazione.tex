\documentclass[../NormeProgetto.tex]{subfiles}

\begin{document}

\section{Documentazione}
In questa sezione sono indicati gli standard riguardanti la struttura e la stesura dei documenti prodotti.

	\subsection{Struttura dei documenti}
	Ogni documento è realizzato a partire da una struttura prestabilita che dovrà essere uguale per tutti i documenti ufficiali ad eccezione dei verbali:
	\begin{enumerate}
	\item frontespizio;
	\item informazione sul documento;
	\item diario delle modifiche;
	\item indice delle sezioni;
	\item indice delle tabelle;
	\item indice delle figure;
	\item introduzione;
	\item contenuto.
	\end{enumerate}
	L'ordine di ognuna delle sezioni è fissato. La numerazione delle prime pagine sarà quella romana, mentre dall'introduzione fino alla fine del documento quella araba.

		\subsubsection{Frontespizio}
		Questa sezione deve trovarsi nella prima pagina di ogni documento e contiene:
		\begin{enumerate}
		\item informazioni sul gruppo:
			\begin{enumerate}[a.]
			\item nome;
			\item logo;
			\item mail.
			\end{enumerate}
		\item informazioni sul progetto:
			\begin{enumerate}[a.]
			\item nome progetto;
			\item nome azienda proponente.
			\end{enumerate}
		\item informazioni sul documento:
			\begin{enumerate}[a.]
			\item nome;
			\item versione.
			\end{enumerate}
		\end{enumerate}

		\subsubsection{Informazioni sul documento}
		In questa sezione vengono indicate le principale informazioni riguardanti il documento quali:
		\begin{enumerate}
		\item versione;
		\item data di redazione;
		\item cognome e nome di coloro che hanno redatto il documento (in ordine alfabetico)
		\item cognome e nome di coloro che hanno verificato il documento (in ordine alfabetico)
		\item ambito d'uso del documento (interno od esterno)
		\item cognome e nome di coloro ai quali è destinato il documento (in ordine alfabetico)
		\end{enumerate}

		\subsubsection{Diario delle modifiche}
		Questa sezione descrive, attraverso l'utilizzo di una tabella, le modifiche che sono state apportate al documento. Ogni riga della tabella corrisponde ad una modifica effettua al documento. La struttura della riga della tabella è la seguente:

		\begin{enumerate}
		\item versione del documento;
		\item data della modifica;
		\item cognome e nome dell'autore della modifica;
		\item ruolo dell'autore della modifica nel momento in cui essa è avvenuta;
		\item sommario delle modifiche apportate
		\end{enumerate}

		Le righe della tabella sono ordinate a partire dalla data dell'ultima modifica effettuata.

		\subsubsection{Indice delle sezioni}
		L'indice delle sezioni contiene l'indice di tutti gli argomenti trattati all'interno del documento. La sua struttura è la seguente:

		\begin{enumerate}
		\item titolo dell'argomento trattato;
		\item numero di pagina.
		\end{enumerate}

		\subsubsection{Indice delle tabele}
		Questa sezione contiene l'indice delle tabele. Per ogni tabella deve essere specificato:

		\begin{enumerate}
		\item titolo della tabella;
		\item numero di pagina di riferimento.
		\end{enumerate}

		Nel caso in cui non siano presenti tabelle all'interno del documento, è possibile omettere questa sezione.

		\subsubsection{Indice delle figure}
		In questa sezione sono riportate tutte le figure presenti all'interno del documento. Per ogni figura deve essere specificato:
		\begin{itemize}
		\item nome figura;
		\item pagina di riferimento della figura.
		\end{itemize}
		Nel caso in cui non siano presenti figure all'interno del documento, è possibile omettere questa sezione.

		\subsubsection{Introduzione}
		Questa sezione deve riportare le seguenti informazioni:
		\begin{enumerate}
		\item scopo del documento;
		\item glossario;
		\item riferimenti utili:
			\begin{enumerate}[a.]
			\item riferimenti normativi;
			\item riferimenti informativi;
			\end{enumerate}
		\end{enumerate}

		\subsubsection{Contenuto}
		Questa sezione contiene il contenuto del documento. Anch'esso deve essere propriamente diviso in sezioni e sottosezioni.
	
	\subsection{Norme tipografiche}
		\subsubsection{Formattazione generale}
			\paragraph{Testatine}
			Ogni pagina di un documento, fatta eccezione per il frontespizio, deve contenere la testina. Essa è composta da:
			\begin{itemize}
			\item logo del gruppo, posizionato in alto a sinistra;
			\item nome del documento, posizionato in alto a destra.
			\end{itemize}
			\paragraph{Piè pagina}
			Ogni pagina di un documento, deve contenere il piè pagina. Esso contiene:
			\begin{itemize}
			\item numero della pagina, posizionato al centro.
			\end{itemize}
			\paragraph{Orfani e vedove}
			Si considerano vedova, la riga di un paragrafo che inizia alla fine di una pagina, mentre si considera orfana, la riga di un paragrafo che finisce all'inizio di una pagina. I documenti dovranno essere redatti cercando di evitare il più possibile queste due tipologie di righe poiché risultato essere poco gradevoli. 
		\subsubsection{Caratteri}
			\paragraph{Virgolette}
			\paragraph{Parentesi}
			\paragraph{Punteggiatura}
			\paragraph{Numeri}
			\paragraph{Lettere}
		\subsubsection{Stile del testo}
			\paragraph{Corsivo}
			\paragraph{Grassetto}
			\paragraph{Sottolineato}
			\paragraph{Monospace}
			\paragraph{Glossario}
		\subsubsection{Composizione del testo}
			\paragraph{Elenchi}
			\paragraph{Descrizioni}
			\paragraph{Note a piè pagina}
		\subsubsection{Formati}
			\paragraph{Data e ora}
			\paragraph{URI}
			\paragraph{Sigle}
			\paragraph{Ruoli di progetto}
			\paragraph{Fasi del progetto}
			\paragraph{Revisioni}
			\paragraph{Nomi}
				\subparagraph{Nome del gruppo}
				\subparagraph{Nome del progetto}
				\subparagraph{Nome del proponente}
				\subparagraph{Nome del committente}
				\subparagraph{Nome proprio}
				\subparagraph{Nome di un file}
				\subparagraph{Nome di un documento}
		\subsubsection{Componenti grafiche}
			\paragraph{Immagini}
			\paragraph{Tabelle}
	\subsection{Tipologie di documenti}
		\subsubsection{Documenti formali}
		\subsubsection{Documenti informali}
		\subsubsection{Glossario}
		\subsubsection{Verbali}
			\paragraph{Verbali di riunioni interne}
			\paragraph{Verbali di riunioni esterne}
	\subsection{Versionamento dei documenti} \label{sec:Versionamento dei documenti}
	\subsection{Avanzamento di un documento} 
	\subsection{Strumenti}
		\subsubsection{Latex}
			\paragraph{Template}
			\paragraph{Comendi personalizzati}
			\paragraph{Rilevamento errori ortografici}
			
			
			
\end{document}
