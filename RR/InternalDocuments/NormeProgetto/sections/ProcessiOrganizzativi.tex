\documentclass[../NormeProgetto.tex]{subfiles}

\begin{document}
\section{Processi organizzativi}
	\subsection{Norme}
		\subsubsection{Comunicazione}
			\paragraph{Comunicazioni interne}
			Per le comunicazioni interne è stato scelto di utilizzare lo strumento Messages messo a disposizione da Teamwork.
			La procedura per inviare un messaggio ai membri del gruppo è la seguente:
			\begin{enumerate}
				\item accedere a Teamwork ed entrare nella sezione Messages;
				\item creare un nuovo messaggio inserendo oggetto e corpo del messaggio;
				\item selezionare i destinatari del messaggio ed inviare il messaggio.
			\end{enumerate}
			Per rispondere ad un messaggio ricevuto la procedura è la seguente:
			\begin{enumerate}
				\item accedere a Teamwork ed entrare nella sezione Messages;
				\item selezionare il messaggio al quale si vuole rispondere ed inviare la risposta.
			\end{enumerate}
			 Tale sistema deve essere utilizzato solo per questioni riguardanti il progetto. \\
			Per i messaggi brevi, al fine di velocizzare le comunicazioni interne, il gruppo utilizzerà lo strumento di messaggistica Telegram. Inoltre, per le videochiamate, è stato scelto di utilizzare di Skype. \\
			Qualora i sistemi sopra elencati venissero utilizzati per decisioni rilevanti per lo sviluppo del progetto è necessario stilare un verbale.\\
			I membri del gruppo sono tenuti a prestare attenzione al numero di messaggi diffusi per non creare difficoltà di comunicazione. 
			\paragraph{Comunicazioni esterne}
			Per la comunicazioni esterne è stato creato un apposito indirizzo di posta elettronica: \mailleaf. \\
			Il \responsabilediprogetto\ ha l'incarico di mantenere i contatti tra il team e le componenti esterne utilizzando tale indirizzo di posta elettronica. Inoltre è suo compito informare i membri del gruppo delle discussioni avvenute con componenti esterne: questo può essere fatto riassumendo la conversazione in una email e inviandola alla mailing list del gruppo.
			\paragraph{Composizione delle email}
			Questa sezione tratta le norme da rispettare nella composizione delle email, sia per la comunicazione interna che esterna.
			\subparagraph{Mittente}
			Nel caso di comunicazione interna il mittente dovrà essere l'indirizzo di posta elettronica personale del membro del gruppo che ha scritto l'email, mentre in caso di comunicazione esterna l'unico indirizzo che deve essere utilizzato è \mailleaf. \\ Nel caso in cui la comunicazione debba avvenire tra un gruppo ristretto di persone all'interno del gruppo, questi potranno utilizzare i loro indirizzi personali.
			\subparagraph{Destinatario}
			Nel caso di comunicazione interna al gruppo l'unico destinatario deve essere \mailinglist\ per facilitare l'invio a tutti i membri del gruppo stesso, mentre in caso di comunicazione esterna il destinatari possono essere il Prof. Tullio Vardanega, il Prof. Riccardo Cardin oppure i committenti del progetto, a seconda dello scopo dell'email.
			\subparagraph{Oggetto}
			L'oggetto dell'email deve sintetizzare il contenuto dell'email stessa in modo più chiaro ed esaustivo possibile. Possibilmente l'oggetto di una nuova email deve essere differente rispetto alle email inviate e ricevute in precedenza, in modo tale da rendere facilmente identificabile ogni messaggio. \\ Per comporre un messaggio di risposta è necessario anteporre all'oggetto il prefisso ``Re:'', mentre nel caso di inoltro di un messaggio è obbligatorio aggiungere il prefisso ``I:''. In entrambi i casi, le rimanenti parti dell'email non vanno modificate.
			\subparagraph{Corpo}
			Il corpo del messaggio deve essere esaustivo, sintetico e deve essere comprensibile a tutti i destinatari del messaggi. \\ In caso di risposta o inoltro è preferibile aggiungere la nuova parte di testo all'inizio dell'email per permettere una più facile lettura del contenuto. \\ Nell'eventualità che nel contenuto di un'email ci si debba riferire a persone è preferibile utilizzare la sintassi: ``'Nome Cognome'', nel caso invece in cui si debba fare esplicito riferimento ad un ruolo di progetto è consigliabile riportarne per intero il nome.
			\subparagraph{Allegati}
			È consigliato di non fare uso di allegati per lo scambio di documenti o file, a meno che non siano strettamente necessario. È preferibile, invece, caricare questi file in una cartella di Google Drive e inviare per email il link al documento o file desiderato. \\ Sia in caso di invio di allegati, che di link a documenti o file è buona norma inserire una breve descrizione di presentazione dell'allegato in modo tale che sia possibile in modo semplice capire di cosa si tratta. \\ Infine si chiede di prestare attenzione al formato dei documenti e file.
		\subsubsection{Riunioni}
			Il \responsabilediprogetto\ ha il compito di indire le riunioni sia interne che esterne. Per ogni riunione il \responsabilediprogetto\ dovrà inviare un'email di convocazione strutturata in questo modo:
			\begin{itemize}
			\item \textbf{oggetto}: convocazione della riunione N per il giorno AAAA-mm-GG
			\item \textbf{corpo}: 
			\begin{itemize}
			\item Data;
			\item Ora;
			\item Luogo;
			\item Tipo di riunione;
			\item Ordine del giorno.
			\end{itemize}
			\end{itemize}
			Dove N rappresenta il numero della riunione e tipo indica se la riunione sia interna richiesta dal \responsabilediprogetto, interna richiesta da uno o più dai membri del gruppo oppure esterna. \\
			Le informazioni sulle riunioni devono essere presentate con più preavviso possibile, almeno tre giorni prima in modo tale che i membri del gruppo possano organizzare i loro impegni in modo tale da essere presenti alla riunione.
			\paragraph{Riunioni interne}
			\subparagraph{Convocazione di una riunione interna}
			In generale, il compito di convocare le riunioni interne spetta al \responsabilediprogetto, che può indire le riunioni quando più lo ritiene opportuno. Gli altri componenti del gruppo possono richiedere una riunione interna straordinaria, presentando al \responsabilediprogetto\ le motivazioni per le quali si ritiene necessaria una riunione. In questi casi il \responsabilediprogetto\ può:
			\begin{itemize}
			\item Autorizzare lo svolgimento della riunione;
			\item Negare lo svolgimento della riunione, nel caso in cui non ritenga le motivazioni presentate valide abbastanza da richiedere una riunione;
			\item Suggerire mezzi di comunicazione differenti.
			\end{itemize}
			In ogni caso spetta al \responsabilediprogetto\ decidere data, ora e luogo dell'incontro contattando i membri del team e chiedendo loro la disponibilità. Questi sono tenuti a rispondere tempestivamente in modo tale da dare la possibilità al \responsabilediprogetto\ di anticipare o posticipare la data o l'ora della riunione.
			\subparagraph{Gestione di una riunioni interna}
			All'inizio di ogni riunione interna il \responsabilediprogetto\ nomina un Segretario che ha il compito di redigere la minuta della riunione, catturando possibilmente tutti i soli gli aspetti più importanti della riunione stessa. Terminato l'incontro, il Segretario ha il compito di redigere il verbale dell'incontro. Questo verbale verrà archiviato nel repository del gruppo, per la consultazione de parte di tutti i membri. \\ Durante le riunioni i partecipanti devono tenere un comportamento che favorisca la discussione all'interno del gruppo e di tutti gli argomenti previsti. 
			\paragraph{Riunioni esterne}
			\subparagraph{Convocazione di una riunione esterna}
			Il \responsabilediprogetto\ ha il compito di fissare le riunioni esterne con i Proponenti o con i Committenti, contattandoli tramite la casella di posta elettronica del gruppo.\\ Il \responsabilediprogetto\ ha, inoltre, il compito di accordarsi con i Proponenti o Committenti riguardo data, orario e luogo dell'incontro, tenendo conto anche della disponibilità degli elementi del gruppo e mettendo il condizione, per quanto possibile, di far partecipare tutti alla riunione.
			\subparagraph{Gestione di una riunione esterna}
			Ad ogni riunione esterna il \responsabilediprogetto\ deve chiedere la disponibilità ai Proponenti o Committenti di registrare l'incontro, in modo tale da redigere un verbale alla fine della riunione che sia quanto più fedele possibile a ciò di cui si è discusso durante l'incontro, in modo tale che, anche se ci fossero stati dei membri assenti, questi possano disporre di un documento che riporti i temi trattati in modo esaustivo. In questo caso, come per le riunioni interne, viene nominato un Segretario che ha il compito di riascoltare l'incontro e di scriverne un verbale. In caso non fosse possibile registrare l'incontro il \responsabilediprogetto\ deve redigere il verbale della riunione esterna, avvalendosi dell'aiuto di tutti i membri del team presenti all'incontro, al fine di avere una trascrizione più fedele possibile dei contenuti.
			\paragraph{Verbale di una riunione}
			Il verbale di una riunione è un documento nel quale vengono riassunti gli argomenti trattati e nel quale vengono indicate eventuali decisioni prese nel corso della riunione stessa.\\ Il compito della stesura del verbale spetta o ad un Segretario, nominato all'inizio di una riunione interna oppure alla fine di una riunione esterna nella quale è stato dato il permesso di registrare l'incontro, oppure al \responsabilediprogetto\ con l'aiuto di tutto il gruppo, nel caso di riunione esterna senza la possibilità di registrare quanto è stato detto.\\ I verbali dovranno avere queste struttura:
			\begin{itemize}
			\item Dove e quando si è svolta la riunione;
			\item Quali membri hanno partecipato alla riunione e quali membri erano, invece, assenti;
			\item Il nome dell'eventuale Segretario oppure del \responsabilediprogetto\ che ha redatto il verbale;
			\item Tipo di riunione;
			\item Ordine del giorno;
			\item Eventuali argomenti dell'ordine del giorno non trattati;
			\item Gli interventi più significativi in ordine temporale;
			\item Suggerimenti, proposte, decisioni emerse durante la riunione;
			\item Dubbi, problematiche emerse durante la riunione;
			\item Eventuali compiti assegnati a membri del gruppo, con indicazione di nome, cognome e ruolo della persona a cui è stato assegnato;
			\item Eventuali argomenti da trattare la prossima riunione.
			\end{itemize}
		\subsubsection{Ruoli di progetto}
		Per lo sviluppo collaborativo del progetto, ai membri del gruppo saranno assegnati dei ruoli differenti che corrispondono a figure professionali. Ogni membro dovrà ricoprire ogni ruolo almeno una volta ed è necessario garantire che il ruolo di ciascun membro del gruppo non sia in conflitto con il ruolo che ha ricoperto in passato. È compito del \verificatore\ controllare che siano rispettate queste condizioni, in caso contrario dovrà avvertire il \responsabilediprogetto\ che dovrà risolvere il problema.
			\paragraph{Responsabile di progetto}
			Il \responsabilediprogetto\ rappresenta il progetto presso il fornitore e presso il committente, accentrando su di sé la responsabilità di scelta e approvazione. Per questo motivo ha responsabilità su:
			\begin{itemize}
			\item Pianificazione, coordinamento e controllo delle attività;
			\item Gestione delle risorse;
			\item Analisi e gestione dei rischi;
			\item Approvazione dei documenti;
			\item Approvazione dell'offerta economica;
			\item Convocazione delle riunioni interne;
			\item Relazioni esterne;
			\item Assegnazione delle attività a persone.
			\end{itemize}
			Per questi motivi ha il compito di:
			\begin{itemize}
			\item Assicurarsi che le attività di verifica e validazione siano svolte seguendo le \normediprogetto;
			\item Garantire il rispetto dei ruoli e dei compiti assegnati nel \pianodiprogetto;
			\end{itemize}
			\paragraph{Amministratore}
			L'\amministratore\ è responsabile del controllo e della gestione dell'ambiente di lavoro. In particolare deve preoccuparsi di:
			\begin{itemize}
			\item Equipaggiare l'ambiente di lavoro con strumenti, procedure, infrastrutture e servizi a supporto dei processi che permettano di automatizzare il più possibile le attività o parti di esse;
			\item Garantire che l'ambiente di lavoro sia completo, dotato di tutti gli strumenti necessari al progetto, ordinato e aggiornato;
			\item Controllare le versioni e configurazioni del prodotto;
			\item Gestire la documentazione, controllarne la diffusione, la disponibilità e l'archiviazione;
			\item Fornire procedure e strumenti per il monitoraggio e segnalazione per il controllo qualità;
			\item Risolvere problemi legati alla gestione dei processi tramite l'adozione di strumenti adatti.
			\end{itemize}
			L'\amministratore\ redige inoltre le \normediprogetto\, dove viene spiegato l'utilizzo degli strumenti, e deve redigere la sezione del \pianodiqualifica\ dove vengono descritti gli strumenti e i metodi atti alla verifica. L'\amministratore\ non compie scelte gestionali, ma tecnologiche concordate con il \responsabilediprogetto.
			\paragraph{Analista}
			L'\analista\ è il responsabile delle attività di analisi. Le mansioni che gli competono riguardano:
			\begin{itemize}
			\item Comprendere la natura e la complessità del problema tramite l'analisi dei bisogni e delle fonti;
			\item Classificare i requisiti;
			\item Stendere i diagrammi dei casi d'uso;
			\item Assegnare i requisiti a parti distinte del sistema;
			\item Assicurarsi che i requisiti trovati siano conformi alle richieste del committente.
			\end{itemize}
			L'\analista\ non si occupa di trovare una soluzione al problema, ma lo definisce redigendo lo \studiodifattibilita\ e l'\analisideirequisiti. Partecipa alla definizione del \pianodiqualifica\ in quanto conosce a fondo il problema e deve avere chiari i livelli di qualità richiesti, oltre alle procedure per ottenerli. 
			\paragraph{Progettista}
			Il \progettista\ è il responsabile delle attività di progettazione. I compiti a lui affidati comprendono:
			\begin{itemize}
			\item Produrre una soluzione attuabile e che sia comprensibile e soddisfacente per gli stakeholders;
			\item Effettuare scelte su aspetti progettuali che applichino al prodotto soluzioni note ed ottimizzate;
			\item Effettuare scelte su aspetti progettuali e tecnologici che rendano il prodotto facilmente manutenibile;
			\item Effettuare scelte su aspetti progettuali e tecnologici che rendano il prodotto realizzabile con costi e scadenze prefissate;
			\end{itemize}
			Il \progettista\ redige i documenti di \specificatecnica, \definizionediprodotto\ e si occupa delle sezioni del \pianodiqualifica\ relative alle metriche di verifica della programmazione.
			\paragraph{Programmatore}
			Il \programmatore\ è responsabile delle attività di codifica e delle componenti di ausilio necessarie per l'esecuzione delle prove di verifica e validazione. In particolare ha i seguenti compiti:
			\begin{itemize}
			\item Implementare in maniera rigorosa le soluzioni descritte dal \progettista;
			\item Scrivere codice che sia documentato, manutenibile e che rispetti le metriche stabilite per la scrittura del codice;
			\item Realizzare i test per la verifica e la validazione del codice stesso.
			\end{itemize}
			Il \programmatore, infine, deve occuparsi di redigere il \manualeutente.
			\paragraph{Verificatore}
			Il \verificatore\ è il responsabile delle attività di verifica. Le mansioni che gli competono sono:
			\begin{itemize}
			\item Garantire che l'attuazione delle attività sia conforme alle norme stabilite;
			\item Verificare che le attività svolte non abbiano introdotti errori;
			\item Controllare la conformità di ogni stadio del ciclo di vita del prodotto.
			\end{itemize}
			Il \verificatore\ deve redigere le sezioni del \pianodiqualifica\ riguardanti l'esito e la completezza delle verifiche effettuate.
			\paragraph{Rotazione dei ruoli}
			Ogni membro del gruppo dovrà ricoprire ciascuno dei ruoli del progetto. La pianificazione dovrà essere redatta prestando attenzione a quanto segue:
			\begin{itemize}
				\item ogni membro del gruppo non dovrà mai ricoprire un ruolo che preveda la verifica dell'operato svolto da lui in precedenza poiché questo potrebbe portare ad un conflitto di interesse;
				\item bisogna tener conto dei possibili impegni o interessi dei singoli membri del gruppo;
				\item ciascun membro dovrà assicurare l'esclusivo svolgimento del ruolo a lui assegnato.
			\end{itemize}				
			
	\subsection{Strumenti}
			
			\subsubsection{Pianificazione} \label{sec: Pianificazione Teamwork}
			Lo strumento scelto per la gestione delle attività di pianificazione di progetto è Teamwork. Le caratteristiche rilevanti di questo software sono le seguenti:
			\begin{itemize}
			\item Creazione di attività e sotto-attività assegnabili ad uno o più membri del progetto;
			\item Possibilità di creare dipendenze tra le attività;
			\item Possibilità di assegnare priorità differenti ad ogni attività;
			\item Creazione di milestones da impostare sul calendario;
			\item Creazione automatica di grafici e report, tra cui il grafico di Gantt;
			\item Dashboard che permette di aver un riepilogo dello stato di avanzamento del progetto, con segnalazione di eventuali ritardi;
			\item Strumento per la segnalazione dei rischi;
			\item Sistema di gestione delle notifiche per ogni attività svolta;
			\item Versione mobile;
			\end{itemize}		 
	Questo strumento, dopo essere stato valutato insieme ad altri strumenti quali Trello, Freedcamp e Zoho, è risultato essere il più completo.
			
			\subsubsection{Creazione dei diagrammi di Gantt}
			Lo strumento scelto per la realizzazione dei diagrammi di Gantt è GanttProject. Le principali ragioni per cui è stato scelto sono:
			\begin{itemize}
				\item gratuito;
				\item opensource;
				\item multipiattaforma;
				\item compatibile con i diagrammi generati da Teamwork;
				\item offre la possibilità di creare i diagrammi PERT;
				\item offre la possibilità di creare grafici di WBS;
				\item generazione dei diagrammi nei formati PDF, PNG, HTML.
			\end{itemize}
\end{document}