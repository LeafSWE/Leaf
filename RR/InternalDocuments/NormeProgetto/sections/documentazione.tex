\documentclass[../NormeProgetto.tex]{subfiles}

\begin{document}

\section{Documentazione}
In questa sezione sono indicati gli standard riguardanti la struttura e la stesura dei documenti prodotti.

	\subsection{Struttura dei documenti}
	Ogni documento è realizzato a partire da una struttura prestabilita che dovrà essere uguale per tutti i documenti ufficiali ad eccezione dei verbali:
	\begin{enumerate}
		\item frontespizio;
		\item informazione sul documento;
		\item diario delle modifiche;
		\item indice delle sezioni;
		\item indice delle tabelle;
		\item indice delle figure;
		\item introduzione;
		\item contenuto.
	\end{enumerate}
	L'ordine di ognuna delle sezioni è fissato. La numerazione delle prime pagine sarà quella romana, mentre dall'introduzione fino alla fine del documento quella araba.

		\subsubsection{Frontespizio}
		Questa sezione deve trovarsi nella prima pagina di ogni documento e contiene:
		\begin{enumerate}
			\item informazioni sul gruppo:
				\begin{enumerate}[a.]
					\item nome;
					\item logo;
					\item mail.
				\end{enumerate}
			\item informazioni sul progetto:
				\begin{enumerate}[a.]
					\item nome progetto;
					\item nome azienda proponente.
				\end{enumerate}
			\item informazioni sul documento:
				\begin{enumerate}[a.]
					\item nome;
					\item versione.
				\end{enumerate}
		\end{enumerate}

		\subsubsection{Informazioni sul documento}
		In questa sezione vengono indicate le principale informazioni riguardanti il documento quali:
		\begin{enumerate}
			\item versione;
			\item data di redazione;
			\item cognome e nome di coloro che hanno redatto il documento (in ordine alfabetico)
			\item cognome e nome di coloro che hanno verificato il documento (in ordine alfabetico)
			\item ambito d'uso del documento (interno od esterno)
			\item cognome e nome di coloro ai quali è destinato il documento (in ordine alfabetico)
		\end{enumerate}

		\subsubsection{Diario delle modifiche}
		Questa sezione descrive, attraverso l'utilizzo di una tabella, le modifiche che sono state apportate al documento. Ogni riga della tabella corrisponde ad una modifica effettua al documento. La struttura della riga della tabella è la seguente:

		\begin{enumerate}
			\item versione del documento;
			\item data della modifica;
			\item cognome e nome dell'autore della modifica;
			\item ruolo dell'autore della modifica nel momento in cui essa è avvenuta;
			\item sommario delle modifiche apportate
		\end{enumerate}

		Le righe della tabella sono ordinate a partire dalla data dell'ultima modifica effettuata.

		\subsubsection{Indice delle sezioni}
		L'indice delle sezioni contiene l'indice di tutti gli argomenti trattati all'interno del documento. La sua struttura è la seguente:

		\begin{enumerate}
			\item titolo dell'argomento trattato;
			\item numero di pagina.
		\end{enumerate}

		\subsubsection{Indice delle tabele}
		Questa sezione contiene l'indice delle tabele. Per ogni tabella deve essere specificato:

		\begin{enumerate}
			\item titolo della tabella;
			\item numero di pagina di riferimento.
		\end{enumerate}

		Nel caso in cui non siano presenti tabelle all'interno del documento, è possibile omettere questa sezione.

		\subsubsection{Indice delle figure}
		In questa sezione sono riportate tutte le figure presenti all'interno del documento. Per ogni figura deve essere specificato:
		\begin{itemize}
			\item nome figura;
			\item pagina di riferimento della figura.
		\end{itemize}
		Nel caso in cui non siano presenti figure all'interno del documento, è possibile omettere questa sezione.

		\subsubsection{Introduzione}
		Questa sezione deve riportare le seguenti informazioni:
		\begin{enumerate}
			\item scopo del documento;
			\item glossario;
			\item riferimenti utili:
			\begin{enumerate}[a.]
				\item riferimenti normativi;
				\item riferimenti informativi;
			\end{enumerate}
		\end{enumerate}

		\subsubsection{Contenuto}
		Questa sezione contiene il contenuto del documento. Anch'esso deve essere propriamente diviso in sezioni e sottosezioni.
	
	\subsection{Norme tipografiche}
		\subsubsection{Formattazione generale}
			\paragraph{Testatine}
			Ogni pagina di un documento, fatta eccezione per il frontespizio, deve contenere la testina. Essa è composta da:
			\begin{itemize}
				\item logo del gruppo, posizionato in alto a sinistra;
				\item nome del documento, posizionato in alto a destra.
			\end{itemize}
			\paragraph{Piè pagina}
			Ogni pagina di un documento, deve contenere il piè pagina. Esso contiene:
			\begin{itemize}
				\item numero della pagina, posizionato al centro.
			\end{itemize}
			\paragraph{Orfani e vedove}
			Si considerano vedova, la riga di un paragrafo che inizia alla fine di una pagina, mentre si considera orfana, la riga di un paragrafo che finisce all'inizio di una pagina. I documenti dovranno essere redatti cercando di evitare il più possibile queste due tipologie di righe poiché risultato essere poco gradevoli. 
		\subsubsection{Caratteri}
			
			\paragraph{Virgolette}
			
			\begin{itemize}
				\item \textbf{Virgolette alte singole ` ' :} devono essere utilizzate per racchiudere un singolo carattere;
				\item \textbf{Virgolette alte doppie `` '' :} devono essere utilizzate per racchiudere nomi di file o comandi. Possono essere utilizzate anche per descrivere parole il cui ordine è prestabilito;
				\item \textbf{Virgolette basse `<<' `>>' :} devono essere utilizzate per racchiudere citazioni.
			\end{itemize}
			Non sono ammessi ulteriori casi d'uso per le virgolette.
			
			\paragraph{Parentesi}
			
			\begin{itemize}
				\item \textbf{Tonde:} possono essere utilizzate per descrivere esempi o per fornire dei sinonimi. Sono le uniche parentesi ammesse all'interno di una frase.
				\item \textbf{Quadre:} possono rappresentare uno standard ISO oppure uno stato relativo ad un ticket. 
			\end{itemize}
			
			\paragraph{Punteggiatura}
			La punteggiatura  deve essere sempre utilizzata attentamente per cercare di rendere il discorso il più chiaro e coeso possibile. Non sono ammesse spaziature prima dell'utilizzo di un carattere di punteggiatura. L'utilizzo del punto è necessario per indicare la fine di un concetto e poter iniziarne un altro.
			
			\paragraph{Numeri}
			I numeri all'interno dei documenti devono essere formattati seguendo lo standard [SI/ISO 31-0]. Esso prevede che la parte frazionaria sia separata da quella decimale utilizzando la virgola. I numeri la cui parte intera supera le tre cifre, devono essere scritti raggruppando in gruppi di tre le cifre di cui è composta la parte intera, partendo dalla cifra meno significativa e separandoli con uno spazio unificatore.
			\paragraph{Lettere}
			Cosa dovrei scrivere delle lettere???
			
		\subsubsection{Stile del testo}

			\paragraph{Corsivo}
			Il corsivo va utilizzato per riportare le seguenti informazioni:
			\begin{itemize}
				\item nome di un documento;
				\item nome di un ruolo;
				\item percorsi di cartelle.
			\end{itemize}						
			
			\paragraph{Grassetto}
			Il grassetto va utilizzato per riportare le seguenti informazioni:
			\begin{itemize}
				\item titoli;
				\item parole su cui è utile focalizzare l'attenzione del lettore all'interno di un argomento;
				\item parole chiave all'interno di elenchi.
			\end{itemize}						
			
			\paragraph{Sottolineato}
			La sottolineatura è indicata qualora si voglia evidenziare l'importanza di una parola all'interno di una frase.					
			
			\paragraph{Monospace}
			Lo stile monospace va applicato nel caso in cui si vogliano riportare all'interno di un documento	comandi oppure parti di codice.
			
			\paragraph{Glossario}
			Questo stile va applicato per tutte le parole che hanno una corrispondenza all'interno del glossario. Ogni parola presente nel glossario deve essere seguita da un pedice contente il carattere `g' scritto in corsivo. Non si applica questa regola nei casi in cui la parola compaia all'interno di titoli, percorsi, nomi di cartelle, comandi o parti di codice.
			
		\subsubsection{Composizione del testo}
			\paragraph{Elenchi}
			Le norme che regolano un elenco sono le seguenti:
			\begin{itemize}
				\item la prima parola di un elenco deve essere maiuscola, fatta eccezione nel caso in cui l'elenco inizi con il carattere `:';
				\item ogni elemento dell'elenco, tranne l'ultimo, deve terminare con il carattere `;'. È fatta eccezione nel caso in cui l'elemento sia composto da più frasi, allora è permesso il `.'.
				\item l'ultimo elemento di un elenco deve sempre terminare con il carattere `.'.
			\end{itemize}
È necessario usare elenchi numerato quando è l'ordine degli elementi è rilevante. Per gli elenchi numerati valgono le seguenti regole:
			\begin{itemize}
				\item nel primo livello si usano numeri interni a partire da uno;
				\item nel secondo livello si usano lettere dell'alfabeto a partire. dalla `a'.
			\end{itemize}
Gli elenchi puntati servono per descrivere elementi di cui non è importante l'ordine espositivo. Essi seguono le seguenti regole:
			\begin{itemize}
				\item nel primo livello bisogna utilizzare cerchi neri pieni;
				\item nel secondo livello trattini neri.
			\end{itemize}
			
			
			\paragraph{Descrizioni}
			\paragraph{Note a piè pagina}
		\subsubsection{Formati}
			\paragraph{Data e ora}
			\paragraph{URI}
			\paragraph{Sigle}
			\paragraph{Ruoli di progetto}
			\paragraph{Fasi del progetto}
			\paragraph{Revisioni}
			\paragraph{Nomi}
				\subparagraph{Nome del gruppo}
				\subparagraph{Nome del progetto}
				\subparagraph{Nome del proponente}
				\subparagraph{Nome del committente}
				\subparagraph{Nome proprio}
				\subparagraph{Nome di un file}
				\subparagraph{Nome di un documento}
		\subsubsection{Componenti grafiche}
			\paragraph{Immagini}
			\paragraph{Tabelle}
	\subsection{Tipologie di documenti}
		\subsubsection{Documenti formali}
		\subsubsection{Documenti informali}
		\subsubsection{Glossario}
		\subsubsection{Verbali}
			\paragraph{Verbali di riunioni interne}
			\paragraph{Verbali di riunioni esterne}
	\subsection{Versionamento dei documenti} \label{sec:Versionamento dei documenti}
	\subsection{Avanzamento di un documento} 
	\subsection{Strumenti}
		\subsubsection{Latex}
			\paragraph{Template}
			\paragraph{Comendi personalizzati}
			\paragraph{Rilevamento errori ortografici}
			
			
			
\end{document}
