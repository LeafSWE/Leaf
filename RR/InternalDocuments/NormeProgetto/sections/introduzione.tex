\documentclass[../NormeProgetto.tex]{subfiles}


\begin{document}
\section{Introduzione}
	\subsection{Scopo del documento}
	Questo documento ha lo scopo di stabilire le norme che tutti i membri del gruppo \leaf dovranno rispettare durante lo sviluppo del progetto \textbf{CLIPS}, gli strumenti che dovranno utilizzare e le procedure che dovranno seguire. \\
	Tutti i membri del gruppo sono tenuti a prendere visione di tale documento e rispettare le norme in esso contenute, al fine di garantire uniformità nello svolgimento del progetto, garantendo maggiore efficienza ed efficacia alle attività. \\
	Il documento rivolge l'attenzione sui seguenti contenuti:
	\begin{itemize}
	\item organizzazione della comunicazione all'interno del gruppo e verso l'esterno;
	\item modalità di stesura dei documenti;
	\item descrizione delle metodologie di lavoro durante lo sviluppo del progetto;
	\item modalità di gestione e utilizzo del repository\g;
	\item strumenti per la gestione dell'ambiente di lavoro.
	\end{itemize}

	\subsection{Scopo del progetto}
	Ancora da decidere
	
	\subsection{Glossario} \label{sec:Glossario}
	Allo scopo di rendere più semplice e chiara la comprensione dei documenti viene allegato il \textit{Glossario v1.00} nel quale verranno raccolte le spiegazioni di:
	\begin{itemize}
	\item terminologia tecnica;
	\item termini ambigui;
	\item abbreviazioni;
	\item acronimi.
	\end{itemize}
	Per evidenziare un termine presente in tale documento, esso verrà marcato con il pedice \g.
	
	\subsection{Riferimenti utili}
		\subsubsection{Riferimenti normativi}
		\begin{itemize}
			\item Capitolati d'appalto: \\\url{http://www.math.unipd.it/~tullio/IS-1/2015/Progetto/Capitolati.html}
		\end{itemize}
		\subsubsection{Riferimenti informativi}
		\begin{itemize}
			\item Materiale di riferimento del corso di Ingegneria del Software: \\\url{http://www.math.unipd.it/~tullio/IS-1/2015};
			\item Portable Document Format:  \\\url{http://en.wikipedia.org/wiki/Portable_Document_Format};
			\item Rappresentazione dei numeri: \\\url{https://en.wikipedia.org/wiki/ISO_31-0#Numbers};
			\item Rappresentazione di date e orari: \\\url{https://en.wikipedia.org/wiki/ISO_8601};
			\item Unicode: \\\url{http://en.wikipedia.org/wiki/Unicode};
			\item Bluetooth low energy: \\\url{https://en.wikipedia.org/wiki/Bluetooth_low_energy};
		\end{itemize}
\end{document}

