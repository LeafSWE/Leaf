\subsection{Documentazione}
In questa sezione sono indicati gli standard riguardanti la struttura e la stesura dei documenti prodotti.

	\subsubsection{Struttura dei documenti}
	Ogni documento è realizzato a partire da una struttura prestabilita che dovrà essere uguale per tutti i documenti ufficiali ad eccezione dei verbali:
	\begin{enumerate}
		\item frontespizio;
		\item informazione sul documento;
		\item diario delle modifiche;
		\item indice delle sezioni;
		\item indice delle tabelle;
		\item indice delle figure;
		\item introduzione;
		\item contenuto.
	\end{enumerate}
	L'ordine di ognuna delle sezioni è fissato. La numerazione delle prime pagine sarà quella romana, mentre dall'introduzione fino alla fine del documento quella araba.

		\paragraph{Frontespizio}
		Questa sezione deve trovarsi nella prima pagina di ogni documento e contiene:
		\begin{enumerate}
			\item informazioni sul gruppo:
				\begin{enumerate}[a.]
					\item nome;
					\item logo;
					\item mail.
				\end{enumerate}
			\item informazioni sul progetto:
				\begin{enumerate}[a.]
					\item nome progetto;
					\item nome azienda proponente.
				\end{enumerate}
			\item informazioni sul documento:
				\begin{enumerate}[a.]
					\item nome;
					\item versione.
				\end{enumerate}
		\end{enumerate}

		\paragraph{Informazioni sul documento}
		In questa sezione vengono indicate le principale informazioni riguardanti il documento quali:
		\begin{enumerate}
			\item versione;
			\item data di redazione;
			\item cognome e nome di coloro che hanno redatto il documento (in ordine alfabetico);
			\item cognome e nome di coloro che hanno verificato il documento (in ordine alfabetico)
			\item ambito d'uso del documento (interno oppure esterno);
			\item cognome e nome di coloro ai quali è destinato il documento (in ordine alfabetico).
		\end{enumerate}

		\paragraph{Diario delle modifiche}
		Questa sezione descrive, attraverso l'utilizzo di una tabella, le modifiche che sono state apportate al documento. Ogni riga della tabella corrisponde ad una modifica effettua al documento. La struttura della riga della tabella è la seguente:

		\begin{enumerate}
			\item versione del documento;
			\item data della modifica;
			\item cognome e nome dell'autore della modifica;
			\item ruolo dell'autore della modifica nel momento in cui essa è avvenuta;
			\item sommario delle modifiche apportate.
		\end{enumerate}

		Le righe della tabella sono ordinate a partire dalla data dell'ultima modifica effettuata, in ordine cronologico inverso.

		\paragraph{Indice delle sezioni}
		L'indice delle sezioni contiene l'indice di tutti gli argomenti trattati all'interno del documento. La sua struttura è la seguente:

		\begin{enumerate}
			\item titolo dell'argomento trattato;
			\item numero di pagina.
		\end{enumerate}

		\paragraph{Indice delle tabelle}
		Questa sezione contiene l'indice delle tabelle. Per ogni tabella deve essere specificato:

		\begin{enumerate}
			\item titolo della tabella;
			\item numero di pagina di riferimento.
		\end{enumerate}

		Nel caso in cui non siano presenti tabelle all'interno del documento, è possibile omettere questa sezione.

		\paragraph{Indice delle figure}
		In questa sezione sono riportate tutte le figure presenti all'interno del documento. Per ogni figura deve essere specificato:
		\begin{itemize}
			\item nome figura;
			\item pagina di riferimento della figura.
		\end{itemize}
		Nel caso in cui non siano presenti figure all'interno del documento, è possibile omettere questa sezione.

		\paragraph{Introduzione}
		Questa sezione deve riportare le seguenti informazioni:
		\begin{enumerate}
			\item scopo del documento;
			\item glossario;
			\item riferimenti utili:
			\begin{enumerate}[a.]
				\item riferimenti normativi;
				\item riferimenti informativi.
			\end{enumerate}
		\end{enumerate}

		\paragraph{Contenuto}
		Questa sezione contiene il contenuto del documento. Anch'esso deve essere propriamente diviso in sezioni e sottosezioni.
	
	\subsubsection{Norme tipografiche} \label{sec:Norme tipografiche}
		\paragraph{Formattazione generale}
			\subparagraph{Testatine}
			Ogni pagina di un documento, fatta eccezione per il frontespizio, deve contenere la testina. Essa è composta da:
			\begin{itemize}
				\item logo del gruppo, posizionato in alto a sinistra;
				\item nome del documento, posizionato in alto a destra.
			\end{itemize}
			\subparagraph{Piè pagina}
			Ogni pagina di un documento, deve contenere il piè pagina. Esso contiene:
			\begin{itemize}
				\item numero della pagina, posizionato al centro.
			\end{itemize}
			\subparagraph{Orfani e vedove}
			Si considerano vedova, la riga di un paragrafo che inizia alla fine di una pagina, mentre si considera orfana, la riga di un paragrafo che finisce all'inizio di una pagina. I documenti dovranno essere redatti cercando di evitare il più possibile queste due tipologie di righe poiché risultano poco gradevoli. 
		\paragraph{Caratteri}
			
			\subparagraph{Virgolette}
			
			\begin{itemize}
				\item \textbf{Virgolette alte singole ` ' :} devono essere utilizzate per racchiudere un singolo carattere;
				\item \textbf{Virgolette alte doppie `` '' :} devono essere utilizzate per racchiudere nomi di file, comandi o collegamenti a sezioni interne dello stesso documento;
				\item \textbf{Virgolette basse `<<' `>>' :} devono essere utilizzate per racchiudere citazioni.
			\end{itemize}
			Non sono ammessi ulteriori casi d'uso per le virgolette.
			
			\subparagraph{Parentesi}
			
			\begin{itemize}
				\item \textbf{Tonde:} possono essere utilizzate per descrivere esempi, per fornire dei sinonimi oppure per dare delle precisazioni. Sono le uniche parentesi ammesse all'interno di una frase.
				\item \textbf{Quadre:} possono rappresentare uno standard ISO oppure uno stato relativo ad un ticket. 
			\end{itemize}
			
			\subparagraph{Punteggiatura}
			La punteggiatura  deve essere sempre utilizzata attentamente per cercare di rendere il discorso il più chiaro e coeso possibile. Non sono ammesse spaziature prima dell'utilizzo di un carattere di punteggiatura. L'utilizzo del punto è necessario per indicare la fine di un concetto e poter iniziarne un altro.
			
			\subparagraph{Numeri}
			I numeri all'interno dei documenti devono essere formattati seguendo lo standard [SI/ISO 31-0]. Esso prevede che la parte frazionaria sia separata da quella decimale utilizzando la virgola. I numeri la cui parte intera supera le tre cifre, devono essere scritti raggruppando in gruppi di tre le cifre di cui è composta la parte intera, partendo dalla cifra meno significativa e separandoli con uno spazio unificatore.
			
		\paragraph{Stile del testo}

			\subparagraph{Corsivo}
			Il corsivo va utilizzato per riportare le seguenti informazioni:
			\begin{itemize}
				\item nome di un documento;
				\item nome di un ruolo;
				\item percorsi di cartelle.
			\end{itemize}						
			
			\subparagraph{Grassetto}
			Il grassetto va utilizzato per riportare le seguenti informazioni:
			\begin{itemize}
				\item titoli;
				\item parole su cui è utile focalizzare l'attenzione del lettore all'interno di un argomento;
				\item parole chiave all'interno di elenchi.
			\end{itemize}						
			
			\subparagraph{Sottolineato}
			La sottolineatura è indicata qualora si voglia evidenziare l'importanza di una parola all'interno di una frase.					
			
			\subparagraph{Monospace}
			Lo stile monospace va applicato nel caso in cui si vogliano riportare all'interno di un documento	comandi oppure parti di codice.
			
			\subparagraph{Glossario}\label{sec:Formattazione termini nel glossario}
			Questo stile va applicato per tutte le parole che hanno una corrispondenza all'interno del glossario. Ogni parola presente nel glossario deve essere seguita da un pedice contente il carattere `g' scritto in corsivo. Non si applica questa regola nei casi in cui la parola compaia all'interno di titoli, percorsi, nomi di cartelle, comandi o parti di codice.
			
		\paragraph{Composizione del testo}
			\subparagraph{Elenchi}
			Le norme che regolano un elenco sono le seguenti:
			\begin{itemize}
				\item la prima parola di un elenco deve essere maiuscola, fatta eccezione nel caso in cui l'elenco inizi con il carattere `:';
				\item ogni elemento dell'elenco, tranne l'ultimo, deve terminare con il carattere `;'. È fatta eccezione nel caso in cui l'elemento sia composto da più frasi, allora è permesso il `.'.
				\item l'ultimo elemento di un elenco deve sempre terminare con il carattere `.'.
			\end{itemize}
È necessario usare elenchi numerato quando è l'ordine degli elementi è rilevante. Per gli elenchi numerati valgono le seguenti regole:
			\begin{itemize}
				\item nel primo livello si usano numeri interni a partire da uno;
				\item nel secondo livello si usano lettere dell'alfabeto a partire dalla `a'.
			\end{itemize}
Gli elenchi puntati servono per descrivere elementi di cui non è importante l'ordine espositivo. Essi seguono le seguenti regole:
			\begin{itemize}
				\item nel primo livello bisogna utilizzare cerchi neri pieni;
				\item nel secondo livello trattini neri.
			\end{itemize}
			
			
			\subparagraph{Descrizioni}
			Nel caso in cui si voglia strutturare la descrizione di qualcosa sottoforma di elenco è necessario utilizzare il costrutto \LaTeX\ \texttt{ \textbackslash begin$\left\{description\right\}$ \textbackslash item[description] \textbackslash end$\left\{description\right\}$}.			
			
			\subparagraph{Note a piè pagina}
			Le note a piè pagina seguono le seguenti regole:
			\begin{itemize}
				\item la loro numerazione è progressiva all'interno di tutto il documento;
				\item devono essere scritte una volta sola;
				\item il primo carattere di ogni nota deve essere maiuscolo. Fanno eccezione i casi in cui la parola sia un acronimo. In questo caso bisogna seguire le regole di formattazione di tale acronimo.
			\end{itemize}
			
		\paragraph{Formati}
			\subparagraph{Date}
			La formattazione delle date segue lo standard [ISO 8601]. Tale standard prevede che una data sia scritta secondo il seguente formalismo: AAAA-MM-GG. Questa rappresentazione va letta nel seguente modo:
			\begin{itemize}
				\item AAAA: numero a quattro cifre che rappresenta l'anno;
				\item MM: numero a due cifre che rappresenta il mese;
				\item GG: numero a due cifre che rappresenta il giorno.
			\end{itemize}
			Nei casi in cui risulti possibile esprimere i mesi o giorni omettendo una cifra, è necessario anteporre uno zero davanti a tale cifra.
			
			\subparagraph{Orari}
			La formattazione degli orari segue lo standard [ISO 8601]. Tale standard prevede che gli orari siano scritti secondo il seguente formalismo: hh:mm. Questa rappresentazione va letta nel seguente modo:
			\begin{itemize}
				\item hh: numero a due cifre che rappresenta il numero di ore trascorse dalla mezzanotte;
				\item mm: numero a due cifre che rappresenta i minuti. 
			\end{itemize}						
			Nei casi in cui risulti possibile esprimere le ore o i minuti omettendo una cifra, è necessario anteporre uno zero davanti a tale cifra.
			
			\subparagraph{URI}
			La stile utilizzato per rappresentare un URI è il corsivo ed il testo deve essere di colore blu.
			 
			\subparagraph{Sigle}
			È possibile fare riferimento a ruoli, documenti e revisioni pianificate utilizzando le seguenti sigle:
			\begin{itemize}
				\item Rp (\responsabilediprogetto);
				\item Am (\amministratore);
				\item An (\analista);
				\item Pt (\progettista);
				\item Pm (\programmatore);
				\item Ve (\verificatore).
				
				\item AR (\analisideirequisiti);
				\item GL (\glossario);
				\item NP (\normediprogetto);
				\item PP (\pianodiprogetto);
				\item PQ (\pianodiqualifica);
				\item SF (\studiodifattibilita);
				\item ST (\specificatecnica);
				
				\item RR (\revisionedeirequisiti);
				\item RA (\revisionediaccettazione);
				\item RP (\revisionediprogettazione);
				\item RQ (\revisionediqualifica);
			\end{itemize}
			L'utilizzo di tale sigle è permesso solo all'interno di:
			\begin{itemize}
				\item Tabelle;
				\item Diagrammi (immagini);
				\item Didascalie di tabelle e immagini;
				\item Note a piè di pagina.
			\end{itemize}
			\subparagraph{Ruoli di progetto}
			Quando si fa riferimento ad un ruolo di progetto bisogna adottare lo stile corsivo e la prima lettera deve essere maiuscola.			
			
			\subparagraph{Fasi del progetto}
			Quando si fa riferimento ad una fase del progetto bisogna adottare lo stile grassetto e la prima lettera deve essere maiuscola.		
			
			\subparagraph{Revisioni}
			Quando si fa riferimento ad una revisione bisogna adottare lo stile grassetto e la prima lettera deve essere maiuscola.
			

			\subparagraph{Nomi}
				
				\subsubparagraph{Nome del gruppo}
				Il nome del gruppo deve essere sempre indicato tramite il comando \LaTeX\ \texttt{\textbackslash leaf}, in modo tale da avere sempre la stessa formattazione.
				
				\subsubparagraph{Nome del progetto}
				Il nome del progetto deve essere sempre indicato tramite il comando \LaTeX\ \texttt{\textbackslash progetto}, in modo tale da avere sempre la stessa formattazione.

				\subsubparagraph{Nome del proponente}
				Il nome del proponente deve essere sempre indicato tramite il comando \LaTeX\ \texttt{\textbackslash proponente}, in modo tale da avere sempre la stessa formattazione.

				\subsubparagraph{Nome del committente}
				Il nome del committente deve essere sempre indicato tramite il comando \LaTeX\ \texttt{\textbackslash committente}, in modo tale da avere sempre la stessa formattazione.

				\subsubparagraph{Nome proprio}
				Ogniqualvolta si intende utilizzare un nome proprio si deve scrivere prima il cognome e successivamente il nome. Quando si fa riferimento a disporre i nomi in ordine alfabetico l'ordine è dettato dalle lettere del cognome, a meno di ulteriori specificazioni.
				
				\subsubparagraph{Nome di un file}
				I nomi dei file vanno formattati utilizzando lo stile monospace e devono essere racchiusi dalle doppie virgolette alte.
				
				\subsubparagraph{Nome di un documento}
			Quando si fa riferimento ad un documento bisogna adottare lo stile corsivo e la prima lettera deve essere maiuscola. I  nomi dei documenti devono essere racchiusi dalle doppie virgolette alte.
				
		\paragraph{Componenti grafiche}
			\subparagraph{Immagini}
			L'utilizzo delle immagini all'interno di un documento è regolamentato secondo quanto segue:
			\begin{itemize}
				\item i formati ammessi per le immagini sono il PNG e il PDF;
				\item devono essere numerate in ordine crescente;
				\item devono essere seguite da una breve descrizione;
				\item deve essere presente un riferimento all'immagine all'interno dell'indice immagini.
			\end{itemize}
			\subparagraph{Tabelle}
			L'utilizzo delle tabelle all'interno di un documento è regolamentato secondo quanto segue:
			\begin{itemize}
				\item devono essere numerate in ordine crescente;
				\item devono essere seguite da una breva descrizione;
				\item devono essere presente un riferimento alla tabella all'interno dell'indice delle tabelle.
			\end{itemize}
			
			Valutare colori tabelle?????!!!! Vedi ProTech.			
			
	\subsubsection{Tipologie di documenti}
		\paragraph{Documenti formali}
		I documenti formali possono essere descritti secondo quanto segue:
		\begin{itemize}
			\item sono documenti approvati dal responsabile di progetto;
			\item eventuali modifiche ad un documento formale, lo rendono informale;
			\item sono gli unici documenti che possono essere distribuiti all'esterno del team di progetto;
		\end{itemize}
		
		\paragraph{Documenti informali}
		I documenti informali possono essere descritti secondo quanto segue:
		\begin{itemize}
			\item sono documenti non ancora approvati dal responsabile di progetto;
			\item possono essere distribuiti solamente all'interno del team di progetto;
			\item possono essere sottoposti a revisione; 
		\end{itemize}
		
		\paragraph{Glossario} \label{sec:Glossario}
		Il glossario nasce dall'esigenza di chiarire il significato ambiguo che possono avere certe parole all'interno di determinati contesti. Al suo interno saranno quindi presenti alcune parole, prese dai documenti, che hanno le seguenti caratteristiche:
		\begin{itemize}
			\item trattano argomenti tecnici;
			\item trattano argomenti poco conosciuti o che possono scatenare ambiguità;
			\item rappresentano delle sigle;
		\end{itemize}
		Il glossario deve essere strutturato secondo quanto segue:
		\begin{itemize}
			\item i termini devono seguire l'ordine lessicografico;
			\item ogni termine deve essere seguito da una spiegazione chiara e concisa del significato del termine stesso. Questa spiegazione non deve essere in alcun modo ambigua.
		\end{itemize}
		Per evitare confusione, la stesura del glossario deve avvenire in maniera parallela alla stesura dei documenti . Al fine di evitare dimenticanze, è ammesso inserire un termine all'interno del glossario senza inserirne immediatamente la spiegazione. È comunque doveroso completare la spiegazione non appena possibile. \\ All'interno dei documenti i termini presenti nel glossario devono essere marcati con il pedice \g (\hyperref[sec:Formattazione termini nel glossario]{Regole per la formattazione di termini nel Glossario}).
		
		\paragraph{Verbali}
		Lo scopo di un verbale è di riassumere, cercando di essere il più possibile fedeli, ciò che che è stato discusso durante un incontro. È previsto che ad ogni riunione tra i membri del gruppo e/o soggetti esterni sia redatto un verbale. Il verbale è soggetto ad un'unica stesura e non può subire modifiche.
			
			\subparagraph{Verbali di riunioni interne}
			Si definisce interna una riunione che coinvolge solamente i membri del gruppo. Il verbale per questo tipo di incontro è da considerarsi di carattere informale. 
			
			Il verbale deve essere redatto seguendo la seguente struttura:
			\begin{enumerate}
				\item frontespizio;
				\item una sezione ``Estremi del Verbale'' contente le seguenti informazioni:
				\begin{enumerate}[a.]
					\item data;
					\item luogo, a meno di incontri telematici;
					\item partecipanti.
				\end{enumerate}
				\item introduzione contenente le motivazioni per le quali è stato richiesto l'incontro;
				\item ordine del giorno;
				\item verbale della riunione.
			\end{enumerate}

			\subparagraph{Verbali di riunioni esterne}
			Si definisce esterna una riunione che avviene tra i membri del team e soggetti esterni. Il verbale redatto per questo tipo di incontro è da considerarsi come parte integrante della documentazione ufficiale e per questo motivo può avere un valore normativo o fornire nuovi requisiti. È previsto che ad ogni incontro venga nominata una persona che si occupi della sua stesura.
			
			Il verbale deve essere redatto seguendo la seguente struttura:
			\begin{enumerate}
				\item frontespizio;
				\item indice;
				\item una sezione ``Estremi del Verbale'' contente le seguenti informazioni:
				\begin{enumerate}[a.]
					\item data;
					\item ora;
					\item luogo, a meno di incontri telematici;
					\item partecipanti.
				\end{enumerate}
				\item introduzione contenente le motivazioni per le quali è stato richiesto l'incontro;
				\item una sezione dedicata alle domande poste e alle risposte ottenute durante l'incontro.
			\end{enumerate}
			
	\subsubsection{Versionamento dei documenti} \label{sec:Versionamento dei documenti}
	Tutti i documenti, ad eccezione dei verbali, sono sottoposti a versionamento. Il versionamento prevede che la versione di un documento venga incrementata ogniqualvolta avvengano delle modifiche all'interno del documento stesso.
	
	La sintassi che indica la versione di un documento è la seguente: vX.YY. La sua interpretazione è la seguente:
	\begin{itemize}
		\item 'v' è un carattere che si riferisce alla parola versione;
		\item X è un numero che indica quante volte è stato formalizzato il documento;
		\item YY è un numero a due cifre che indica quante modifiche sono state effettuate al documento dalla sua ultima formalizzazione.
	\end{itemize}
	\subsubsection{Avanzamento di un documento} 
		\paragraph{Regole di avanzamento di versione}
		L'avanzamento di versione avviene secondo le seguenti regole:
		\begin{itemize}
			\item X inizia da 0 e viene incrementato di una unità nel momento in cui il \responsabilediprogetto\ formalizza il documento;
			\item YY inizia da 00 e viene incrementato di una unità ad ogni modifica che viene effettua al documento. Ogni volta che il documento viene formalizzato riparte da 00.
		\end{itemize}
		La prima versione di ogni documento è indicata dalla versione v0.01.		
		
		\paragraph{Procedura per la formalizzazione di un documento}
		La formalizzazione di un documento segue la seguente procedura:
		\begin{enumerate}
			\item il documento viene redatto da coloro che sono incaricati della sua stesura ed eventuale correzione di errori;
			\item il \responsabilediprogetto\ assegna uno o più Verificatori al documento i quali dovranno occuparsi di controllare la correttezza del documento stesso;
			\item se i Verificatori riscontrano anomalie si ritorna al punto 1 altrimenti il documento viene consegnato al \responsabilediprogetto;
			\item il \responsabilediprogetto\ decide se approvare il documento, e quindi formalizzarlo, oppure se rifiutarlo, e quindi ritornare al punto 1.
		\end{enumerate}
	\subsubsection{Strumenti}
		\paragraph{Latex}
		La stesura dei documenti deve essere effettuata utilizzando il linguaggio di markup latex. È stato scelto questo strumento poiché permette una facile separazione tra formattazione e presentazione. La scelta dell'editor da utilizzare è lasciata libera ai membri del gruppo.
		
			\subparagraph{Template}
			Per poter creare omogeneità tra i documenti, è stato creato un template latex nel quale sono state definite tutte le regole tipografiche da applicare al documento. Questo permette di poter scrivere i documenti senza dover tener conto della loro formattazione.
			\subparagraph{Comandi personalizzati}
			Sono stati definiti dei comandi latex personalizzati al fine di poter rendere più semplice ed immediata l'applicazione delle norme tipografiche. Questi comandi si occupano delle corretta formattazione del testo secondo le norme che sono state definite. La lista dei comandi è presente nella sezione formati. (DA VEDERE!!!)
			\subparagraph{Rilevamento errori ortografici}
			(STRUMENTO DA SCEGLIERE)
