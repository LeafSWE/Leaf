
%-------------------INIZIO comandi personalizzati-----------------------------------------
\newcommand{\leaf}{\textit{Leaf}}
\newcommand{\mail}[1]{\href{mailto:#1}{\texttt{#1}}}
\newcommand{\mailleaf}{\mail{leaf.gruppo@gmail.com}}
\newcommand{\mailinglist}{\mail{nome@mailing.list}}
\newcommand{\responsabilediprogetto}{\textit{Responsabile di progetto}}
\newcommand{\amministratore}{\textit{Amministratore}}
\newcommand{\analista}{\textit{Analista}}
\newcommand{\analisti}{\textit{Analisti}}
\newcommand{\progettista}{\textit{Progettista}}
\newcommand{\programmatore}{\textit{Programmatore}}
\newcommand{\verificatore}{\textit{Verificatore}}
\newcommand{\verificatori}{\textit{Verificatori}}
\newcommand{\pianodiprogetto}{\textit{Piano di progetto}}
\newcommand{\pianodiqualifica}{\textit{Piano di qualifica}}
\newcommand{\normediprogetto}{\textit{Norme di progetto}}
\newcommand{\studiodifattibilita}{\textit{Studio di fattibilità}}
\newcommand{\analisideirequisiti}{\textit{Analisi dei requisiti}}
\newcommand{\specificatecnica}{\textit{Specifica tecnica}}
\newcommand{\definizionediprodotto}{\textit{Definizione di prodotto}}
\newcommand{\manualeutente}{\textit{Manuale utente}}
\newcommand{\glossario}{\textit{Glossario}}
\newcommand{\revisionedeirequisiti}{\textbf{Revisione dei requisiti}}
\newcommand{\revisionediaccettazione}{\textbf{Revisione di accettazione}}
\newcommand{\revisionediprogettazione}{\textbf{Revisione di progettazione}}
\newcommand{\revisionediqualifica}{\textbf{Revisione di qualifica}}
\newcommand{\g}{\tiny \ped{|G|} \normalsize}

\newcommand{\frmdate}[3]{#3-#2-#1}
\newcommand{\frmtime}[2]{#1:#2}
\newcommand{\frmURI}[1]{\color{blue}\textit{#1}}
\newcommand{\frmspath}[1]{\texttt{#1}}
\newcommand{\frmrole}[1]{\textit{#1}}
\newcommand{\frmdoc}[1]{\textit{“#1”}}
\newcommand{\frmfile}[1]{\texttt{“#1”}}
\newcommand{\frmrev}[1]{\textbf{#1}}
\newcommand{\frmphase}[1]{\textbf{#1}}

\newcommand{\progetto}{Clips}

%-------------------INIZIO creazione subsubparagraph--------------------------------------
\makeatletter
\newcounter{subsubparagraph}[subparagraph]
\def\toclevel@subsubparagraph{6}
\renewcommand\thesubsubparagraph{%
  \thesubparagraph.\@arabic\c@subsubparagraph}
\newcommand\subsubparagraph{%
  \@startsection{subsubparagraph}    % counter
    {6}                              % level
    {\parindent}                     % indent
    {3.25ex \@plus 1ex \@minus .2ex} % beforeskip
    {-1em}                           % afterskip
    {\normalfont\normalsize\bfseries}}
\newcommand\l@subsubparagraph{\@dottedtocline{6}{13.5em}{5em}}
\newcommand{\subsubparagraphmark}[1]{}
\setcounter{tocdepth}{6}
\setcounter{secnumdepth}{6} % aggiunge contatore ai paragrafi
\makeatother
%-------------------FINE creazione subsubparagraph--------------------------------------

%-------------------FINE comandi personalizzati-----------------------------------------