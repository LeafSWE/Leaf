\documentclass[../PianoDiQualifica.tex]{subfiles}

\begin{document}
\section{Qualit�}
	\subsection{Qualit� di processo}
	Un processo per essere classificato ha bisogno di essere misurato tramite dei parametri. Uno di questi � la qualit� di processo. Assicurare la qualit� dei processi � indispensabile durante lo svolgimento del progetto per le seguenti
	ragioni:
	\begin{itemize}
	\item aiuta ad ottimizzare l'uso delle risorse;
	\item fa in modo che i costi siano maggiormente contenuti;
	\item migliora la stima dei rischi e degli impegni.
	\end{itemize}
	Per il gruppo avere qualit� di processo significa avere processi che possono essere misurati tramite strumenti e continuamente migliorati al fine di raggiungere il massimo livello di qualit�. Ogni processo per essere misurato e migliorato ha bisogno di essere sottoposto a processi di verifica che hanno il compito di valutare il livello di qualit� raggiunto e indicare gli aspetti critici da migliorare.\\
	Il modello che il gruppo usa per valutare il grado di maturit� dei processi � il \textbf{Capability Maturity Model} che permette di dare una classificazione dei processi e fornire istruzioni su come migliorarli.
	\subsection{Qualit� di prodotto}
	Il gruppo si prefigge di mantenere la stessa qualit� sia nei processi che nei prodotti.\\
	Per garantire la migliore qualit� del prodotto anche il processo da cui proviene deve avere una buona qualit�.
	Il gruppo per mantenere la qualit� del prodotto cercher� di seguire al meglio lo standard di qualit� \textbf{ISO/IEC 9126}.\\
	Il gruppo si impegna dunque a garantire le seguenti caratteristiche:
	\begin{itemize}
	\item il prodotto permette agli utenti di utilizzare le funzionalit� in maniera semplice ed efficace;
	\item il prodotto fornisce prestazioni accettabili;
	\item il prodotto garantisce un funzionamento senza interruzioni;
	\item il prodotto � facilmente installabile;
	\item il prodotto possiede le funzionalit� descritte all'interno dei requisiti minimi.
	\end{itemize}
\end{document}