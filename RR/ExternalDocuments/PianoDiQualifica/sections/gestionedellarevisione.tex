\documentclass[../PianoDiQualifica.tex]{subfiles}

\begin{document}
\section{Gestione della revisione}
	\subsection{Definizione e comunicazione delle anomalie}
	Un'anomalia si genera nel momento in cui si verificano incongruenze o errori.  
	Esempi di anomalie possono corrispondere a:
	\begin{itemize}
	\item un errore concettuale all'interno della documentazione di progetto;
	\item un errore ortografico;
	\item una violazione delle norme tipografiche riportate all'interno del documento \textit{Norme di Progetto v1.00};
	\item un'uscita dai range di accettazione descritti nella sezione Misure e metriche del presente
	documento;
	\item un'incongruenza nel prodotto software rispetto alle funzionalit� descritte all'interno del
	documento \textit{Analisi dei Requisiti v1.00};
	\item un'incongruenza del codice rispetto a quanto � stato progettato.
	\end{itemize}
	Per segnalare ogni anomalia va sollevata una issue tramite la piattaforma GitHub.
	\subsection{Procedure di controllo della qualit�}
		\subsubsection{Procedure di controllo della qualit� di processo}
		Il gruppo intende misurare con continuit� le caratteristiche di qualit� dei vari processi,
		al fine di poterli migliorare. Per mettere in atto ci� ci si basa su quanto descritto in seguito.
		\begin{itemize}
		\item Si adotta il modello CMM.
		\item Grazie all'uso del CMM � possibile valutare il livello di qualit� di un processo nello stato attuale.
		\item Per misurare la qualit� di un processo pu� essere utile verificare quella del suo
		prodotto: se essa � scarsa, ci� implica che probabilmente anche il processo dal quale deriva
		non � per nulla di qualit�.
		\item Per ottenere un miglioramento continuo si applica ai processi attivi il metodo di gestione PDCA.
		\end{itemize}
		\subsubsection{Procedure di controllo della qualit� di prodotto}
			\paragraph{Verifica}
			 Quando si effettuano delle verifiche si possono usare due tipi di analisi: analisi statica e analisi dinamica. L'analisi statica pu� essere applicata sia alla documentazione che al software, mentre l'analisi dinamica solo al software.
				\subparagraph{Analisi statica} 
				\begin{description}
					\item[Inspection] Questa tecnica di analisi presuppone l'esperienza da parte del verificatore nel individuare gli errori e le anomalie pi� frequenti. A tal scopo � stata stilata una lista di controllo nella quale sono elencate le sezioni critiche. Questo ci consente una verifica pi� rapida e meno risorse umane. Dopo aver terminato l'analisi, � necessario stilare un rapporto di verifica.
					\item[Walkthrough] Questa tecnica di analisi prevede una lettura critica del codice o del documento prodotto. Tale tecnica � molto dispendiosa in termini di risorse, poich� viene applicata all'intero documento, senza avere una precisa idea di quale sia il tipo di anomalia e di dove ricercarla;
				\end{description}
				\subparagraph{Analisi dinamica}
				\begin{description}
					\item [Test di unit�] Consiste nel verificare ogni singola unit� del prodotto software, ovvero sia la pi� piccola parte di software che conviene testare da sola, attraverso l'utilizzo di strumenti come logger, stub o driver. Data la sua natura, la dimensione dell'unit� da testare verr� definita al momento del test. Lo scopo dello unit testing � di verificare il corretto funzionamento di un'unit� per permettere una precoce individuazione dei bug. Uno unit testing accurato produce vari vantaggi, ad esempio:
					\begin{itemize}
						\item semplifica le modifiche;
						\item semplifica l'integrazione;
						\item supporta la documentazione.
					\end{itemize}
					\item[Test di integrazione] Consiste nella verifica dei componenti del sistema che sono formati dalla combinazione di pi� unit�. Ha lo scopo di evidenziare gli eventuali errori residui, non individuati durante la realizzazione dei singoli moduli.
					\item[Test di sistema] Consiste nell'eseguire nuovamente i test di unit� e integrazione per le componenti che hanno subito modifiche o per le nuove funzionalit�. Lo scopo � verificare di avere un prodotto di alta qualit� per ogni nuova funzionalit� o modifica importante.
					\item[Test di accettazione] � il collaudo del prodotto software che viene eseguito in presenza del proponente. Se tale collaudo viene superato positivamente si pu� procedere al rilascio ufficiale del prodotto sviluppato. 
				\end{description}
			\paragraph{Validazione}
			La validazione avviene nel momento in cui il prodotto ha superato i test di verifica ed � pronto al suo rilascio.
			Il prodotto dopo aver superato la validazione ci conferma che � conforme alle aspettative e soddisfa tutti i requisiti, di conseguenza � pronto per essere rilasciato.
\end{document}