\documentclass[../PianoDiQualifica.tex]{subfiles}

\begin{document}
\begin{appendices}

\section{Capability Maturity Model}
	Il CMM, acronimo di Capability Maturity Model, � un modello di sviluppo creato in seguito allo studio, finanziato dal Dipartimento della Difesa Statunitense, dei dati raccolti dalle organizzazioni che collaboravano con esso.
	Tale modello mira a migliorare i processi di sviluppo software esistenti. Il nome stesso del modello suggerisce i concetti su cui si basa:
	\begin{description}
		\item[capability:] � una caratteristica di ogni processo che indica quanto esso sia adeguato per gli scopi per cui � stato definito; tale caratteristica determina l'apporto in termini di efficacia ed efficienza finali raggiungibile attraverso un processo;
		\item[maturity:] � una caratteristica di un insieme di processi, attraverso la quale � possibile misurare quanto � governato il sistema dei processi di un'azienda;
		\item[model:] � la definizione di un insieme di requisiti, sempre pi� stringenti, che consentono di valutare il percorso di miglioramento dei processi di un'azienda. 
	\end{description}
	Il modello CMM fornisce:
	\begin{itemize}
		\item una base concettuale su cui appoggiarsi per valutare il livello dei processi;
		\item un insieme di best practice consolidate negli anni da esperti e utilizzatori;
		\item un linguaggio comune e una visione condivisa;
		\item un metodo per definire un miglioramento in ambito organizzativo.
	\end{itemize}
	\subsection{Struttura}
	Il modello CMMG comprende cinque aspetti:
	\begin{description}
		\item[Livelli di maturit�:] sono cinque livelli di maturit�, dove il pi� alto (il quinto) � uno stato teoricamente ideale in cui i processi vengono sistematicamente gestiti attraverso una combinazione di ottimizzazioni di processi e miglioramenti continui di processi;
		\item[Aree chiave di processo:] un'area chiave di processo identifica un gruppo di attivit� correlate che, quando vengono eseguite insieme, producono una serie di obiettivi considerati strategici;
		\item[Obiettivi:] gli obiettivi di un'area chiave di processo riassumono gli stati che devono esistere per quell'area per essere implementati in modo completo e duraturo. La quantit� di obiettivi che sono stati raggiunti � un indicatore della capability che l'organizzazione ha raggiunto in un certo livello di maturit�;
		\item[Caratteristiche comuni:] le caratteristiche comuni includono le pratiche che sviluppano e regolamentano un'area chiave di processo. Ci sono cinque tipi di caratteristiche comuni:
		\begin{itemize}
			\item l'impegno nell'esecuzione;
			\item l'abilit� nell'esecuzione;
			\item le attivit� eseguite;
			\item le misurazioni e le analisi;
			\item la verifica e l'implementazione.
		\end{itemize}
		\item[Le pratiche fondamentali:] le pratiche fondamentali descrivono gli elementi dell'infrastruttura e le pratiche che contribuiscono in modo particolare all'implementazione e alla regolamentazione dell'area.
	\end{description}
	\subsection{Livelli}
	
\end{appendices}
\end{document}