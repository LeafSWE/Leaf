\documentclass[../PianoDiQualifica.tex]{subfiles}

\begin{document}
\section{Introduzione}
	\subsection{Scopo del documento}
	Il documento ha l'obiettivo di definire un piano per conseguire la qualit� di prodotto e di processo. Questo documento dar� inoltre una visione di come il gruppo Leaf affronter� le varie fasi di verifica per poter conseguire il miglior risultato possibile in termini di qualit�.
	\subsection{Scopo del prodotto}
	Lo scopo del prodotto � implementare un metodo di navigazione indoor che sia funzionale alla tecnologia BLE.
	Il prodotto comprender� un prototipo software che permetta la navigazione all'interno di un'area predefinita, basandosi sui concetti di IPS e smart places.
	\subsection{Glossario}
	Per evitare ambiguit�e aiutare la comprensione del documento si � redatto un apposito glossario (\textit{Glossario v1.00}) che contiene la spiegazione degli acronimi e delle terminologie tecniche utilizzate. Per facilitare la lettura, i vocaboli in questione all'interno del presente documento sono marcati da una "G" maiuscola a pedice.
	\subsection{Riferimenti utili}
		\subsubsection{Riferimenti normativi}
		\begin{itemize}
			\item \textbf{Norme di Progetto:} \textit{Norme di Progetto v1.00};
			\item \textbf{Piano di Progetto:} \textit{Piano di Progetto v1.00};
			\item \textbf{Standard 9126:2001:} \url{https://en.wikipedia.org/wiki/ISO/IEC\_9126};
			\item \textbf{Standard 15504:2004:} \url{https://en.wikipedia.org/wiki/ISO/IEC\_15504};
			\item \textbf{Capability Maturity Model (CMM):} \\\url{https://en.wikipedia.org/wiki/Capability\_Maturity\_Model};
			\item \textbf{Plan-Do-Check-Act (PDCA):} \url{https://en.wikipedia.org/wiki/PDCA}.
		\end{itemize}
		\subsubsection{Riferimenti informativi}
		\begin{itemize}
			\item \textbf{Indice Gulpease:} \url{https://it.wikipedia.org/wiki/Indice\_Gulpease};
			\item \textbf{Materiale del corso di Ingegneria del Software:} \\\url{http://www.math.unipd.it/~tullio/IS-1/2015}.
		\end{itemize}
\end{document}