\documentclass[../PianoDiQualifica.tex]{subfiles}

\begin{document}
\begin{appendices}
\section{Resoconto delle attività di verifica - fase PA}
All'interno di questa fase\g, secondo quanto riportato nel documento \pianodiprogetto, sono previsti più momenti in cui viene attivato il processo\g\ di verifica. Si è cercato di riportare in questa sezione tutti i risultati che sono stati ottenuti durante questi momenti. Ove fosse necessario, si sono tratte anche delle conclusioni sui risultati ottenuti e su come essi possono essere migliorati.
	
	\subsection{Verifica sui processi}
		\subsubsection{processo di documentazione}
			\paragraph{Miglioramento costante}
			All'inizio della fase\g\ PA il processo\g\ di documentazione si posizionava al livello 3 della scala CMM\g.\\
			A causa del carico di lavoro richiesto dagli altri documenti, il gruppo non è riuscito a definire nuove norme e ricercare nuovi strumenti per permettere il passaggio al livello 4 della scala CMM\g.
			
			\paragraph{Rispetto della pianificazione}
			Per capire se le attività di un processo\g\ sono in ritardo rispetto a quanto pianificato all'interno del \pianodiprogetto\ viene utilizzata la seguente metrica: Schedule Variance.\\
			Si desidera che il ritardo accumulato sia minore del 5\% rispetto al totale pianificato. Sarebbe invece ottimale essere esattamente in linea con quanto prevede il \pianodiprogetto, o essere addirittura in anticipo.\\
			Di seguito sono riportati i valori ottenuti calcolando la Schedule Variance sui tempi di stesura di ogni documento nella fase\g\ PA:
			\begin{table}[H]
				\centering
				\begin{tabular}{l * {2}{c}}
					\toprule
					\textbf{Documento} & \textbf{Schedule Variance} & \textbf{Esito} \\
					\midrule
					\textit{Piano di progetto v3.00} & 0\% &  Ottimale \\
					\textit{Norme di progetto v3.00} & +33\% & Non accettabile \\
					\textit{Analisi dei requisiti v3.00} & +25\% & Non accettabile \\
					\textit{Piano di qualifica v3.00} & 0\% & Ottimale \\
					\textit{Glossario v3.00} & -6\% & Ottimale \\
					\textit{Specifica tecnica v1.00} & +11\% & Non accettabile \\
					\bottomrule
				\end{tabular}
				\caption{Esiti del calcolo della Schedule Variance sul processo di documentazione durante la fase PA}
				\label{tab:esiti_schedule_variance}
			\end{table}
			Come è possibile osservare dai dati della tabella, c'è stata una sottostima dei giorni necessari a completare i documenti \normediprogetto\ e \analisideirequisiti: ciò è dovuto principalmente al ritardo accumulato nella fase\g\ precedente, che ha ritardato la data di inizio delle attività di questa fase\g\ e di conseguenza influito negativamente sulla Schedule Variance.
			
			\paragraph{Rispetto del budget}
			Per capire se i costi di un processo\g\ rientrano nel budget previsto dal \pianodiprogetto\ viene utilizzata la seguente metrica: Budget Variance.\\
			L'obiettivo minimo è quello di avere dei costi che non superano il budget a disposizione per più del 10\%. Sarebbe invece ottimale che i costi fossero esattamente in linea con il preventivo o che addirittura si avesse speso meno.\\
			Di seguito sono riportati i valori ottenuti calcolando la Budget Variance sui tempi di stesura di ogni documento nella fase\g\ PA:
			\begin{table}[H]
				\centering
				\begin{tabular}{l * {2}{c}}
					\toprule
					\textbf{Documento} & \textbf{Budget Variance} & \textbf{Esito} \\
					\midrule
					\textit{Piano di progetto v3.00} & -10\% &  Ottimale \\
					\textit{Norme di progetto v3.00} & -6\% & Ottimale \\
					\textit{Analisi dei requisiti v3.00} & -14\% & Ottimale \\
					\textit{Piano di qualifica v3.00} & -22\% & Ottimale \\
					\textit{Glossario v3.00} & 0\% & Ottimale \\
					\textit{Specifica tecnica v1.00} & +16\% & Non Accettabile \\
					Totale processo\g\ di documentazione & -5\% & Ottimale \\
					\bottomrule
				\end{tabular}
				\caption{Esiti del calcolo della Budget Variance sul processo di documentazione durante la fase PA}
				\label{tab:esiti_budget_variance}
			\end{table}
			 Si può notare dalla tabella che il documento \specificatecnica\ ha richiesto più ore da \progettista\ di quante ne fossero state preventivate, mentre per i restanti documenti c'è stata una sovrastima di ore per i rispettivi ruoli.
						
		\subsubsection{processo di verifica}
			\paragraph{Miglioramento costante}
			Il processo\g\ di verifica inizia ad essere ben documentato e ben gestito, inoltre ogni membro del gruppo ha ormai chiare in mente le procedura di verifica e conosce a fondo le \normediprogetto: tutto ciò ha permesso di raggiungere il terzo livello della scala CMM\g\ (Defined).
			
			\paragraph{Rispetto della pianificazione}
			Per capire se le attività di un processo\g\ sono in ritardo rispetto a quanto pianificato all'interno del \pianodiprogetto\ viene utilizzata la seguente metrica: Schedule Variance.\\
			Si desidera che il ritardo accumulato sia minore del 5\% rispetto al totale pianificato. Sarebbe invece ottimale essere esattamente in linea con quanto prevede il \pianodiprogetto, o essere addirittura in anticipo.\\
			Di seguito sono riportati i valori ottenuti calcolando la Schedule Variance sui tempi di verifica nella fase\g\ PA:
			\begin{table}[H]
				\centering
				\begin{tabular}{l * {2}{c}}
					\toprule
					\textbf{Processo} & \textbf{Schedule Variance} & \textbf{Esito} \\
					\midrule
					processo\g\ di verifica & -42\% &  Ottimale \\
					\bottomrule
				\end{tabular}
				\caption{Esiti del calcolo della Schedule Variance sul processo di verifica durante la fase PA}
				\label{tab:esiti_schedule_variance}
			\end{table}
			
			\paragraph{Rispetto del budget}
			Per il processo\g\ di verifica è stato investito un minor numero di risorse rispetto a quanto preventivato, di conseguenza il valore della Budget Variance risulta \textbf{ottimale}.\\
			Di seguito sono riportati i valori ottenuti:
			\begin{table}[H]
				\centering
				\begin{tabular}{l * {2}{c}}
					\toprule
					\textbf{Processo} & \textbf{Budget Variance} & \textbf{Esito} \\
					\midrule
					processo\g\ di verifica & -22\% &  Ottimale \\
					\bottomrule
				\end{tabular}
				\caption{Esiti del calcolo della Budget Variance sul processo di verifica durante la fase PA}
				\label{tab:esiti_budget_variance}
			\end{table}
			
			
	\subsection{Verifica sui prodotti}
	In questa sezione verranno riportati i dati emessi dalle procedure di controllo della qualità di prodotto\g.
		\subsubsection{Documenti}
		In questa sezione vengono riportati gli esiti delle attività di verifica svolte sui documenti.\\
		Tali esiti sono strettamente correlati agli obiettivi di qualità dei documenti enunciati alla sezione \ref{ObiettiviDiQualità} del presente documento.
			
			\paragraph{Leggibilità e comprensibilità}
			Per cercare di capire quanto i documenti siano effettivamente leggibili e comprensibili da persone dotate di una licenza superiore viene utilizzato l’indice Gulpease\g.\\
			Si desidera che i documenti posseggano costantemente un indice maggiore a 40 (soglia di accettabilità). Si dovrebbe tuttavia cercare di raggiungere un valore più alto, considerato ottimale, ovvero 60.\\
			Il documento \textit{Glossario v3.00} ha dato esito \textbf{ottimale}, mentre tutti gli altri documenti prodotti hanno dato esito \textbf{accettabile}.
			
			\paragraph{Correttezza ortografica}
			Per capire quanto i documenti siano effettivamente corretti a livello ortografico viene utilizzata la seguente metrica: percentuale di errori ortografici rinvenuti e non corretti.\\
			Si desidera che tutti gli errori ortografici che sono stati trovati siano corretti. In questo caso, dunque, l'obiettivo minimo coincide con l'obiettivo ottimale.\\
			Di seguito sono riportati gli errori ortografici trovati tramite verifica automatica dei documenti durante la fase\g\ PA.
			\begin{table}[H]
				\centering
				\begin{tabular}{l * {2}{c}}
					\midrule
					Errori ortografici & 7 \\
					\midrule
				\end{tabular}
				\caption{Errori ortografici trovati tramite verifica automatica dei documenti durante la fase PA}
				\label{tab:errori_automatica}
			\end{table}
			Tutti gli errori ortografici rinvenuti sono stati corretti, quindi è stato raggiunto l'obiettivo \textbf{ottimale}.
			
			\paragraph{Correttezza concettuale}
			Per capire quanto i documenti siano effettivamente corretti a livello concettuale viene utilizzata la seguente metrica: percentuale di errori concettuali rinvenuti e non corretti.\\
			Si desidera che al massimo il 5\% degli errori concettuali rinvenuti non siano corretti. L'obiettivo ottimale sarebbe quello di correggere tutti gli errori trovati. \\
			Di seguito sono riportati gli errori concettuali trovati dei documenti durante la fase\g\ PA.
			\begin{table}[H]
				\centering
				\begin{tabular}{l * {2}{c}}
					\midrule
					Errori concettuali & 9 \\
					\midrule
				\end{tabular}
				\caption{Errori concettuali trovati tramite verifica manuale dei documenti durante la fase PA}
				\label{tab:errori_concettuali}
			\end{table}
			Tutti gli errori concettuali rinvenuti sono stati corretti, quindi è stato raggiunto l'obiettivo \textbf{ottimale}.
			
\end{appendices}
\end{document}