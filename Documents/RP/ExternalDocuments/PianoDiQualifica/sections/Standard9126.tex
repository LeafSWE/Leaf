\documentclass[../PianoDiQualifica.tex]{subfiles}

\begin{document}
\begin{appendices}

\section{Standard ISO/IEC 9126}
	Lo Standard ISO/IEC 9126 si suddivide in quattro parti:
		\begin{enumerate}
			\item modello della qualità del software\g\ (9126-1);
			\item metriche per la qualità esterna (9126-2);
			\item metriche per la qualità interna (9126-3);
			\item metriche per la qualità in uso (9126-4).
		\end{enumerate}
	Lo standard tratta la qualità del software\g\ da tre punti di vista:
	\begin{description}
		\item[qualità interna:] è la qualità del prodotto\g\ software\g\ vista dall'interno e fa quindi riferimento alle caratteristiche implementative del software\g\ quali l'architettura e il codice che ne deriva.
		\item[qualità esterna:] è la qualità del prodotto\g\ software\g\ vista dall'esterno nel momento in cui esso viene eseguito e testato in un ambiente di prova.
		\item[qualità in uso:] è la qualità del prodotto\g\ software\g\ dal punto di vista dell'utilizzatore che ne fa uso all'interno di uno specifico sistema e contesto.
	\end{description}
	
	\subsection{Modello della qualità del software} 
	Nella prima parte vengono descritti i modelli per la qualità esterna, interna ed in uso.
		\subsubsection{Modello della qualità esterna ed interna}
		Il modello di qualità esterna ed interna stabilito nella prima parte dello standard è classificato da sei caratteristiche generali:
		\begin{description}
			\item[funzionalità:] la funzionalità rappresenta la capacità del software\g\ di fornire le funzioni necessarie per operare in determinate condizioni, cioè in un determinato contesto.
			\item[affidabilità:] l'affidabilità è la capacità del prodotto\g\ software\g\ di mantenere un certo livello di prestazioni quando viene usato in condizioni specifiche e per un intervallo di tempo fissato.
			\item[usabilità:] l'usabilità rappresenta la capacità di un prodotto\g\ software\g\ di essere comprensibile. Un software\g\ è considerato usabile in proporzione alla facilità con cui gli utenti operano per sfruttare a pieno le funzionalità che il software realizza.
			\item[efficienza:] l'efficienza rappresenta la capacità di un prodotto\g\ di realizzare le funzioni richieste nel minor tempo possibile ed utilizzando nel miglior modo le risorse necessarie.
			\item[manutenibilità:] la manutenibilità rappresenta la capacità di un prodotto\g\ software\g\ di essere modificato (a costi accessibili ed in tempi rapidi). Le modifiche possono includere correzioni o adattamenti del software\g\ a cambiamenti negli ambienti, nei requisiti e nelle specifiche funzionali.
			\item[portabilità:] la portabilità rappresenta la capacità di un prodotto\g\ software\g\ di poter essere trasportato da un ambiente all'altro (in modo sufficientemente veloce). L'ambiente include aspetti hardware e software\g.
		\end{description}
		Tali caratteristiche sono misurabili attraverso delle metriche.
		
		\subsubsection{Modello della qualità in uso}
		Gli attributi presenti nel modello relativo alla qualità del software\g\ in uso sono rappresentati da quattro grandi categorie:
		\begin{description}
			\item[efficacia:] l'efficacia di un prodotto\g\ software\g\ rappresenta la capacità di permettere all'utente di raggiungere obiettivi specifici con accuratezza e completezza in uno specifico contesto di utilizzo.
			\item[produttività:] la produttività di un prodotto\g\ software\g\ rappresenta la capacità di permettere all'utente di impegnare un numero definito di risorse, in relazione all’efficienza raggiunta in uno specifico contesto di utilizzo.
			\item[sicurezza fisica:] la sicurezza fisica di un prodotto\g\ software\g\ rappresenta la capacità di raggiungere un livello accettabile di rischio per i dati, le persone, il business, la proprietà o gli ambienti in uno specifico contesto di utilizzo.
			\item[soddisfazione:] la soddisfazione di un prodotto\g\ software\g\ rappresenta la capacità di soddisfare gli utenti in uno specifico contesto di utilizzo.
		\end{description}
	
	\subsection{Metriche per la qualità del software}
	Nelle restanti tre parti vengono trattate le metriche per la qualità esterna, interna e in uso.
	
	\subsubsection{Metriche per la qualità esterna}
	Le metriche esterne misurano i comportamenti del prodotto\g\ software\g\ rilevabili dai test, dall'operatività e dall'osservazione durante la sua esecuzione. L'esecuzione del prodotto\g\ software\g\ è fatta in un contesto tecnico rilevante.
	Le metriche esterne sono scelte in base alle caratteristiche che il prodotto\g\ finale dovrà dimostrare durante la sua esecuzione in esercizio.
	
	\subsubsection{Metriche per la qualità interna}
	Le metriche interne si applicano al software\g\ non eseguibile (come, ad esempio, il codice sorgente) e alla documentazione. Le misure effettuate permettono di prevedere il livello di qualità esterna ed in uso del prodotto\g\ finale poiché gli attributi interni influenzano le caratteristiche esterne e quelle in uso.
	
	\subsubsection{Metriche per la qualità in uso}
	Le metriche della qualità in uso misurano il grado con cui il prodotto\g\ software\g\ permette agli utenti di svolgere le proprie attività con efficacia, produttività, sicurezza e soddisfazione nel contesto operativo previsto.
	
\end{appendices}
\end{document}