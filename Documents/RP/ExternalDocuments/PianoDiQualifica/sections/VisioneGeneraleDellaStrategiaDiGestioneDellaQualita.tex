\documentclass[../PianoDiQualifica.tex]{subfiles}

\begin{document}
\section{Visione generale della strategia di gestione della qualità}
	\subsection{Obiettivi di qualità}
	In questa sezione vengono riportati gli obiettivi di qualità che il gruppo \leaf\ si impegna a perseguire durante lo svolgimento dell'intero progetto. Per capire se un certo obiettivo è stato raggiunto oppure no, il gruppo fa uso di standard, modelli e metriche. Ognuno di questi fa uso di scale differenti e fissate a priori: qualunque sia il criterio utilizzato per misurare e dunque quantificare la vicinanza a un certo obiettivo abbiamo fissato dei valori minimi che intendiamo raggiungere nell'arco dell'intero progetto. Oltre a ciò abbiamo anche fissato dei valori che riteniamo ottimali e che devono essere sperabilmente (ma non obbligatoriamente) raggiunti.
		\subsubsection{Qualità di processo}
		Assicurare la qualità dei processi è indispensabile durante lo svolgimento del progetto per le seguenti ragioni:
		\begin{itemize}
		\item aiuta ad ottimizzare l'uso delle risorse;
		\item fa in modo che i costi siano maggiormente contenuti;
		\item migliora la stima dei rischi e degli impegni.
		\end{itemize}
		Un altro fattore da tenere sempre in considerazione risiede nel fatto che molto spesso prodotti scadenti derivano da pessimi processi.\\
		Le caratteristiche ottimali che desideriamo che i processi posseggano vengono riportate in seguito:
		\begin{itemize}
			\item un processo dovrebbe essere in grado di migliorare continuamente le proprie performance
			\begin{itemize}
				\item le performance di un processo dovrebbero essere costantemente misurabili;
				\item un processo, durante il proprio svolgimento, dovrebbe cercare di perseguire sempre degli obiettivi quantitativi di miglioramento fissati a priori.
			\end{itemize}
			\item le attività di un processo dovrebbero proseguire nei tempi indicati nel documento \pianodiprogettov;
			\item i costi effettivi di ogni processo dovrebbero essere in linea con quanto dichiarato nel documento \pianodiprogettov.
		\end{itemize}
		Nelle prossime sezioni si enunciano gli obiettivi che si intendono raggiungere. Per ognuno di essi vengono specificati i criteri con i quali si effettuano le misurazioni sulla qualità (per capire quanto si è vicini all’obiettivo).\\
		Infine, per ogni criterio scelto vengono dichiarati i valori minimi che si intendono raggiungere, oltre a quelli ottimali.
			
		\paragraph{Miglioramento costante - OQPC1}
		Per misurare quanto si è vicini all’obiettivo di avere processi in grado di misurare le proprie performance e che sono quindi in grado di porsi obiettivi quantitativi di miglioramento si è deciso di adottare il modello CMM.\\
		In particolare si vuole raggiungere almeno il livello 2 previsto da tale scala. Il livello ottimale che sperabilmente dovremmo raggiungere è il 4.\\
		Riassumendo:
		\begin{description}
			\item[Modello utilizzato per quantificare gli obiettivi:] CMM;
			\item[Soglia di accettabilità:] livello 2 previsto da CMM;
			\item[Soglia di ottimalità:] livello 4 previsto da CMM;
		\end{description}
		Per una migliore e più dettagliata descrizione del modello CMM qui adottato si faccia riferimento all'appendice A "Capability Maturity Model".\\
		Per approfondire la scelta dei range di accettazione e ottimalità si consulti invece la metrica MPC1 "Capability Maturity Model".
		
		\paragraph{Rispetto della pianificazione - OQPC2}
		
		\paragraph{Rispetto del budget - OQPC3}
		
		
		\subsubsection{Qualità di prodotto}
		Il gruppo si prefigge di mantenere la stessa qualità sia nei processi che nei prodotti: per garantire la migliore qualità del prodotto\g\ anche il processo\g\ da cui proviene deve avere una buona qualità. Il gruppo per mantenere la qualità del prodotto\g\ 
cercherà di seguire al meglio lo standard di qualità [ISO/IEC 9126:2001].\\
		I prodotti che vengono realizzati durante l'intero progetto sono sostanzialmente di due tipi: documenti e software\g. Nelle prossime sezioni, si enunciano gli obiettivi che si intendono raggiungere suddivisi per tipologia di prodotto\g.
			\paragraph{Qualità dei documenti}
			Gli obiettivi di qualità riguardanti i documenti ai quali il gruppo \leaf\ desidera arrivare nell'arco dell'intero progetto sono i seguenti:
			\begin{itemize}
				\item i documenti devono essere comprensibili da individui dotati di un'istruzione media, caratteristica verificata attraverso l'utilizzo dell'indice Gulpease\g;
				\item i documenti devono essere corretti a livello ortografico;
				\item i documenti non devono contenere concetti errati.
			\end{itemize}
			Abbiamo individuato le metriche e i criteri che si intendono utilizzare per quantificare la vicinanza a ognuno degli obiettivi sopra descritti e le soglie di accettabilità e ottimalità per questi obiettivi, per fissare quantitativamente i punti ai quali desideriamo arrivare. Queste metriche e soglie sono definite nel presente documento all'interno della sezione 3.4.2 "Metriche per i documenti".
				
				\subparagraph{Leggibilità e comprensibilità - OQPRD1}
				\subparagraph{Correttezza ortografica - OQPRD2}
				\subparagraph{Correttezza concettuale - OQPRD3}
			
			\paragraph{Qualità del software}
			Gli obiettivi di qualità del software\g\ ai quali il gruppo \leaf\ desidera arrivare nell'arco del progetto sono alcuni di quelli che sono enunciati all'interno dello standard [ISO/IEC 9126:2001]. Vengono riassunti in seguito:
			\begin{itemize}
				\item il prodotto\g\ possiede le funzionalità descritte all'interno dei requisiti obbligatori e desiderabili;
				\item il prodotto\g\ permette agli utenti di utilizzare le funzionalità in maniera semplice ed efficace;
				\item il codice risulta manutenibile e facilmente comprensibile;
				\item il prodotto\g\ è robusto e non interrompe l'esecuzione in seguito a situazioni anomale;
				\item il prodotto\g\ è testato in ogni sua parte e in ogni situazione nella quale si può trovare;
				\item il prodotto\g\ garantisce un funzionamento senza interruzioni;
				\item il prodotto\g\ è facilmente installabile.
			\end{itemize}
	\subsection{Scadenze temporali}
	Le scadenze che il gruppo \leaf\ ha deciso di rispettare sono riportate nel \pianodiprogettov
\end{document}