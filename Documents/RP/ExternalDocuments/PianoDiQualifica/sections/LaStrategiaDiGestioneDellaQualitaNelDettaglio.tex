\documentclass[../PianoDiQualifica.tex]{subfiles}

\begin{document}
\section{La strategia di gestione della qualità nel dettaglio}
		\subsection{Risorse}
		Per garantire un buon funzionamento del processo\g\ di verifica verranno impiegati i seguenti tipi di risorse:
		\begin{itemize}
			\item risorse umane;
			\item risorse hardware;
			\item risorse software\g.
		\end{itemize}
			\subsubsection{Risorse necessarie}
				\paragraph{Risorse umane}
				Le risorse umane di cui il processo\g\ di verifica avrà bisogno sono il \responsabilediprogetto\ e i \verificatori.
				Informazioni più dettagliate sui ruoli sono riportate nelle \normediprogettov.
				\paragraph{Risorse hardware}
				Per eseguire la verifica, il gruppo dovrà avere a disposizione dei computer con un'adeguata potenza di calcolo in grado di sopportare il carico di lavoro.		    
				\paragraph{Risorse software}
			    Le risorse software\g\ necessarie alla verifica sono gli strumenti software\g\ 
che permettono di eseguire controlli sui documenti e verificare che essi aderiscano alle \normediprogettov.\\
			    Gli strumenti software\g\ dovranno avere le seguenti caratteristiche:
			    \begin{itemize}
			    	\item rilevare (durante la scrittura) eventuali errori ortografici;
			    	\item costruire e visualizzare in tempo reale il documento scritto in \LaTeX\g\ (in modo che sia facile accorgersi di errori nell'utilizzo dei comandi).
			    \end{itemize}
			    Inoltre è necessario disporre di una piattaforma che raccolga i vari errori incontrati e li segnali ai componenti del gruppo che dovranno occuparsene. 
		    \subsubsection{Risorse disponibili}
			    \paragraph{Risorse umane}
			    Tutti i membri del gruppo sono a disposizione per eseguire operazioni di verifica. Ognuno dei componenti, a turno, ricoprirà il ruolo di \responsabilediprogetto\ o di \verificatore\ come definito nel \pianodiprogettov.
			    \paragraph{Risorse hardware}
			    Le risorse hardware disponibili sono i vari computer dei componenti del gruppo incaricati di svolgere il ruolo di \responsabilediprogetto\ o \verificatore. Eventualmente sono disponibili anche i computer del Servizio Calcolo dell'Università di Padova.
				\paragraph{Risorse software}
				Le risorse software\g\ disponibili comprendono editor \LaTeX\g\ con controlli integrati e script per controllare la leggibilità e la complessità dei documenti in riferimento all'indice Gulpease\g.\\
				Sarà disponibile anche il sistema di sollevamento delle issue\g\ offerto dalla piattaforma GitHub\g. Per maggiori informazioni sulla procedura di sollevamento e gestione delle issue\g\ 
si veda il documento \normediprogettov.
		\subsection{Misure e metriche}\label{MisureMetriche}
			\subsubsection{Misure}
			Ogni volta che viene effettuata una misura sui processi o sui prodotti essa va rapportata in una scala. Di seguito vengono riportati i valori della scala:
			\begin{description}
			\item[Negativo] Valore non accettabile, bisogna effettuare ulteriori verifiche e correggere gli errori presenti.
			\item[Accettabile] Valore accettabile, l'oggetto sottoposto a verifica ha raggiunto una soglia minima.
			\item[Ottimale] Valore accettabile, l'oggetto sottoposto a verifica ha raggiunto le massime aspettative del team\g. L'obiettivo dovrebbe essere quello di avere tutti i valori all'interno di tale range. 
			\end{description}
			
			\subsubsection{Metriche per i processi}
			Le seguenti metriche sono identificate da un codice identificativo, al fine di rendere più semplice il tracciamento tra l'obiettivo di qualità da soddisfare e la metrica che permettere di verificarne il soddisfacimento.\\
			La procedura di denominazione delle metriche è spiegata in dettaglio nel documento \normediprogettov.
			
			\paragraph{Capability Maturity Model - MPC1}\label{MPC1}
			Per controllare e verificare la qualità dei processi, il gruppo adotterà le metriche fornite dal modello CMM\g\ dove per ogni fase\g\ di lavoro si andrà a fornire un indice che descriverà la qualità della fase\g\ presa in esame. L'indice sarà relativo ad una scala già definita dal CMM\g.\\
			Effettuando questo tipo di verifiche il team\g\ avrà subito un riscontro della qualità del processo\g. CMM\g\ ci consente di individuare la maturità di un processo\g: essa può assumere un valore da 1 (il
			peggiore) a 5 (il migliore).\\
			Mettendo ora in relazione i risultati di tale modello con i range da noi stabiliti otteniamo quanto segue:
			\begin{itemize}
				\item il valore 1 è considerato negativo;
				\item i valori 2 e 3 sono considerati accettabili;
				\item i valori 4 e 5 sono considerati ottimali.
			\end{itemize}
			
			\paragraph{Schedule Variance - MPC2}\label{MPC2}
			La presente metrica indica se le attività di progetto sono in anticipo o in ritardo rispetto a quelle pianificate nel \pianodiprogetto.\\
			Costituisce un indicatore di efficacia dei processi e viene calcolata come la differenza fra la data pianificata di fine di un'attività e la data di fine reale dell'attività stessa.\\
			Se la schedule variance è minore di 0 significa che il progetto sta producendo con minor velocità rispetto a quanto pianificato, viceversa se positivo. Se è pari a 0 significa che si è perfettamente in linea con la pianificazione.\\
			I range di accettazione per questa metrica sono:
			\begin{itemize}
				\item un deficit maggiore del 5\% del tempo pianificato per il processo è considerato negativo;
				\item un deficit minore del 5\% del tempo pianificato per il processo è considerato accettabile;
				\item valori maggiori o uguali a 0 sono considerati ottimali.
			\end{itemize}
					
			\paragraph{Budget Variance - MPC3}\label{MPC3}
			La presente metrica indica se alla data corrente i costi sono maggiori o minori rispetto a quanto previsto.\\
			Costituisce un indice di efficienza e si calcola confrontando il preventivo con il consuntivo.\\
			I range di accettazione per questa metrica sono:
			\begin{itemize}
				\item un deficit maggiore del 10\% delle risorse preventivate per il processo è considerato negativo;
				\item un deficit minore del 10\% delle risorse preventivate per il processo è accettabile;
				\item un valore maggiore o uguale a 0 è considerato ottimale.
			\end{itemize}
			
			\subsubsection{Metriche per i prodotti}
			Le seguenti metriche sono identificate da un codice identificativo, al fine di rendere più semplice il tracciamento tra l'obiettivo di qualità da soddisfare e la metrica che permettere di verificarne il soddisfacimento.\\
			La procedura di denominazione delle metriche è spiegata in dettaglio nel documento \normediprogettov.
			
			\paragraph{Metriche per i documenti}
			La qualità di un documento dipende prima di tutto dai suoi contenuti. La loro qualità, tuttavia, è difficilmente quantificabile allo stato attuale del progetto a causa dell'esperienza pressoché nulla del team\g\ in quest'ambito. Si è deciso dunque di limitarsi a valutare parametri maggiormente oggettivi e soprattutto misurabili automaticamente attraverso strumenti software\g.
				\subparagraph{Indice di leggibilità - MPRD1}\label{MPRD1}
				Una metrica che si è deciso di utilizzare per poter stimare la qualità di un documento è l'indice di leggibilità. In particolare, è stato considerato l'indice Gulpease\g, studiato appositamente per la lingua italiana.				\\Questo particolare indice si basa sulla lunghezza della parola e sulla lunghezza della frase rispetto al numero di lettere. La formula per il suo calcolo è la seguente:
				\begin{equation}
					Indice \  Gulpease = 89 + \frac{300*numeroFrasi-10*numeroLettere}{numeroParole}
				\end{equation}
				Il risultato di tale equazione tipicamente è compreso tra 0 e 100, dove valori alti indicano leggibilità elevata e viceversa.\\
				In generale, risulta che testi con un indice:
				\begin{itemize}
					\item inferiore a 80 risultano difficili da leggere per chi ha la licenza elementare;
					\item inferiore a 60 risultano difficili da leggere per chi ha la licenza media;
					\item inferiore a 40 risultano difficili da leggere per chi ha la licenza superiore.
				\end{itemize}
				Vengono di seguito riportati i range stabiliti per la metrica appena introdotta. Si noti che viene tenuto in considerazione il fatto che la documentazione è destinata a persone sufficientemente preparate, competenti ed istruite.
				\begin{itemize}
					\item Valori minori di 40 sono considerati negativi.
					\item Valori compresi tra 40 e 60 sono considerati accettabili.
					\item Valori maggiori di 60 sono considerati ottimali. 
				\end{itemize}
				\subparagraph{Errori ortografici rinvenuti e non corretti - MPRD2}\label{MPRD2}
				Tale metrica è necessaria per capire quanto un documento sia corretto dal punto di vista ortografico. Infatti, supponendo che gli strumenti automatici siano in grado di trovare tutti (o perlomeno la maggior parte) degli errori ortografici all'interno di un testo, la correttezza ortografica non può che basarsi sul numero di errori rinvenuti ma non successivamente corretti. Notare che per errori corretti si intende un errore revisionato manualmente da parte di un \verificatore. Le correzioni automatiche, infatti, non sono molto attendibili. \\
				Vengono di seguito riportati i range stabiliti per la metrica appena introdotta:
				\begin{itemize}
					\item una percentuale di errori non corretti maggiore allo 0\% è ritenuta negativa;
					\item una percentuale di errori non corretti pari allo 0\% è ritenuta accettabile;
					\item una percentuale di errori non corretti pari allo 0\% è ritenuta ottimale.
				\end{itemize}
				Notare che non è accettabile che vi siano errori rinvenuti e non corretti da qualche membro del gruppo.
				\subparagraph{Errori concettuali rinvenuti e non corretti - MPRD3}\label{MPRD3}
				Tale metrica è necessaria per capire quanto un documento sia corretto dal punto di vista concettuale. Infatti, supponendo che in seguito alle revisioni siano stati trovati tutti (o perlomeno la maggior parte) i maggiori errori di questo tipo, la correttezza concettuale non può che basarsi sul numero di errori rinvenuti e fatti notare ma non successivamente corretti. Notare che per errori corretti si intende un errore fatto notare dal committente o da qualche \verificatore\ (con relativa approvazione del \responsabilediprogetto) e successivamente corretto (sulla base di discussioni interne o con il committente).\\
				Vengono di seguito riportati i range stabiliti per la metrica appena introdotta:
				\begin{itemize}
					\item una percentuale di errori non corretti maggiore al 5\% è ritenuta negativa;
					\item una percentuale di errori non corretti minore del 5\% è ritenuta accettabile;
					\item una percentuale di errori non corretti pari allo 0\% è ritenuta ottimale;
				\end{itemize}	

			\paragraph{Metriche per il software}
			Il gruppo \leaf\ per garantire la qualità del software ha deciso di adottare delle metriche. Esse hanno il compito di monitorare la qualità interna, qualità esterna e la qualità in uso. In base alle risorse a disposizione e agli obiettivi di qualità del software preposti il gruppo ha deciso di adottare alcune metriche presenti all'interno dello standard [ISO/IEC 9126:2001].\\
			Ogni metrica scelta viene associata ad una caratteristica di qualità presente all'interno dello standard:
			\begin{table}[H]
				\centering
				\begin{tabular}{l * {2}{c}}
					\toprule
					\textbf{Metriche scelte} & \textbf{Caratteristiche di Qualità} \\
					\midrule
					MPRS1 - Copertura Requisiti Obbligatori & Funzionalità \\
					MPRS2 - Copertura Requisiti Desiderabili &Funzionalità\\
					MPRS3 - Numero di statement per metodo & Manutenibilità\\
					MPRS4 - Numero di parametri per metodo & Manutenibilità\\
					MPRS5 - Numero di campi dati per classe & Manutenibilità\\
					MPRS6 - Grado di accoppiamento & Manutenibilità\\
					MPRS7 - Cyclomatic Number & Manutenibilità\\
					MPRS8 - Adequacy of variable names & Manutenibilità\\
					MPRS9 - Average Module Size & Manutenibilità\\
					MPRS10 - Test Passati Richiesti & Affidabilità\\
					MPRS11 - Failure Avoidance & Affidabilità\\
					MPRS12 - Breakdown Avoidance & Affidabilità\\
					\midrule
				\end{tabular}
				\caption{Mappa Metriche-Caratteristiche}
				\label{tab:mappa_metriche_caratteristice}
			\end{table}
			\subparagraph{Copertura Requisiti Obbligatori - MPRS1}\label{MPRS1}
			Questa metrica ci permette di verificare in ogni momento lo stato dei requisiti obbligatori coperti. Essa controlla il rapporto tra i requisiti obbligatori soddisfatti e il numero totale di requisiti obbligatori ricavati. Per una maggiore comprensione il risultato verrà riportato in percentuale.
			\begin{equation}
				Copertura \ Requisiti \ Obbligatori = \frac{\#\ requisiti\ obbligatori\ soddisfatti }{\#\ requisiti\ obbligatori\ totali}
			\end{equation}
			Vengono di seguito riportati i range stabiliti per la metrica appena introdotta:
			\begin{itemize}
				\item una percentuale minore del 100\% è ritenuta negativa;
				\item una percentuale uguale al 100\% è ritenuta accettabile;
				\item una percentuale uguale al 100\% ottimale.
			\end{itemize}
			\subparagraph{Copertura Requisiti Desiderabili - MPRS2}\label{MPRS2}
			Questa metrica ci permette di verificare in ogni momento lo stato dei requisiti desiderabili coperti. Essa controlla il rapporto tra i requisiti desiderabili soddisfatti e il numero totale di requisiti desiderabili ricavati. Per una maggiore comprensione il risultato verrà riportato in percentuale.
			\begin{equation}
			Copertura \ Requisiti \ Desiderabili = \frac{\#\ requisiti\ desiderabili\ soddisfatti }{\#\ requisiti\ desiderabili\ totali}
			\end{equation}
			Vengono di seguito riportati i range stabiliti per la metrica appena introdotta:
			\begin{itemize}
				\item una percentuale minore del 100\% è ritenuta negativa;
				\item una percentuale uguale al 100\% è ritenuta accettabile;
				\item una percentuale uguale al 100\% ottimale.
			\end{itemize}
			\subparagraph{Numero di statement per metodo - MPRS3}\label{MPRS3}
			La metrica è in grado di determinare se il numero di statement per metodo implementati rientri tra i valori definiti. Questo ci permette di tenere un livello di manutenibilità del codice accettabile.
			Vengono di seguito riportati i range stabiliti per la metrica appena introdotta:
			\begin{itemize}
				\item un valore maggiore di 40 è ritenuto negativo;
				\item un valore compreso tra 30 e 40 è ritenuto accettabile;
				\item un valore minore di 30 è ritenuto ottimale.
			\end{itemize}
			\subparagraph{Numero di parametri per metodo - MPRS4}\label{MPRS4}
			La metrica è in grado di determinare se il numero di parametri di un metodo rientri tra i valori definiti. Questo ci permette di tenere un grado di manutenibilità del codice accettabile.
			Vengono di seguito riportati i range stabiliti per la metrica appena introdotta:
			\begin{itemize}
				\item un valore maggiore di 10 è ritenuto negativo;
				\item un valore compreso tra 4 e 10 è ritenuto accettabile;
				\item un valore minore di 4 è ritenuto ottimale.
			\end{itemize}
			\subparagraph{Numero di campi dati per classe - MPRS5}\label{MPRS5}
			La metrica permette di verificare che il numero di campi dati per classe rientri tra i valori definiti. Questo ci permette di tenere un grado di manutenibilità e comprensibilità del codice accettabile.
			Vengono di seguito riportati i range stabiliti per la metrica appena introdotta:
			\begin{itemize}
			\item un valore maggiore di 15 è ritenuto negativo;
			\item un valore tra 10 e 15 è ritenuto accettabile;
			\item un valore minore di 10 è ritenuto ottimale.
			\end{itemize}
			\subparagraph{Grado di accoppiamento - MPRS6}\label{MPRS6}
			La metrica determina se il numero di dipendenze tra classi in un package\g rientri tra i valori definiti. Avere poche dipendenze tra classi implica che ci sia un maggiore grado di disaccoppiamento. Questo aumenta molto la manutenibilità e la comprensibilità del codice.\\
			Vengono di seguito riportati i range stabiliti per la metrica appena introdotta:
			\begin{itemize}
				\item un valore maggiore di 7 è ritenuto negativo;
				\item un valore compreso tra 3 e 7 è ritenuto accettabile;
				\item un valore minore di 3 è ritenuto ottimale.
			\end{itemize}
			\subparagraph{Cyclomatic Number - MPRS7}\label{MPRS7}
			La metrica controlla che la complessità ciclomatica del codice rientri tra i valori definiti. Essa ci permette di vedere il numero di cammini linearmente indipendenti presenti all'interno del codice. Il numero di cammini è direttamente proporzionale alla complessità ciclomatica del codice.
			\begin{equation}
			Cyclomatic \ Number = e - n + 2p 
			\end{equation}
			\begin{itemize}
				\item e = numero di archi;
				\item n = numero di nodi;
				\item p = numero di componenti connesse.
		    \end{itemize}
			Vengono di seguito riportati i range stabiliti per la metrica appena introdotta:
			\begin{itemize}
				\item un valore maggiore di 10 è ritenuto negativo;
				\item un valore compreso tra 4 e 10 è ritenuto accettabile;
				\item un valore minore di 4 è ritenuto ottimale.
			\end{itemize}
			%\subparagraph{Conditional Statement - MPRS8}\label{MPRS8}
			%La metrica misura il numero di statement condizionali presenti per metodo. Questo ci permette di valutare la complessità del codice di un metodo.  
			%Vengono di seguito riportati i range stabiliti per la metrica appena introdotta:
			%\begin{itemize}
				%\item un valore maggiore di 50 è ritenuto negativo;
				%\item un valore compreso tra 20 e 50 è ritenuto accettabile;
				%\item un valore minore di 20 è ritenuto ottimale.
			%\end{itemize}
			\subparagraph{Adequacy of variable names - MPRS8}\label{MPRS8}
			La metrica controlla che la divergenza dei nomi delle variabili da quanto deciso nella \definizionediprodottov rientri tra i valori definiti. Questo ci permette di mantenere una buona comprensibilità del codice. 
			\begin{equation}
			Adequacy \ of \ variable \ names=\frac{\#\ variabili \ con \ nomi \ corretti}{\# \ totale \ di \ variabili \ definite \ nella \ DP }
			\end{equation}
			\begin{itemize}
				\item DP = \definizionediprodottov.
			\end{itemize}
			Vengono di seguito riportati i range stabiliti per la metrica appena introdotta:
			\begin{itemize}
				\item un valore minore di 80\% è ritenuto negativo;
				\item un valore compreso tra 80\% e 90\% è ritenuto accettabile;
				\item un valore maggiore di 90\% è ritenuto ottimale.
			\end{itemize}
			\subparagraph{Average Module Size - MPRS9}\label{MPRS9}
			La metrica controlla che la dimensione media di un modulo in termini di linee di codice rientri tra i valori definiti. Avere moduli di grandi dimensioni compromette la manutenibilità del codice.
			\begin{equation}
			Average \ module \ size  = \frac{\# \ totale \ di \ linee \ di \ codice \ in \ tutti \ i \ moduli  }{\#\ moduli}
			\end{equation}
			Vengono di seguito riportati i range stabiliti per la metrica appena introdotta:
			\begin{itemize}
				\item un valore maggiore di 400 è ritenuto negativo;
				\item un valore compreso tra 300 e 400 è ritenuto accettabile;
				\item un valore compreso tra 200 e 300 è ritenuto ottimale.
			\end{itemize}
			\subparagraph{Test Passati Richiesti - MPRS10}\label{MPRS10}
			La metrica controlla che la percentuale di successo dei test ricavati dai requisiti rientri tra i valori definiti. Questo ci permette di valutare se il prodotto supera la maggior parte dei test.
			\begin{equation}
			Test \ Passati \ Richiesti  = \frac{\# \ di \ test \ passati }{\#\ totale \ di \ test \ richiesti}
			\end{equation}
			Vengono di seguito riportati i range stabiliti per la metrica appena introdotta:
			\begin{itemize}
				\item una percentuale minore del 80\% e maggiore del 98\% è ritenuta negativa;
				\item una percentuale compresa tra 80\% e 90\% è ritenuta accettabile;
				\item una percentuale compresa tra 90\% e 98\% è ritenuta ottimale.
			\end{itemize}
			Da notare che una percentuale maggiore del 98\% è ritenuta negativa, perchè lo scopo dei test è quello di trovare dei problemi. Al raggiungimento del 100\% dei test passati il set di test previsto verrà cambiato. 
			\subparagraph{Failure Avoidance - MPRS11}\label{MPRS11}
			La metrica controlla che la percentuale di situazioni anomale evitate dal prodotto rientri nei valori definiti. Questo ci permette di valutare se il prodotto è robusto e risponde bene ad eventuali situazioni anomale. 
			\begin{equation}
			Failure \ Avoidance  = \frac{\#\ situazioni \ anomale \ evitate }{\#\ totale \ di \ situazioni \ anomale \ presentate }
			\end{equation}
			Vengono di seguito riportati i range stabiliti per la metrica appena introdotta:
			\begin{itemize}
				\item una percentuale minore del 80\% è ritenuta negativa;
				\item una percentuale compresa tra 80\% e 90\% è ritenuta accettabile;
				\item una percentuale maggiore di 90\% è ritenuta ottimale.
			\end{itemize}
			\subparagraph{Breakdown Avoidance - MPRS12}\label{MPRS12}
			La metrica controlla che la percentuale di interruzioni evitate dal prodotto rientri tra i valori definiti. Il valore su cui si applicherà la metrica verrà calcolato come il complemento delle interruzioni verificate. Questa metrica ci permette di controllare che il prodotto lavori senza interruzioni.
			\begin{equation}
			Breakdown \ Avoidance  =1 - \frac{\#\ di \ interruzioni}{\#\ di \ situazioni \ anomale \ presentate }
			\end{equation}
			Vengono di seguito riportati i range stabiliti per la metrica appena introdotta:
			\begin{itemize}
				\item una percentuale minore del 80\% è ritenuta negativa;
				\item una percentuale compresa tra 80\% e 90\% è ritenuta accettabile;
				\item una percentuale maggiore di 90\% è ritenuta ottimale.
			\end{itemize}
\end{document}