\documentclass[../PianoDiQualifica.tex]{subfiles}

\begin{document}
\begin{appendices}

\section{PDCA}
	Il PDCA\g, acronimo di Plan-Do-Check-Act, conosciuto anche come "Ciclo di Deming" o "Ciclo di miglioramento continuo", è un modello studiato per il miglioramento continuo della qualità in un'ottica a lungo raggio.\\
	Questo strumento permette di fissare degli obiettivi di miglioramento a partire dagli esiti delle misurazioni effettuate durante le varie attività di verifica. Una volta fissati gli obiettivi che si desiderano raggiungere, si iterano le attività previste dal Ciclo di Deming fino al raggiungimento di tali obiettivi.\\
	I miglioramenti ai quali si fa riferimento sono legati all'efficienza e all'efficacia. Migliorare l'efficienza significa usare meno risorse per fare lo stesso lavoro. Migliorare l'efficacia significa divenire più conformi alle aspettative.\\
	Vengono riportate di seguito le quattro attività previste dal Ciclo di Deming:
	\begin{itemize}
		\item \textbf{Plan - Pianificare}: consiste nel definire gli obiettivi di miglioramento e le strategie da utilizzare per raggiungere tali obiettivi. Durante questa attività viene inoltre pianificato il modo in cui attuare queste strategie per raggiungere gli obiettivi di miglioramento fissati;
		\item \textbf{Do - Eseguire}: consiste nell'attuazione di quanto è stato pianificato al punto precedente. Oltre a fare ciò, si devono anche raccogliere i dati necessari all’analisi che viene svolta ai punti successivi;
		\item \textbf{Check - Verificare}: consiste nel verificare l'esito del processo\g\ (per efficienza ed efficacia) in seguito all'attuazione delle strategie di miglioramento. I risultati possono avere tre tipi di esito:
		\begin{itemize}
			\item un miglioramento secondo le aspettative;
			\item un miglioramento superiore alle aspettative;
			\item un miglioramento inferiore alle aspettative.
		\end{itemize}
		\item \textbf{Act - Agire}: consiste nell'attuazione di soluzioni correttive, ovvero nell'attuazione delle strategie che hanno portato miglioramenti, anche al di fuori dei singoli processi per i quali si erano stati fissati gli obiettivi di miglioramento.
	\end{itemize}
	Bisogna tener presente che se l'obiettivo è il miglioramento continuo, le attività devono essere analizzabili, ripetibili e tracciabili. 
	Unendo queste tre caratteristiche è possibile individuare eventuali errori e correggerli.
\end{appendices}
\end{document}