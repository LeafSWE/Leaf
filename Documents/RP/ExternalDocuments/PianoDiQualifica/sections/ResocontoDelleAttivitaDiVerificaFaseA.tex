\documentclass[../PianoDiQualifica.tex]{subfiles}

\begin{document}
\begin{appendices}
\section{Resoconto delle attività di verifica - fase A}
All'interno di questa prima fase\g, secondo quanto riportato nel documento \pianodiprogetto, sono previsti più momenti in cui viene attivato il processo\g\ di verifica. Si è cercato di riportare in questa sezione tutti i risultati che sono stati ottenuti durante questi momenti. Ove fosse necessario, si sono tratte anche delle conclusioni sui risultati ottenuti e su come essi possono essere migliorati.
	\subsection{Resoconto delle attività di verifica sui prodotti}
	In questa sezione verranno riportati i dati emessi dalle procedure di controllo della qualità di prodotto\g.
		\subsubsection{Documenti}
		In questa sezione vengono riportati i risultati delle attività di verifica svolte sui documenti. Esse sono di due tipi:
		\begin{itemize}
			\item verifiche manuali;
			\item verifiche automatizzate.
		\end{itemize}
			\paragraph{Verifiche manuali}
			Le attività di verifica manuale della documentazione prodotta sono state svolte in base alla procedura riguardante la verifica dei documenti che è descritta nel documento \normediprogetto. La verifica manuale ha permesso di individuare soprattutto errori che riguardano le seguenti tipologie:
			\begin{itemize}
				\item periodi troppo lunghi e complessi da capire e interpretare;
				\item aggettivi o verbi utilizzati in modo non appropriato;
				\item incongruenze tra parti diverse dello stesso documento o appartenenti a documenti diversi;
				\item errori nei concetti esposti;
				\item violazioni di quanto stabilito nelle norme tipografiche.
			\end{itemize}
			Di seguito è presentato un riassunto della quantità di errori trovati (e successivamente risolti) utilizzando la verifica manuale durante l'intera fase\g\ A.
\begin{table}[H]
		\centering
		\begin{tabular}{l * {2}{c}}
			\midrule
			Periodi lunghi o complessi &	11 \\
			Parole non appropriate & 9 \\
			Incongruenze & 15 \\
			Errori concettuali & 18 \\
			Violazioni delle norme tipografiche & 100 \\
			\midrule
		\end{tabular}
		\caption{Errori trovati tramite verifica manuale dei documenti durante la fase A}
		\label{tab:errori_manuale}
\end{table}
			La verifica manuale, in aggiunta, ha permesso di individuare nuovi termini da aggiungere al \glossario. Di seguito è presentato un riassunto della quantità di nuovi termini da aggiungere al \glossario\ che sono stati individuati.
\begin{table}[H]
		\centering
		\begin{tabular}{l * {2}{c}}
			\midrule
			Termini candidati ad essere aggiunti &	76 \\
			Termini aggiunti al \glossario & 70 \\
			\midrule
		\end{tabular}
		\caption{Nuovi termini da inserire nel \glossario\ individuati tramite verifica manuale dei documenti durante la fase A}
		\label{tab:termini_glossario}
\end{table}	
			È stata infine verificata la correttezza dei diagrammi UML\g\ utilizzati all'interno dei vari documenti, sempre seguendo le procedure contenute nel documento \normediprogetto.
			\paragraph{Verifiche automatiche}
			Le attività di verifica automatizzate, oltre a rispettare le procedure descritte all'interno delle \normediprogetto, fanno uso degli strumenti automatici previsti all'interno dello stesso documento. Questi hanno permesso di individuare numerosi errori di ortografia.\\
			Di seguito è presentato un riassunto della quantità di errori trovati (e successivamente risolti) utilizzando la verifica automatica. Si tenga in considerazione il fatto che alcuni degli strumenti automatici utilizzati non sono stati disponibili fin dall'inizio.
\begin{table}[H]
		\centering
		\begin{tabular}{l * {2}{c}}
			\midrule
			Errori ortografici &	31 \\
			\midrule
		\end{tabular}
		\caption{Errori trovati tramite verifica automatica dei documenti durante la fase A}
		\label{tab:errori_automatica}
\end{table}	
			Merita un discorso a parte il calcolo dell'indice Gulpease\g, per il quale sono stati imposti nel presente documento dei range che determinano se un documento è accettabile o meno. Di seguito sono stati riportati gli indici ottenuti (relativi ai documenti completi).
\begin{table}[H]
		\centering
		\begin{tabular}{l * {2}{c}}
			\toprule
			\textbf{Documento} & \textbf{Gulpease} & \textbf{Esito} \\
			\midrule
			\textit{Piano di progetto v1.00} & 54 &  Ottimale \\
			\textit{Norme di progetto v1.00} & 60 & Ottimale \\
			\textit{Studio di fattibilità v1.00} & 55 & Ottimale \\
			\textit{Analisi dei requisiti v1.00} & 50 & Ottimale \\
			\textit{Piano di qualifica v1.00} & 51 & Ottimale \\
			\textit{Glossario v1.00} & 67 & Ottimale \\
			%\midrule		
			\bottomrule
		\end{tabular}
		\caption{Esiti del calcolo dell’indice di leggibilità effettuato tramite strumenti automatici durante la fase A}
		\label{tab:esiti_gulpease}
\end{table}
	\subsection{Resoconto delle attività di verifica sui processi}
		\subsubsection{processo di documentazione}
			\paragraph{Livello CMM}
			Il gruppo ha cercato di valutare la qualità del processo\g\ di documentazione secondo le metriche stabilite dal modello CMM\g: chiaramente, all'inizio della fase\g\ A il processo\g\ si posizionava al livello 1.\\
			In seguito alla redazione del documento \normediprogetto\ (uno dei primi ad essere realizzato) sono state rese disponibili norme valide per ogni tipo di documentazione, strumenti comuni da poter utilizzare e procedure da seguire per effettuare determinate attività. Questo ha permesso di controllare maggiormente il processo\g\ di documentazione, che ha in questo modo guadagnato ripetibilità (richiesta dal livello 2 di CMM\g).
			Possiamo quindi affermare di aver raggiunto il livello 2 della scala CMM\g, perché il processo\g\ di documentazione non possiede ancora la principale caratteristica richiesta dal terzo livello, ovvero la proattività.\\
			Questo livello è ritenuto accettabile secondo quanto descritto nel presente documento alla sezione \ref{MisureMetriche} "Misure e metriche", ma, durante le prossime fasi, si prevede comunque di continuare a lavorare per poter ottenere miglioramenti sotto questi punti di vista (sfruttando PDCA\g).
		\subsubsection{processo di verifica}
			\paragraph{Livello CMM}
			Essendo il processo\g\ di verifica molto costoso, il nostro obiettivo è di renderlo il più efficace e allo stesso tempo il più efficiente possibile. Per ottenere ciò si deve rendere il processo\g\ controllabile.\\
			Anche per quanto riguarda il processo\g di verifica, come per quello di documentazione, siamo in grado di dire che è stato raggiunto il livello 2 nella scala prevista da CMM\g. Il processo\g\ ha infatti superato l'iniziale stato caotico nel quale si trovava all'inizio della fase\g\ A (grazie, per esempio, all'utilizzo sistematico di script e di procedure). \\
			Il team\g\ non può ancora affermare che il processo\g\ di verifica adottato abbia raggiunto il livello 3 della Scala\g\ CMM\g, in quanto è stata documentata in modo accettabile solo l'attività di realizzazione del processo\g\ e non quella di gestione dello stesso. Tuttavia il livello raggiunto è ritenuto accettabile secondo quanto descritto nel presente documento alla sezione \ref{MisureMetriche} "Misure e metriche", anche se, durante le prossime fasi, si prevede comunque di continuare a lavorare per poter ottenere miglioramenti sotto questi punti di vista (sfruttando PDCA\g).
\end{appendices}
\end{document}