\documentclass[../PianoDiQualifica.tex]{subfiles}

\begin{document}
\begin{appendices}
\section{Resoconto delle attività di verifica - Fase AD}
All'interno di questa prima fase\g, secondo quanto riportato nel documento \pianodiprogettov, sono previsti più momenti in cui viene attivato il processo\g\ di verifica. Si è cercato di riportare in questa sezione tutti i risultati che sono stati ottenuti durante questi momenti. Ove fosse necessario, si sono tratte anche delle conclusioni sui risultati ottenuti e su come essi possono essere migliorati.
	
	\subsection{Verifica sui processi}
		\subsubsection{Processo di documentazione}
			\paragraph{Miglioramento costante}
			All'inizio della fase AD il processo di documentazione si posizionava al livello 2 della scala CMM.\\
			In seguito alla riorganizzazione delle \normediprogettov\ e grazie ad una maggiore esperienza dei membri del gruppo, i processi e la loro organizzazione sono notevolmente migliorati. Ciò ha permesso di raggiungere il terzo livello CMM.\\
			In questa fase abbiamo, inoltre, iniziato a misurare la qualità dei processi ampliando le metriche utilizzate e gli obiettivi di qualità scelti, fissandoli in modo quantitativo.
			
			\paragraph{Rispetto della pianificazione}
			Per capire se le attività di un processo sono in ritardo rispetto a quanto pianificato all’interno del \pianodiprogetto\ viene utilizzata la seguente metrica: Schedule Variance.\\
			Si desidera che il ritardo accumulato sia minore del 5\% rispetto al totale pianificato. Sarebbe invece ottimale essere esattamente in linea con quanto prevede il \pianodiprogetto, o essere addirittura in anticipo.\\
			Di seguito sono riportati i valori ottenuti calcolando la Schedule Variance sui tempi di stesura di ogni documento nella fase AD:
			\begin{table}[H]
				\centering
				\begin{tabular}{l * {2}{c}}
					\toprule
					\textbf{Documento} & \textbf{Schedule Variance} & \textbf{Esito} \\
					\midrule
					\textit{Piano di progetto v2.00} & -43\% &  Ottimale \\
					\textit{Norme di progetto v2.00} & -26\% & Ottimale \\
					\textit{Analisi dei requisiti v2.00} & 0\% & Ottimale \\
					\textit{Piano di qualifica v2.00} & 0\% & Ottimale \\
					\bottomrule
				\end{tabular}
				\caption{Esiti del calcolo della Schedule Variance sul processo di documentazione durante la fase\g\ AD}
				\label{tab:esiti_schedule_variance}
			\end{table}
			
			\paragraph{Rispetto del budget}
			------------------------------------------------
			
		\subsubsection{Processo di verifica}
			\paragraph{Miglioramento costante}
			Nonostante l'adozione di nuove metriche e una maggiore regolamentazione del processo di verifica, il team non è ancora in grado di individuare miglioramenti tali da raggiungere il terzo livello CMM, pertanto il processo di verifica rimane al secondo livello (Ripetibile).
			
			\paragraph{Rispetto della pianificazione}
			Per capire se le attività di un processo sono in ritardo rispetto a quanto pianificato all’interno del \pianodiprogetto\ viene utilizzata la seguente metrica: Schedule Variance.\\
			Si desidera che il ritardo accumulato sia minore del 5\% rispetto al totale pianificato. Sarebbe invece ottimale essere esattamente in linea con quanto prevede il \pianodiprogetto, o essere addirittura in anticipo.\\
			Di seguito sono riportati i valori ottenuti calcolando la Schedule Variance sui tempi di verifica nella fase AD:
			\begin{table}[H]
				\centering
				\begin{tabular}{l * {2}{c}}
					\toprule
					\textbf{Processo} & \textbf{Schedule Variance} & \textbf{Esito} \\
					\midrule
					\textit{Processo di verifica} & -15\% &  Ottimale \\
					\bottomrule
				\end{tabular}
				\caption{Esiti del calcolo della Schedule Variance sul processo di verifica durante la fase\g\ AD}
				\label{tab:esiti_schedule_variance}
			\end{table}
			
			\paragraph{Rispetto del budget}
			---------------------------------------------------
			
			
	\subsection{Verifica sui prodotti}
	In questa sezione verranno riportati i dati emessi dalle procedure di controllo della qualità di prodotto\g.
		\subsubsection{Documenti}
		In questa sezione vengono riportati gli esiti delle attività di verifica svolte sui documenti.\\
		Tali esiti sono strettamente correlati agli obiettivi di qualità dei documenti enunciati alla sezione \ref{ObiettiviDiQualità} del presente documento.
			
			\paragraph{Leggibilità e comprensibilità}
			Per cercare di capire quanto i documenti siano effettivamente leggibili e comprensibili da persone dotate di una licenza superiore viene utilizzato l’indice Gulpease\g.\\
			Si desidera che i documenti posseggano costantemente un indice maggiore a 40 (soglia di accettabilità). Si dovrebbe tuttavia cercare di raggiungere un valore più alto, considerato ottimale, ovvero 60.\\
			Il documento \glossariov\ ha dato esito \textbf{ottimale}, mentre tutti gli altri documenti prodotti hanno dato esito \textbf{accettabile}.
			
			\paragraph{Correttezza ortografica}
			Per capire quanto i documenti siano effettivamente corretti a livello ortografico viene utilizzata la seguente metrica: percentuale di errori ortografici rinvenuti e non corretti.\\
			Si desidera che tutti gli errori ortografici che sono stati trovati siano corretti. In questo caso, dunque, l'obiettivo minimo coincide con l’obiettivo ottimale.\\
			Di seguito sono riportati gli errori ortografici trovati tramite verifica automatica dei documenti durante la fase AD.
			\begin{table}[H]
				\centering
				\begin{tabular}{l * {2}{c}}
					\midrule
					Errori ortografici & 11 \\
					\midrule
				\end{tabular}
				\caption{Errori ortografici trovati tramite verifica automatica dei documenti durante la fase\g\ AD}
				\label{tab:errori_automatica}
			\end{table}
			Tutti gli errori ortografici rinvenuti sono stati corretti, quindi è stato raggiunto l'obiettivo \textbf{ottimale}.
			
			\paragraph{Correttezza concettuale}
			Per capire quanto i documenti siano effettivamente corretti a livello concettuale viene utilizzata la seguente metrica: percentuale di errori concettuali rinvenuti e non corretti.\\
			Si desidera che al massimo il 5\% degli errori concettuali rinvenuti non siano corretti. L’obiettivo ottimale sarebbe quello di correggere tutti gli errori trovati. \\
			Di seguito sono riportati gli errori concettuali trovati dei documenti durante la fase AD.
			\begin{table}[H]
				\centering
				\begin{tabular}{l * {2}{c}}
					\midrule
					Errori concettuali & 5 \\
					\midrule
				\end{tabular}
				\caption{Errori concettuali trovati tramite verifica manuale dei documenti durante la fase\g\ AD}
				\label{tab:errori_concettuali}
			\end{table}
			Tutti gli errori concettuali rinvenuti sono stati corretti, quindi è stato raggiunto l'obiettivo \textbf{ottimale}.
			
\end{appendices}
\end{document}