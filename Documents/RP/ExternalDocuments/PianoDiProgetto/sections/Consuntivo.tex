\documentclass[../PianoProgetto.tex]{subfiles}

\begin{document}

\section{Consuntivo di periodo}
\label{sec:consuntivo}

	Verranno indicate di seguito le spese effettivamente sostenute, sia per ruolo che per persona.
	 
	Il bilancio risultante potrà essere: 
	\begin{itemize}
		\item \textbf{positivo}: se il preventivo supera il consuntivo;
		\item \textbf{in pari}: se consuntivo e preventivo sono equivalenti;
		\item \textbf{negativo}: se il consuntivo supera il preventivo;
	\end{itemize}

	\subsection{Fase A}
		\subsubsection{Consuntivo}
		Le ore di lavoro sostenute in questa fase sono da considerarsi come ore di approfondimento personale, in quanto il gruppo \leaf{} non è ancora stato scelto come fornitore ufficiale per il progetto \progetto.
		
		Tali dati riguardano quindi le ore non rendicontate.

		
\begin{table}[h]
		\centering
		\begin{tabular}{l * {2}{c}}
			\toprule
			\textbf{Ruolo} & \textbf{Ore} & \textbf{Costo (\euro{})} \\
			\midrule
			Responsabile &	33 (+7) & 990,00 (+210,00)\\
			%\midrule
			Amministratore & 87 (+12) & 1.740,00 (+240,00)\\
			%\midrule
			Progettista & 0 & 0,00 \\
			%\midrule
			Analista & 86 (+3) & 2.150,00 (+75,00)\\
			%\midrule
			Programmatore & 0 & 0,00 \\
			%\midrule
			Verificatore & 74 (-14) & 1.110,00 (-210,00)\\
			\midrule
			\textbf{Totale Preventivo} & 280
 & 5.990,00
 \\		
			\textbf{Totale Consuntivo} & 288 & 6.305,00
 \\
			\midrule
			\textbf{Differenza} & +8 & +315,00 \\
			\bottomrule
		\end{tabular}
		
		\caption{Fase A - Consuntivo}
		\label{tab:consuntivoA}
		
	\end{table}		
		
		\subsubsection{Conclusioni}	
		Come si evince dalla tabella \ref{tab:consuntivoA}, che presenta i dati relativi al consuntivo della fase A, è stato necessario investire più tempo del previsto nei ruoli di \responsabilediprogetto{}, \amministratore{} e \analista, di conseguenza il bilancio risultante è \textbf{negativo}.
		
		L'attività del \responsabilediprogetto{} ha richiesto più tempo del previsto a causa dell'inesperienza nell'ambito della pianificazione e della mancanza di progetti conosciuti sui quali basare la preventivazione dei costi.		
		
		L'attività degli \amministratori{} ha richiesto più tempo del previsto in quanto è stato necessario apportare modifiche non banali al software adottato per il tracciamento dei requisiti.
		
		L'attività degli \analisti{} ha richiesto più tempo del previsto in quanto il capitolato scelto richiede una buona dose di innovazione e ricerca che, in questa fase, ha impattato sulla specifica dei casi d'uso e dei requisiti.

	\subsection{Fase AD}
	\subsubsection{Consuntivo}
	Il gruppo dopo aver affrontato la \revisionedeirequisiti\ è diventato fornitore ufficiale. Le ore prese in considerazione sono ore rendicontate. 
	
	\begin{table}[h]
		\centering
		\begin{tabular}{l * {2}{c}}
			\toprule
			\textbf{Ruolo} & \textbf{Ore} & \textbf{Costo (\euro{})} \\
			\midrule
			Responsabile &	9 (+1) & 270,00 (+30,00)\\
			%\midrule
			Amministratore &  14 (+6) & 280,00 (+120,00)\\
			%\midrule
			Progettista & 0 & 0,00 \\
			%\midrule
			Analista & 20 (+9) & 500,00 (+225,00)\\
			%\midrule
			Programmatore & 0 & 0,00 \\
			%\midrule
			Verificatore & 38 (-11) & 570,00 (-165,00)\\
			\midrule
			\textbf{Totale Preventivo} & 81
			& 1620,00
			\\		
			\textbf{Totale Consuntivo} & 86 & 1830,00 
			\\
			\midrule
			\textbf{Differenza} & +5 & +210,00 \\
			\bottomrule
		\end{tabular}
		
		\caption{Fase AD - Consuntivo}
		\label{tab:consuntivoAD}
		
	\end{table}		
	
	\subsubsection{Conclusioni}	
     Anche in questa fase il consuntivo ha avuto esito \textbf{negativo}.
     Le ore spese in più dal gruppo derivano da una pianificazione non particolarmente precisa che non ha tenuto conto degli imprevisti presentati nei primi cinque giorni del periodo.
     Come da consuntivo si notano che sono state spese ore non previste nei ruoli di \amministratore\ e \analista. Queste ore sono state impiegate per effettuare le correzioni comunicate dal committente e nel caso degli \amministratori\ c'è stato il bisogno di rivedere il \pianodiqualifica. 
	
     \subsection{Fase PA}
	\subsubsection{Consuntivo}
	Il gruppo dopo aver superato la fase AD è passato nella fase nella quale ha dovuto effettuare la progettazione architetturale del software\g. Le ore prese in considerazione sono ore rendicontate. 
	
	\begin{table}[h]
		\centering
		\begin{tabular}{l * {2}{c}}
			\toprule
			\textbf{Ruolo} & \textbf{Ore} & \textbf{Costo (\euro{})} \\
			\midrule
			Responsabile &		10 (-1) & 300,00  (-30,00) \\
			%\midrule
			Amministratore &	19 (-2) & 380,00  (-40,00) \\
			%\midrule
			Progettista & 		45 (+7) & 990,00  (+154,00)\\
			%\midrule
			Analista & 			75	(-10)	& 1625,00   (-250,00)       \\
			%\midrule
			Programmatore & 	0		& 0,00 				\\
			%\midrule
			Verificatore & 		18 (-4) & 270,00 (-60,00)	\\
			\midrule
			\textbf{Totale Preventivo} & 167
			& 3815,00
			\\		
			\textbf{Totale Consuntivo} & 161 & 3565,00 
			\\
			\midrule
			\textbf{Differenza} & -6 & -250,00 \\
			\bottomrule
		\end{tabular}
		
		\caption{Fase PA - Consuntivo}
		\label{tab:consuntivoPA}
		
	\end{table}		
	
	\subsubsection{Conclusioni}	
     In questa fase, l'esito del consuntivo è \textbf{positivo}. 
     Ciò è dovuto principalmente ad una ripianificazione delle ore effettuata nella fase AD e ad una sovrastima delle ore da \analista. D'altro canto, le ore del \progettista\ (ruolo principale durante questa fase), sono state sottostimate.
\end{document}
