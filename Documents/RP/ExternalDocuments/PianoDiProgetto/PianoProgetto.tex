%ricordarsi di compilare più di una volta
\documentclass[a4paper,12pt]{article}

\usepackage{leaf}
\setcounter{table}{-1}
\usepackage{float}
\restylefloat{table}

\titlepage{}

\author{Eduard Bicego, Federico Tavella, Andrea Tombolato}
\date{20/01/2016}
\intestazioni{Piano di Progetto}
\pagenumbering{gobble}
\begin{document}
	\begin{titlepage}
		\centering
		{\huge\bfseries CLIPS\par}
		Communication \& Localization with Indoor Positioning Systems \\
		\line(1,0){350} \\
		{\scshape\LARGE Università di Padova \par}
		\vspace{1cm}
		{\scshape\Large Piano di progetto v2.00\par}
		\logo
		%\vspace{2cm}
		%{\Large\itshape Eduard Bicego, Federico Tavella, Andrea Tombolato\par}
	
		\vfill \vfill
		% devono essere compilati questi campi ogni volta
		\begin{tabular}{c|c}
			{\hfill \textbf{Versione}} 			&	2.00\\
			{\hfill\textbf{Data Redazione}} 	& 	2016-02-28\\
			{\hfill\textbf{Redazione}} 			&   Cristian Andrighetto \\ 
			{\hfill\textbf{Verifica}} 			& 	Marco Zanella 	\\
			{\hfill\textbf{Approvazione}} 		& 	Cristian Andrighetto \\
			{\hfill\textbf{Uso}} 				& Esterno			\\
			{\hfill\textbf{Distribuzione}} 		& Prof. Vardanega Tullio \\
												& Prof. Cardin Riccardo \\
												& Miriade S.p.A. \\
	\end{tabular}
\end{titlepage}

	\pagestyle{myfront}	
	
	\newpage
		
\newcolumntype{V}{>{\hsize=.40\hsize}X[cm]}
	\section*{Diario delle modifiche}
\begin{longtabu} to \textwidth {V X[c m 0.8cm] X[c m 0.6cm] X[c m 0.8cm] X[cm]}
	\toprule
	\textbf{Versione} & \textbf{Data}  & \textbf{Autore} & \textbf{Ruolo} & \textbf{Descrizione}\\
	\midrule
	\endhead
	\arrayrulecolor{gray}
2.01 & 2016-03-01 & Marco Zanella & Responsabile di Progetto & Aggiunta analisi dei rischi per la fase di progettazione architetturale \\ 
\midrule
2.00 & 2016-02-19 & Cristian Andrighetto & Responsabile di Progetto & Approvazione del documento \\ 
\midrule
1.08 & 2016-02-19 & Marco Zanella & Verificatore & Verifica del documento \\
\midrule
1.07 & 2016-02-18 & Cristian Andrighetto & Responsabile di Progetto & Consuntivo Fase AD \\
\midrule
1.06 & 2016-02-18 & Cristian Andrighetto & Responsabile di Progetto & Incremento Analisi dei Rischi - "Possibili Conseguenze" e "Riscontro Effettivo" - Periodo "Analisi di Dettaglio" per i rischi da 3.1.1 a 3.5.1 \\ 
\midrule
1.05 & 2016-02-18 & Cristian Andrighetto & Responsabile di Progetto & Riformulazione forma della parte introduttiva dell' "Analisi dei rischi", aggiunti riferimenti ipertestuali ad ogni rischio nel registro dei rischi \\ 
\midrule
1.04 & 2016-02-17 & Cristian Andrighetto & Responsabile di Progetto & Corretta sezione "Analisi dei rischi", aggiunte tabelle analisi per fase e tabelle riscontro effettivo per fase \\ 
\midrule
1.03 & 2016-02-16 & Cristian Andrighetto & Responsabile di Progetto & Correzione delle date di consegna nella sezione "Pianificazione" \\ 
\midrule
1.02 & 2016-02-16 & Cristian Andrighetto & Responsabile di Progetto & Aggiunta esplicitazione della RP in cui si intende consegnare nella sezione "Scadenze" \\ 
\midrule
1.02 & 2016-02-16 & Cristian Andrighetto & Responsabile di Progetto & Aggiunta esplicitazione della RP in cui si intende consegnare nella sezione "Scadenze" \\ 
\midrule
1.01 & 2016-02-16 & Cristian Andrighetto & Responsabile di Progetto & Sostituzione termine "Ciclo di sviluppo" con "Modello di sviluppo" nella sezione "Modello di sviluppo" \\ 
\midrule
1.00 & 2016-01-20 & Federico Tavella & Responsabile di Progetto & Approvazione del documento \\ 
\midrule
0.19 & 2016-01-20 & Davide Castello & Verificatore & Verifica consuntivo \\ 
\midrule
0.18 & 2016-01-19 & Federico Tavella & Responsabile di Progetto & Stesura consuntivo \\ 
\midrule
0.17 & 2016-01-10 & Eduard Bicego & Responsabile di Progetto & Inserimento Bar Chart e Pie Chart in Preventivo \\ 
\midrule
0.16 & 2016-01-09 & Eduard Bicego & Responsabile di Progetto & Impaginazione generale documento migliorata \\ 
\midrule
0.15 & 2016-01-07 & Federico Tavella & Responsabile di Progetto & Correzione diagrammi di Gantt nella Pianificazione \\ 
\midrule
0.14 & 2016-01-06 & Davide Castello & Verificatore & Verifica intero documento \\ 
\midrule
0.13 & 2016-01-03 & Andrea Tombolato & Responsabile di Progetto & Aggiunta sezione Meccanismi di controllo e rendicontazione \\ 
\midrule
0.12 & 2016-01-02 & Andrea Tombolato & Responsabile di Progetto & Fine stesura preventivo \\ 
\midrule
0.12 & 2016-01-02 & Andrea Tombolato & Responsabile di Progetto & Fine stesura preventivo \\ 
\midrule
0.11 & 2015-12-28 & Andrea Tombolato & Responsabile di Progetto & Inizio stesura preventivo \\ 
\midrule
0.09 & 2015-12-27 & Federico Tavella & Responsabile di Progetto & Aggiunti diagrammi di Gantt mancanti nella Pianificazione \\ 
\midrule
0.08 & 2015-12-23 & Federico Tavella & Responsabile di Progetto & Stesura Pianificazione fase PDRD, PDROP, V \\ 
\midrule
0.07 & 2015-12-20 & Federico Tavella & Responsabile di Progetto & Stesura Pianificazione fase PA, PDROB e aggiunti diagrammi di Gantt fase A, AD, PA e PDROB \\ 
\midrule
0.06 & 2015-12-16 & Federico Tavella & Responsabile di Progetto & Stesura Pianificazione fase A, AD \\ 
\midrule
0.05 & 2015-12-11 & Federico Tavella & Responsabile di Progetto & Inizio stesura sezione Pianificazione \\ 
\midrule
0.04 & 2015-12-11 & Federico Tavella & Responsabile di Progetto & Stesura sezione Analisi dei rischi  \\ 
\midrule
0.03 & 2015-12-11 & Andrea Tombolato & Responsabile di Progetto & Stesura sezione Ciclo di sviluppo  \\ 
\midrule
0.02 & 2015-12-10 & Andrea Tombolato & Responsabile di Progetto & Stesura sezione introduttiva del documento e Organigramma \\ 
\midrule
0.01 & 2015-12-09 & Andrea Tombolato & Responsabile di Progetto & Stesura struttura documento \\ 

\arrayrulecolor{black}
	\bottomrule
\end{longtabu}
	
	\newpage
		\tableofcontents 	% produce l'indice delle sezioni del documento
	\newpage
		\listoftables 		% produce l'indice delle tabelle del documento
	\newpage	
		\listoffigures		% produce l'indice delle figure del documento	
	
	\label{LastFrontPage}

	\newpage
		\pagestyle{mymain}
	
	\newpage
		\subfile{sections/Introduzione}
	\newpage
		\subfile{sections/Scadenze}

	\newpage
		\subfile{sections/AnalisiDeiRischi}
		
	\newpage
		\subfile{sections/ModelloDiSviluppo}
		
	\newpage
		\subfile{sections/Pianificazione}	
		
	\newpage
		\subfile{sections/MeccanismiDiControllo}	

    \newpage
		\subfile{sections/Preventivo}
	\newpage
		\subfile{sections/Consuntivo}
        		
	\label{LastPage}

\end{document}
