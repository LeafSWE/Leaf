\documentclass[../DefinizioneDiProdotto.tex]{subfiles}
\begin{document}
\section{Specifica dei componenti}

	\subsection{Metodo e formalismo di specifica}

		L'esposizione dell'architettura in dettaglio dell'applicazione è esposta di seguito seguendo un approccio top-down a livelli. Si descrive quindi l'architettura partendo dal generale esponendo inizialmente le componenti più teoriche: i package fino a quelle più concrete: le classi con i relativi metodi, attributi e relazioni di ereditarietà.
		
		Per ogni package si specifica:
			\begin{itemize}
				\item il nome;
				\item una descrizione;
				\item il package da cui discende;
				\item le interazioni con gli altri package;
				\item gli eventuali package contenuti;
				\item le classi contenute affiancate da un riferimento alla descrizione completa.
			\end{itemize}
		Per ogni classe si specifica:
			\begin{itemize}
				\item il nome;
				\item il tipo;
				\item l'eventuale classe che estende;
				\item le eventuali interfacce che implementa;
				\item la visibilità;
				\item una descrizione;
				\item la lista dettagliata degli attributi;
				\item la lista dettagliata dei metodi.
			\end{itemize}
		Per i diagrammi dei package e delle classi si utilizza il formalismo \textit{UML 2.0}.
	
	% componenti da esportare da Tracy
	
	
	
\end{document}