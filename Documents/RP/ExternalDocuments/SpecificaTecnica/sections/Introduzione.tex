\documentclass[../SpecificaTecnica.tex]{subfiles}
\begin{document}
\section{Introduzione}
	\subsection{Scopo del documento}
	
	\subsection{Glossario} \label{sec:Glossario}
	Allo scopo di rendere più semplice e chiara la comprensione dei documenti viene allegato il \glossariov\ nel quale verranno raccolte le spiegazioni di  terminologia tecnica o  ambigua,
	abbreviazioni ed acronimi. Per evidenziare un termine presente in tale documento, esso verrà marcato con il pedice \g.
	\subsection{Riferimenti utili}
		\subsubsection{Riferimenti normativi}
		\begin{itemize}
			\item rif
		\end{itemize}
		\subsubsection{Riferimenti informativi}
		\begin{itemize}
			\item Documentazione Android: \\ \url{http://developer.android.com/training/index.html};
			\item Documentazione Java: \\ \url{https://www.java.com/it/about/};
			\item Documentazione SQLite: \\ \url{http://developer.android.com/reference/android/database/sqlite/package-summary.html} \\ \url{https://it.wikipedia.org/wiki/SQLite} \\ \url{https://www.sqlite.org/about.html};
			\item Documentazione AltBeacon: \\ \url{https://altbeacon.github.io/android-beacon-library/index.html} \\ \url{https://github.com/AltBeacon/spec};
			\item Documentazione JGraphT: \\ \url{http://jgrapht.org/}.
			
		\end{itemize}
\end{document}