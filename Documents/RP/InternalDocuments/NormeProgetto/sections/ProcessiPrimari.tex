\documentclass[../NormeProgetto.tex]{subfiles}

\begin{document}
\section{Processi primari}
	\subsection{Fornitura}
		\subsubsection{Studio di fattibilità}
			Il documento riguardante lo studio di fattibilità deve essere redatto rapidamente ed in modo accurato dagli \analisti\ sulla base di ciò che è emerso nelle prime riunioni, nelle quali si deve discutere di temi riguardanti i capitolati, come:
			\begin{itemize}
				\item rischi nell'affrontare ogni capitolato\g;
				\item rapporto tra i costi ed i benefici, sia in base al mercato attuale che futuro, sia in base al costo di produzione e alla possibile redditività futura;
				\item il dominio applicativo e tecnologico di ogni capitolato\g.
			\end{itemize}
	\subsection{Sviluppo}
		\subsubsection{Analisi dei requisiti}
			L'\analisideirequisiti\ è il documento dove devono essere catalogati e descritti tutti i requisiti che il prodotto\g\ finale deve soddisfare. Ogni requisito deve emergere da una delle seguenti fonti:
			\begin{itemize}
				\item capitolati\g\ d'appalto;
				\item incontri con il proponente;
				\item incontri con il committente;
				\item valutazioni effettuate durante riunioni interne al gruppo.
			\end{itemize}
			Tale documento deve inoltre riportare il modo in cui ogni requisito deve essere verificato.
				\paragraph{Classificazione dei casi d'uso}
					È compito degli \analisti\ redigere una descrizione, dare una classificazione e fornire un diagramma conforme allo standard UML\g\ per ogni caso d'uso. Ogni caso d'uso dev'essere descritto con le seguenti informazioni, possibilmente in quest'ordine:
					\begin{enumerate}
						\item codice identificativo del caso d'uso, nella forma \begin{center}\textbf{UC[X].[Y]}\end{center} dove:
						\begin{itemize}
							\item \textbf{X} è il codice univoco del padre;
							\item \textbf{Y} è un codice progressivo di livello.
						\end{itemize}
						Il codice progressivo può includere diversi livelli di gerarchia separati da un punto.
						\item titolo, che deve descrivere sinteticamente il caso d'uso;
						\item attori principali;
						\item attori secondari, se questi sono presenti;
						\item precondizioni, ovvero le condizioni che necessariamente devono verificarsi prima del caso d'uso;
						\item postcondizioni, ciò che deve essere verificato successivamente al caso d'uso;
						\item flusso principale degli eventi, dove si descrive il flusso dei casi d'uso figli. Per ogni evento va specificato:
						\begin{itemize}
							\item una descrizione testuale dell'evento;
							\item gli attori coinvolti;
							\item se l'azione è descritta dettagliatamente da un altro caso d'uso.
						\end{itemize}
						\item Scenari alternativi, ovvero scenari in cui si verificano eccezioni o errori. Per ognuno di questi deve essere indicato:
							\begin{itemize}
								\item una descrizione testuale dell'evento;
								\item gli attori coinvolti;
								\item se l'azione è descritta dettagliatamente da un altro caso d'uso.
							\end{itemize}
					\end{enumerate}
				\paragraph{Classificazione dei requisiti}
					È compito degli \analisti\ redigere e classificare i requisiti del prodotto\g.  I requisiti devono essere classificati in base al tipo e alla priorità, utilizzando la seguente notazione:
					\begin{center}\textbf{R[X][Y][Z]}\end{center} dove:
						\begin{enumerate}
							\item \textbf{X} indica l'importanza strategica del requisito. Deve assumere solo i seguenti valori:
							\begin{itemize}
								\item \textbf{Obb}: Indica un requisito obbligatorio;
								\item \textbf{Des}: Indica un requisito desiderabile;
								\item \textbf{Opz}: Indica un requisito opzionale.
							\end{itemize}
							\item \textbf{Y} indica la tipologia del requisito. Deve assumere solo i seguenti valori:
							\begin{itemize}
								\item \textbf{F}: Indica un requisito funzionale;
								\item \textbf{Q}: Indica un requisito di qualità;
								\item \textbf{P}: Indica un requisito prestazionale;
								\item \textbf{V}: Indica un requisito vincolo.
							\end{itemize}
							\item \textbf{Z} rappresenta il codice univoco di ogni requisito in forma gerarchica.
						\end{enumerate}
		\subsubsection{Progettazione}
			Norme, procedure e strumenti riguardanti la progettazione verranno definiti nelle versioni successive di questo documento.
		\subsubsection{Codifica}
			Norme, procedure e strumenti riguardanti la codifica verranno definiti nelle versioni successive di questo documento.
		
		\subsubsection{Strumenti}
			\paragraph{Strumento per la creazione dei diagrammi UML}
			Lo strumento per la creazione dei diagrammi UML\g\ utilizzato è Astah\g.
			\paragraph{Strumento per il tracciamento dei requisiti}
			Lo strumento scelto per la il tracciamento dei requisiti è Tracy\g. Questo software\g\ è stato sviluppato dal gruppo di Ingegneria del Software Don't Panic. Il software\g\ è stato scelto per le seguenti caratteristiche:
			\begin{itemize}
				\item open source\g;
				\item tracciamento dei requisiti;
				\item tracciamento use case;
				\item tracciamento delle fonti;
				\item stesura automatica in \LaTeX\g\ dei requisiti.
			\end{itemize}
                        Poiché questo software\g\ non risulta essere completamente perfetto, il gruppo ha previsto di riadattarlo
 sulla base delle esigenze che sono emerse durante la stesura dell'\analisideirequisiti.
			
\end{document}

