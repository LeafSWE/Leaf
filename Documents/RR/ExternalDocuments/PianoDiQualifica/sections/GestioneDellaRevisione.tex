\documentclass[../PianoDiQualifica.tex]{subfiles}

\begin{document}
\section{Gestione della revisione}
	\subsection{Definizione e comunicazione delle anomalie}
	Un'anomalia si genera nel momento in cui si verificano incongruenze o errori.  
	Esempi di anomalie possono corrispondere a:
	\begin{itemize}
	\item un errore concettuale all'interno della documentazione di progetto;
	\item un errore ortografico;
	\item una violazione delle norme tipografiche riportate all'interno del documento \textit{Norme di Progetto v1.00};
	\item un'uscita dai range di accettazione descritti nella sezione Misure e metriche del presente
	documento;
	\item un'incongruenza nel prodotto software rispetto alle funzionalit� descritte all'interno del
	documento \textit{Analisi dei Requisiti v1.00};
	\item un'incongruenza del codice rispetto a quanto � stato progettato.
	\end{itemize}
	Per segnalare ogni anomalia va sollevata una issue tramite la piattaforma GitHub.
\end{document}