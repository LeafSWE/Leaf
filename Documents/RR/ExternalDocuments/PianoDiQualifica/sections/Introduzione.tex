\documentclass[../PianoDiQualifica.tex]{subfiles}

\begin{document}
\section{Introduzione}
	\subsection{Scopo del documento}
	Il presente documento ha l'obiettivo di fissare gli obiettivi di qualit� di prodotto e di processo e le strategie che il gruppo Leaf ha deciso di adottare per perseguirli. Questo documento dar� inoltre una visione di come il gruppo affronter� le varie fasi di verifica per poter conseguire il miglior risultato possibile in termini di qualit�.
	\subsection{Scopo del prodotto}
	Lo scopo del prodotto � implementare un metodo di navigazione indoor che sia funzionale alla tecnologia BLE.
	Il prodotto comprender� un prototipo software che permetta la navigazione all'interno di un'area predefinita, basandosi sui concetti di IPS e smart places.
	\subsection{Glossario}
	Allo scopo di rendere pi� semplice e chiara la comprensione dei documenti viene allegato il \textit{Glossario v1.00} nel quale verranno raccolte le spiegazioni di  terminologia tecnica o ambigua,
abbreviazioni ed acronimi. Per evidenziare un termine presente in tale documento, esso verr� marcato con il pedice \g.
	\subsection{Riferimenti utili}
		\subsubsection{Riferimenti normativi}
		\begin{itemize}
			\item \textbf{Norme di Progetto:} \textit{Norme di Progetto v1.00};
			\item \textbf{Standard [ISO/IEC 9126:2001]:} \\\url{https://en.wikipedia.org/wiki/ISO/IEC\_9126};
			\item \textbf{Capability Maturity Model (CMM):} \\\url{https://en.wikipedia.org/wiki/Capability\_Maturity\_Model};
			\item \textbf{Plan-Do-Check-Act (PDCA):} \\\url{https://en.wikipedia.org/wiki/PDCA}.
		\end{itemize}
		\subsubsection{Riferimenti informativi}
		\begin{itemize}
			\item \textbf{Piano di Progetto:} \textit{Piano di Progetto v1.00};
			\item \textbf{Indice Gulpease:} \url{https://it.wikipedia.org/wiki/Indice\_Gulpease};
			\item \textbf{Slide del corso di Ingegneria del Software relative alla qualit� del software:} \\\url{http://www.math.unipd.it/~tullio/IS-1/2015/Dispense/L08.pdf};
			\item \textbf{Slide del corso di Ingegneria del Software relative alla qualit� del processo:} \\\url{http://www.math.unipd.it/~tullio/IS-1/2015/Dispense/L09.pdf}.
		\end{itemize}
\end{document}