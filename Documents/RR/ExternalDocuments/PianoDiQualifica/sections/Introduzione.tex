\documentclass[../PianoDiQualifica.tex]{subfiles}

\begin{document}
\section{Introduzione}
	\subsection{Scopo del documento}
	Il presente documento ha l'obiettivo di fissare gli obiettivi di qualità e le strategie che il gruppo \leaf\ ha deciso di adottare per perseguirli. Questo documento darà inoltre una visione di come il gruppo affronterà le varie fasi di verifica per poter conseguire il miglior risultato possibile in termini di qualità.
	\subsection{Scopo del prodotto}
	Lo scopo del prodotto\g\ è implementare un metodo di navigazione indoor\g\ che sia funzionale alla tecnologia BLE\g.
	Il prodotto\g\ comprenderà un prototipo software\g\ che permetta la navigazione all'interno di un'area predefinita, basandosi sui concetti di IPS\g\ e smart places\g.
	\subsection{Glossario}
	Allo scopo di rendere più semplice e chiara la comprensione dei documenti viene allegato il \glossariov\ nel quale verranno raccolte le spiegazioni di  terminologia tecnica o ambigua,
abbreviazioni ed acronimi. Per evidenziare un termine presente in tale documento, esso verrà marcato con il pedice \g.
	\subsection{Riferimenti utili}
		\subsubsection{Riferimenti normativi}
		\begin{itemize}
			\item \textbf{Norme di Progetto:} \normediprogettov;
			\item \textbf{Standard [ISO/IEC 9126:2001]:} \\\url{https://en.wikipedia.org/wiki/ISO/IEC\_9126};
			\item \textbf{Capability Maturity Model (CMM\g):} \\\url{https://en.wikipedia.org/wiki/Capability\_Maturity\_Model};
			\item \textbf{Plan-Do-Check-Act (PDCA\g):} \\\url{https://en.wikipedia.org/wiki/PDCA}.
		\end{itemize}
		\subsubsection{Riferimenti informativi}
		\begin{itemize}
			\item \textbf{Piano di Progetto:} \pianodiprogettov;
			\item \textbf{indice Gulpease\g:} \url{https://it.wikipedia.org/wiki/Indice\_Gulpease};
			\item \textbf{Slide del corso di Ingegneria del software\g\ relative alla qualità del software\g:} \\\url{http://www.math.unipd.it/~tullio/IS-1/2015/Dispense/L08.pdf};
			\item \textbf{Slide del corso di Ingegneria del software\g\ relative alla qualità del processo\g:} \\\url{http://www.math.unipd.it/~tullio/IS-1/2015/Dispense/L09.pdf}.
		\end{itemize}
\end{document}