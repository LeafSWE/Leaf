\documentclass[../PianoDiQualifica.tex]{subfiles}

\begin{document}
\begin{appendices}
\section{Resoconto delle attivit� di verifica - Fase A}
All'interno di questa prima fase, secondo quanto riportato dal \textit{Piano di Progetto v1.00}, sono previsti pi� momenti in cui viene attivato il processo di verifica. Si � cercato di riportare in questa sezione tutti i risultati che sono stati ottenuti durante questi momenti. Ove fosse necessario, si sono tratte anche delle conclusioni sui risultati ottenuti e su come essi possono essere migliorati.
	\subsection{Resoconto delle attivit� di verifica sui prodotti}
	In questa sezione verranno riportati i dati emessi dalle procedure di controllo della qualit� di prodotto.
		\subsubsection{Documenti}
		In questa sezione vengono riportati i risultati delle attivit� di verifica svolte sui documenti. Esse sono di due tipi:
		\begin{itemize}
			\item verifiche manuali;
			\item verifiche automatizzate.
		\end{itemize}
			\paragraph{Verifiche manuali}
			Le attivit� di verifica manuale della documentazione prodotta sono state svolte in base alla procedura riguardante la verifica dei documenti che � descritta nel documento \textit{Norme di Progetto v1.00}. La verifica manuale ha permesso di individuare soprattutto errori che riguardano le seguenti tipologie:
			\begin{itemize}
				\item periodi troppo lunghi e complessi da capire e interpretare;
				\item aggettivi o verbi utilizzati in modo non appropriato;
				\item incongruenze tra parti diverse dello stesso documento o appartenenti a documenti diversi;
				\item errori nei concetti esposti.
			\end{itemize}
			Di seguito � presentato un riassunto della quantit� di errori trovati (e successivamente risolti) utilizzando la verifica manuale durante l'intera Fase A.
\begin{longtable}{|l|c|c|}
 \hline Periodi lunghi o complessi      &  71\\
 \hline Parole non appropriate     &  43\\
 \hline Incongruenze       &  11\\
 \hline Errori concettuali      &  19\\
\hline
\caption{Errori trovati tramite verifica manuale dei documenti durante la Fase A}
\end{longtable}
			La verifica manuale, in aggiunta, ha permesso di individuare nuovi termini da aggiungere al \textit{Glossario}. Di seguito � presentato un riassunto della quantit� di nuovi termini da aggiungere al \textit{Glossario} che sono stati individuati.
\begin{longtable}{|l|c|c|}
 \hline Termini candidati ad essere aggiunti      &  79\\
 \hline Termini aggiunti al \textit{Glossario}     &  73\\
\hline
\caption{Nuovi termini da inserire nel \textit{Glossario} individuati tramite verifica manuale dei documenti durante la Fase A}
\end{longtable}	
			� stata infine verificata la correttezza dei diagrammi UML utilizzati all'interno dei vari documenti, sempre seguendo le procedure contenute nel documento \textit{Norme di Progetto v1.00}.
			\paragraph{Verifiche automatiche}
			Le attivit� di verifica automatizzate, oltre a rispettare le procedure descritte all'interno delle \textit{Norme di Progetto v1.00}, fanno uso degli strumenti automatici previsti all'interno dello stesso documento. Questi hanno permesso di individuare numerosi errori che riguardano le seguenti tipologie:
			\begin{itemize}
				\item ortografia errata;
				\item utilizzo errato dei comandi \LaTeX\ previsti dalle \textit{Norme di Progetto v1.00};
				\item norme tipografiche non rispettate (esempio: i punti e virgola necessari alla fine degli elenchi puntati o numerati, fatta eccezione per l'ultimo elemento dell'elenco che richiede il punto).
			\end{itemize}
			Di seguito � presentato un riassunto della quantit� di errori trovati (e successivamente risolti) utilizzando la verifica automatica. Si tenga in considerazione il fatto che alcuni degli strumenti automatici utilizzati non sono stati disponibili fin dall'inizio.
\begin{longtable}{|l|c|c|}
 \hline Errori ortografici      &  117\\
 \hline Utilizzo errato \LaTeX     &  13\\
 \hline Errori riguardanti norme tipografiche       &  29\\
\hline
\caption{Errori trovati tramite verifica automatica dei documenti durante la Fase A}
\end{longtable}	
			Merita un discorso a parte il calcolo dell'indice Gulpease, per il quale sono stati imposti nel presente documento dei range che determinano se un documento � accettabile o meno. Di seguito sono stati riportati gli indici ottenuti (relativi ai documenti completi).
\begin{longtable}{|l|c|c|}
\hline
 \multicolumn{1}{|c|}{\textbf{Documenti}}
 & \multicolumn{1}{|c|}{\textbf{Gulpease}}
 & \multicolumn{1}{|c|}{\textbf{Esito}}
\\ \hline
 \hline \textit{Piano di progetto v1.00}      &  0  & Negativo\\
 \hline \textit{Norme di Progetto v1.00}     &  0  & Negativo\\
 \hline \textit{Studio di Fattibilit� v1.00}       &  0  & Negativo\\
 \hline \textit{Analisi dei Requisiti v1.00}      &  0   & Negativo\\
 \hline \textit{Piano di Qualifica v1.00}        &  0   & Negativo\\
 \hline \textit{Glossario v1.00}     &  0   & Negativo\\
\hline
\caption{Esiti del calcolo dell'indice di leggibilit� effettuato tramite strumenti automatici durante la Fase A}
\end{longtable}
	\subsection{Resoconto delle attivit� di verifica sui processi}
		\subsubsection{Processo di documentazione}
			\paragraph{Livello CMM}
			Il gruppo ha cercato di valutare la qualit� del processo di documentazione secondo le metriche stabilite dal modello CMM: inizialmente, livello 1.\\
			In seguito alla redazione del documento \textit{Norme di Progetto} (uno dei primi ad essere realizzato) si ha avuto a disposizione norme valide per ogni tipo di documentazione, strumenti comuni da poter utilizzare e procedure da seguire per effettuare determinate attivit�: questo ha permesso di controllare maggiormente il processo di documentazione, che ha in questo modo guadagnato ripetibilit� (richiesta dal livello 2 di CMM).
			Quindi, tuttora livello 2, perch� il processo di documentazione non possiede ancora la principale caratteristica richiesta dal terzo livello, ovvero la proattivit�.
			Questo livello � ritenuto accettabile (secondo quanto descritto nel presente documento alla sezione "Misure e metriche"), ma durante le prossime fasi si prevede comunque di continuare a lavorare per poter ottenere miglioramenti sotto questi punti di vista (sfruttando PDCA).
		\subsubsection{Processo di verifica}
			\paragraph{Livello CMM}
			L'obiettivo, fin da subito, � stato quello di realizzare un processo di verifica di qualit�. La principale ragione per fare ci� � che la verifica occupa una parte cospicua del progetto: di conseguenza essa � molto costosa. La si vuole rendere pi� efficace e allo stesso tempo pi� efficiente possibile. Per ottenere tali obiettivi si deve rendere il processo controllabile. \\
			Anche per quanto riguarda questo processo, come per quello di documentazione, siamo in grado di dire che � stato raggiunto il livello 2 nella scala prevista da CMM. Il processo ha infatti superato l'iniziale stato caotico nel quale si trovava all'inizio della Fase A (grazie, per esempio, all'utilizzo sistematico di script e di procedure). \\
			Il team non pu� ancora affermare che il processo di verifica adottato abbia raggiunto un livello di maturit� 3, in quanto � stata documentata in modo accettabile solo l'attivit� di realizzazione del processo, e non quella di gestione dello stesso; tuttavia il livello raggiunto � ritenuto accettabile (secondo quanto descritto nel presente documento alla sezione "Misure e metriche"), anche se durante le prossime fasi si prevede comunque di continuare a lavorare per poter ottenere miglioramenti sotto questi punti di vista (sfruttando PDCA).
\end{appendices}
\end{document}