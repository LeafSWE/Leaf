\documentclass[../PianoDiQualifica.tex]{subfiles}

\begin{document}
\begin{appendices}

\section{Test}
	\subsection{Test di accettazione}
	Il test di accettazione serve ad accertare il soddisfacimento dei \textbf{requisiti utente}. Viene effettuato in presenza del proponente che può, in questo modo, avere un primo approccio con il prodotto software\g\ terminato. Nel caso in cui il test avesse esito positivo, si può procedere al rilascio ufficiale del prodotto\g\ 
sviluppato.\\
	Di seguito verranno riportati i test di accettazione, nel momento in cui il gruppo sarà in grado di definirli nel dettaglio.
	\subsection{Test di sistema}
	Il test di sistema verifica il comportamento dinamico del sistema completo al fine di verificare il soddisfacimento dei \textbf{requisiti software}. La maggior parte degli errori dovrebbe essere già stata identificata durante i test di unità e di integrazione. Il test di sistema viene di solito considerato appropriato per verificare il sistema anche rispetto ai requisiti non funzionali, come quelli prestazionali, di qualità e di vincolo. A questo livello, viene effettuata anche una serie di test in una struttura opportunamente mappata da beacon\g\ per verificare il corretto funzionamento del software\g\ ed evidenziare eventuali bug\g\ o mancanze a livello di performance e precisione. Bisognerà verificare quindi:
	\begin{itemize}
		\item che il software\g\ soddisfi tutti i requisiti obbligatori;
		\item che il software\g\ soddisfi tutti i requisiti desiderabili ed opzionali che il gruppo si è impegnato a soddisfare;
		\item che il software\g\ possieda tutte le caratteristiche garantite dal gruppo nel presente documento alla sezione 2.1.2 "Qualità di prodotto";
		\item come il software\g\ si comporta su vari ambienti hardware che soddisfano i vincoli imposti nel documento \analisideirequisiti\ alla sezione 2.5 "Vincoli generali";
	\end{itemize}
	Di seguito verranno riportati i test di sistema, nel momento in cui il gruppo sarà in grado di definirli nel dettaglio.
\end{appendices}
\end{document}