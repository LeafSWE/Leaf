\documentclass[../PianoDiQualifica.tex]{subfiles}

\begin{document}
\begin{appendices}

\section{Test}
	\subsection{Test di accettazione}
	Il test di accettazione serve ad accertare il soddisfacimento dei \textbf{requisiti utente}. Viene effettuato in presenza del proponente che pu�, in questo modo, avere un primo approccio con il prodotto software terminato. Nel caso in cui il test avesse esito positivo, si pu� procedere al rilascio ufficiale del prodotto sviluppato.\\
	Di seguito verranno riportati i test di accettazione, nel momento in cui il gruppo sar� in grado di definirli nel dettaglio.
	\subsection{Test di sistema}
	Il test di sistema verifica il comportamento dinamico del sistema completo al fine di verificare il soddisfacimento dei \textbf{requisiti software}. La maggior parte degli errori dovrebbe essere gi� stata identificata durante i test di unit� e di integrazione. Il test di sistema viene di solito considerato appropriato per verificare il sistema anche rispetto ai requisiti non funzionali, come quelli prestazionali, di qualit� e di vincolo. A questo livello, viene effettuata anche una serie di test in una struttura opportunamente mappata da beacon per verificare il corretto funzionamento del software ed evidenziare eventuali bug o mancanze a livello di performance e precisione. Bisogner� verificare quindi:
	\begin{itemize}
		\item che il software soddisfi tutti i requisiti obbligatori;
		\item che il software soddisfi tutti i requisiti desiderabili ed opzionali che il gruppo si � impegnato a soddisfare;
		\item che il software possieda tutte le caratteristiche garantite dal gruppo nella sezione 2.1.2 di questo documento;
		\item come il software si comporta su vari ambienti hardware che soddisfano i vincoli imposti nel documento \textit{Analisi dei Requisiti v1.00} nella sezione 2.5 \textit{Vincoli generali};
	\end{itemize}
	Di seguito verranno riportati i test di sistema, nel momento in cui il gruppo sar� in grado di definirli nel dettaglio.
\end{appendices}
\end{document}