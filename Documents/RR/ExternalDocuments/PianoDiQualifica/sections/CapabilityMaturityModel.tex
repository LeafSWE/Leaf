\documentclass[../PianoDiQualifica.tex]{subfiles}

\begin{document}
\begin{appendices}

\section{Capability Maturity Model}
	Il CMM, acronimo di Capability Maturity Model, è un modello di sviluppo creato in seguito allo studio, finanziato dal Dipartimento della Difesa Statunitense, dei dati raccolti dalle organizzazioni che collaboravano con esso.
	Tale modello mira a migliorare i processi di sviluppo software esistenti. Il nome stesso del modello suggerisce i concetti su cui si basa:
	\begin{description}
		\item[capability:] è una caratteristica di ogni processo che indica quanto esso sia adeguato per gli scopi per cui è stato definito; tale caratteristica determina l'apporto in termini di efficacia ed efficienza finali raggiungibile attraverso un processo;
		\item[maturity:] è una caratteristica di un insieme di processi, attraverso la quale è possibile misurare quanto è governato il sistema dei processi di un'azienda;
		\item[model:] è la definizione di un insieme di requisiti, sempre più stringenti, che consentono di valutare il percorso di miglioramento dei processi di un'azienda. 
	\end{description}
	Il modello CMM fornisce:
	\begin{itemize}
		\item una base concettuale su cui appoggiarsi per valutare il livello dei processi;
		\item un insieme di best practice consolidate negli anni da esperti e utilizzatori;
		\item un linguaggio comune e una visione condivisa;
		\item un metodo per definire un miglioramento in ambito organizzativo.
	\end{itemize}
	\subsection{Struttura}
	Il modello CMM comprende cinque aspetti:
	\begin{description}
		\item[Livelli di maturità:] sono cinque livelli di maturità, dove il più alto (il quinto) è uno stato teoricamente ideale in cui i processi vengono sistematicamente gestiti attraverso una combinazione di ottimizzazioni di processi e miglioramenti continui di processi;
		\item[Aree chiave di processo:] un'area chiave di processo identifica un gruppo di attività correlate che, quando vengono eseguite insieme, producono una serie di obiettivi considerati strategici;
		\item[Obiettivi:] gli obiettivi di un'area chiave di processo riassumono gli stati che devono esistere per quell'area per essere implementati in modo completo e duraturo. La quantità di obiettivi che sono stati raggiunti è un indicatore della capability che l'organizzazione ha raggiunto in un certo livello di maturità;
		\item[Caratteristiche comuni:] le caratteristiche comuni includono le pratiche che sviluppano e regolamentano un'area chiave di processo. Ci sono cinque tipi di caratteristiche comuni:
		\begin{itemize}
			\item l'impegno nell'esecuzione;
			\item l'abilità nell'esecuzione;
			\item le attività eseguite;
			\item le misurazioni e le analisi;
			\item la verifica e l'implementazione.
		\end{itemize}
		\item[Le pratiche fondamentali:] le pratiche fondamentali descrivono gli elementi dell'infrastruttura e le pratiche che contribuiscono in modo particolare all'implementazione e alla regolamentazione dell'area.
	\end{description}
	\subsection{Livelli}
	Il primo e più importante degli aspetti del modello visti nella sezione precedente riguarda i livelli che indicano il grado di maturità raggiunto da un'azienda.
	\begin{description}
		\item[Primo livello - Iniziale (Caotico)] I processi che rientrano fra quelli di questo livello tipicamente risultano privi di ogni forma di documentazione e in uno stato di continuo cambiamento, riadattati ogni volta alle necessità del momento, poco riusabili e incontrollati. Tutto ciò porta ad un ambiente caotico e instabile per i processi.
		\item[Secondo livello - Ripetibile] I processi di questo livello sono generalmente ripetibili, e spesso danno buoni risultati; inizia a vedersi una certa disciplina nei processi che li porta ad essere rigorosi e robusti.
		\item[Terzo livello - Definito] I processi iniziano ad essere raggruppati secondo standard definiti, vengono documentati e sono soggetti a miglioramenti nel lungo periodo. A questo livello gli standard di processo sono usati per consolidare l'esecuzione dei processi nell'organizzazione.
		\item[Quarto livello - Gestito] A questo livello iniziano ad essere usate metriche di processo e i manager dell'azienda sono in grado di individuare i modi di adeguare e migliorare i processi rispetto a specifici progetti, senza rilevare perdite di qualità o deviazioni dalle specifiche.
		\item[Quinto livello - Ottimizzante] I processi in questo livello sono volti a migliorare continuamente le performance attraverso cambiamenti e miglioramenti sia incrementali che tecnologicamente innovativi.
	\end{description}
\end{appendices}
\end{document}