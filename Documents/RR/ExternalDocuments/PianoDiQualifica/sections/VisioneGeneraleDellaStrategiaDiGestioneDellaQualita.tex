\documentclass[../PianoDiQualifica.tex]{subfiles}

\begin{document}
\section{Visione generale della strategia di gestione della qualit�}
	\subsection{Obiettivi di qualit�}
		\subsubsection{Qualit� di processo}
		Un processo per essere classificato ha bisogno di essere misurato tramite dei parametri. Uno di questi � la qualit� di processo. Assicurare la qualit� dei processi � indispensabile durante lo svolgimento del progetto per le seguenti
		ragioni:
		\begin{itemize}
		\item aiuta ad ottimizzare l'uso delle risorse;
		\item fa in modo che i costi siano maggiormente contenuti;
		\item migliora la stima dei rischi e degli impegni.
		\end{itemize}
		Per il gruppo avere qualit� di processo significa avere processi che possono essere misurati tramite strumenti e continuamente migliorati al fine di raggiungere il massimo livello di qualit�. Ogni processo per essere misurato e migliorato ha bisogno di essere sottoposto a processi di verifica che hanno il compito di valutare il livello di qualit� raggiunto e indicare gli aspetti critici da migliorare.\\
		Il modello che il gruppo usa per valutare il grado di maturit� dei processi � il \textbf{Capability Maturity Model} che permette di dare una classificazione dei processi e fornire istruzioni su come migliorarli.
		\subsubsection{Qualit� di prodotto}
		Il gruppo si prefigge di mantenere la stessa qualit� sia nei processi che nei prodotti.\\
		Per garantire la migliore qualit� del prodotto anche il processo da cui proviene deve avere una buona qualit�.
		Il gruppo per mantenere la qualit� del prodotto cercher� di seguire al meglio lo standard di qualit� [ISO/IEC 9126:2001].\\
		Il gruppo si impegna dunque a garantire le seguenti caratteristiche:
		\begin{itemize}
		\item il prodotto permette agli utenti di utilizzare le funzionalit� in maniera semplice ed efficace;
		\item il prodotto fornisce prestazioni accettabili;
		\item il prodotto garantisce un funzionamento senza interruzioni;
		\item il prodotto � facilmente installabile;
		\item il prodotto possiede le funzionalit� descritte all'interno dei requisiti minimi.
		\end{itemize}
	\subsection{Scadenze temporali}
	Le scadenze che il gruppo Leaf ha deciso di rispettare sono le seguenti:
	\begin{itemize}
		\item Revisione dei Requisiti: 2016-02-16;
		\item Revisione di Progettazione: 2016-04-18;
		\item Revisione di Qualifica: 2016-05-23;
		\item Revisione di Accettazione: 2016-06-17.
	\end{itemize}
	\subsection{Responsabilit�}
		La qualit� � responsabilit� di tutti, nessuno escluso. Tutti i membri del team contribuiscono con il loro lavoro a costruire (o a non costruire) la qualit� del prodotto finale e dei processi attraverso i quali ci si arriva. La qualit�, infatti, viene costruita nel tempo, anche grazie alla cura e all'attenzione che viene posta nello svolgere i vari compiti.
		Di seguito vengono riportate le responsabilit� riguardanti la qualit�, catalogate in base al ruolo.
		\\\\\textbf{Responsabile di Progetto:}
			 \begin{itemize}
				\item deve assicurarsi che i processi siano attentamente controllati e valutati in modo oggettivo (in modo tale che essi siano migliorabili);
				\item deve assegnare i compiti relativi alla verifica di prodotti a persone per quanto possibile indipendenti dallo sviluppo di essi;
				\item deve pianificare attentamente controlli sul processo di qualit� stesso.
			\end{itemize}
 			\textbf{Amministratore di Progetto:}
			\begin{itemize}
				\item deve assicurarsi che siano sempre disponibili le risorse necessarie, sia realizzative che di verifica e validazione;
				\item deve fare in modo che il processo di verifica sia quanto pi� automatizzabile possibile (e quindi efficiente).
			\end{itemize}
			\textbf{Analista:}
			\begin{itemize}
				\item deve assicurarsi di documentare i requisiti qualitativi oltre a quelli funzionali;
				\item deve assicurarsi di aderire agli standard e alle norme riguardanti la documentazione da lui stesso prodotta.
			\end{itemize}
			\textbf{Progettista:}
			\begin{itemize}
				\item deve indirizzare nelle specifica tecnica i requisiti di qualit�;
				\item deve realizzare la progettazione in modo da indirizzare completamente, correttamente ed efficacemente i requisiti di qualit�;
				\item deve assicurarsi di aderire agli standard applicabili nella progettazione.
			\end{itemize}
			\textbf{Programmatore:}
			\begin{itemize}
				\item deve codificare secondo le norme imposte all'interno del progetto;
				\item deve codificare utilizzando gli standard applicabili;
				\item deve fornire i test necessari per effettuare parte delle verifiche sulle unit� software prodotte.
			\end{itemize}
			\textbf{Verificatore:}
			\begin{itemize}
				\item deve eseguire le procedure di verifica previste dal presente documento e descritte nelle \textit{Norme di Progetto v1.00};
				\item deve tracciare gli errori rilevati durante ciascuna fase del progetto affinch� possano essere risolti.
			\end{itemize}
		Per maggiori dettagli circa i compiti assegnati a ciascun ruolo si vedano le \textit{Norme di Progetto v1.00}.
\end{document}