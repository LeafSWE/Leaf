\documentclass[../PianoDiQualifica.tex]{subfiles}

\begin{document}
\section{Visione generale della strategia di gestione della qualità}
	\subsection{Obiettivi di qualità}
	In questa sezione vengono riportati gli obiettivi di qualità che il gruppo \leaf\ si impegna a perseguire durante lo svolgimento dell'intero progetto. Per capire se un certo obiettivo è stato raggiunto oppure no, il gruppo fa uso di standard, modelli e metriche. Ognuno di questi fa uso di scale differenti e fissate a priori: qualunque sia il criterio utilizzato per misurare e dunque quantificare la vicinanza a un certo obiettivo, abbiamo fissato dei valori minimi che intendiamo raggiungere nell'arco dell'intero progetto. Oltre a ciò abbiamo anche fissato dei valori che riteniamo ottimali e che devono essere sperabilmente (ma non obbligatoriamente) raggiunti.
		\subsubsection{Qualità di processo}
		Assicurare la qualità dei processi è indispensabile durante lo svolgimento del progetto per le seguenti ragioni:
		\begin{itemize}
		\item aiuta ad ottimizzare l'uso delle risorse;
		\item fa in modo che i costi siano maggiormente contenuti;
		\item migliora la stima dei rischi e degli impegni.
		\end{itemize}
		Per il gruppo avere qualità di processo\g\ significa avere processi che possono essere misurati tramite strumenti e continuamente migliorati al fine di raggiungere il massimo livello possibile di qualità. Ogni processo\g\ 
per essere misurato e migliorato ha bisogno di essere sottoposto ad attività di verifica che hanno il compito di valutare il livello di maturità raggiunto ed indicare gli aspetti critici da migliorare.\\
		Il modello che il gruppo usa per valutare il grado di maturità dei processi è il \textbf{Capability Maturity Model} (vedi appendice A "Capability Maturity Model") che permette di dare una classificazione dei processi e fornire istruzioni su come migliorarli. A tal riguardo, abbiamo definito le soglie di accettabilità e ottimabilità per i valori che i processi otterranno dopo la verifica del loro livello attuale nella scala\g\ CMM\g: queste soglie sono definite nel presente documento all'interno della sezione 3.4.3 "Metriche per i processi".\\
		Gli obiettivi di qualità di processo\g\ che il gruppo \leaf\ si impegna a perseguire sono:
		\begin{itemize}
			\item miglioramento costante della qualità dei processi, utilizzando il modello del miglioramento continuo PDCA\g;
			\item rispetto della pianificazione definita nel documento \pianodiprogettov;
			\item rispetto del budget definito nel documento \pianodiprogettov.
		\end{itemize}
		\subsubsection{Qualità di prodotto}
		Il gruppo si prefigge di mantenere la stessa qualità sia nei processi che nei prodotti: per garantire la migliore qualità del prodotto\g\ anche il processo\g\ da cui proviene deve avere una buona qualità. Il gruppo per mantenere la qualità del prodotto\g\ 
cercherà di seguire al meglio lo standard di qualità [ISO/IEC 9126:2001].\\
		I prodotti che vengono realizzati durante l'intero progetto sono sostanzialmente di due tipi: documenti e software\g. Nelle prossime sezioni, si enunciano gli obiettivi che si intendono raggiungere suddivisi per tipologia di prodotto\g.
			\paragraph{Qualità dei documenti}
			Gli obiettivi di qualità riguardanti i documenti ai quali il gruppo \leaf\ desidera arrivare nell'arco dell'intero progetto sono i seguenti:
			\begin{itemize}
				\item i documenti devono essere comprensibili da individui dotati di un'istruzione media, caratteristica verificata attraverso l'utilizzo dell'indice Gulpease\g;
				\item i documenti devono essere corretti a livello ortografico;
				\item i documenti non devono contenere concetti errati.
			\end{itemize}
			Abbiamo individuato le metriche o i criteri che si intendono utilizzare per quantificare la vicinanza a ognuno degli obiettivi sopra descritti e le soglie di accettabilità e ottimalità per questi obiettivi, per fissare quantitativamente i punti ai quali desideriamo arrivare. Queste metriche e soglie sono definite nel presente documento all'interno della sezione 3.4.2 "Metriche per i documenti".
			\paragraph{Qualità del software}
			Gli obiettivi di qualità del software\g\ ai quali il gruppo \leaf\ desidera arrivare nell'arco del progetto sono alcuni di quelli che sono enunciati all'interno dello standard [ISO/IEC 9126:2001]. Vengono riassunti in seguito:
			\begin{itemize}
				\item il prodotto\g\ possiede le funzionalità descritte all'interno dei requisiti obbligatori e desiderabili;
				\item il prodotto\g\ permette agli utenti di utilizzare le funzionalità in maniera semplice ed efficace;
				\item il codice risulta manutenibile e facilmente comprensibile;
				\item il prodotto\g\ è robusto e non interrompe l'esecuzione in seguito a situazioni anomale;
				\item il prodotto\g\ è testato in ogni sua parte e in ogni situazione nella quale si può trovare;
				\item il prodotto\g\ garantisce un funzionamento senza interruzioni;
				\item il prodotto\g\ è facilmente installabile.
			\end{itemize}
	\subsection{Scadenze temporali}
	Le scadenze che il gruppo \leaf\ ha deciso di rispettare sono le seguenti:
	\begin{itemize}
		\item \revisionedeirequisiti: \frmdate{16}{02}{2016};
		\item \revisionediprogettazione: \frmdate{18}{04}{2016};
		\item \revisionediqualifica: \frmdate{23}{05}{2016};
		\item \revisionediaccettazione: \frmdate{17}{06}{2016}.
	\end{itemize}
	\subsection{Responsabilità}
		La qualità è responsabilità di tutti, nessuno escluso. Tutti i membri del team\g\ contribuiscono con il loro lavoro a costruire (o a non costruire) la qualità del prodotto\g\ finale e dei processi attraverso i quali ci si arriva. La qualità, infatti, viene costruita nel tempo, anche grazie alla cura e all'attenzione che viene posta nello svolgere i vari compiti.
		Di seguito vengono riportate le responsabilità riguardanti la qualità, catalogate in base al ruolo.
		\\\\\textbf{Responsabile di Progetto:}
			 \begin{itemize}
				\item deve assicurarsi che i processi siano attentamente controllati e valutati in modo oggettivo (in modo tale che essi siano migliorabili);
				\item deve assegnare i compiti relativi alla verifica di prodotti a persone per quanto possibile indipendenti dallo sviluppo di essi;
				\item deve pianificare attentamente controlli sul processo\g\ di qualità stesso.
			\end{itemize}
 			\textbf{Amministratore di Progetto:}
			\begin{itemize}
				\item deve assicurarsi che siano sempre disponibili le risorse necessarie, sia realizzative che di verifica e validazione;
				\item deve fare in modo che il processo\g\ di verifica sia quanto più automatizzabile possibile (e quindi efficiente).
			\end{itemize}
			\textbf{Analista:}
			\begin{itemize}
				\item deve assicurarsi di documentare i requisiti qualitativi oltre a quelli funzionali;
				\item deve assicurarsi di aderire agli standard e alle norme riguardanti la documentazione da lui stesso prodotta.
			\end{itemize}
			\textbf{Progettista:}
			\begin{itemize}
				\item deve indirizzare nella \specificatecnica\ i requisiti di qualità;
				\item deve realizzare la progettazione in modo da indirizzare completamente, correttamente ed efficacemente i requisiti di qualità;
				\item deve assicurarsi di aderire agli standard applicabili nella progettazione.
			\end{itemize}
			\textbf{Programmatore:}
			\begin{itemize}
				\item deve codificare secondo le norme imposte nel documento \normediprogettov;
				\item deve codificare utilizzando gli standard applicabili;
				\item deve fornire i test necessari per effettuare parte delle verifiche sulle unità software\g\ prodotte.
			\end{itemize}
			\textbf{Verificatore:}
			\begin{itemize}
				\item deve eseguire le procedure di verifica previste dal presente documento e descritte nelle \normediprogettov;
				\item deve tracciare gli errori rilevati durante ciascuna fase\g\ del progetto affinché possano essere risolti.
			\end{itemize}
		Per maggiori dettagli circa i compiti assegnati a ciascun ruolo si vedano le \normediprogettov.
\end{document}