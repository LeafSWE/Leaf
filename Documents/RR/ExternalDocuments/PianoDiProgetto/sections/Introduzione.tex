\documentclass[../PianoProgetto.tex]{subfiles}

\begin{document}
\section{Introduzione}
	\subsection{Scopo del documento}
	Questo documento espone l'organizzazione delle attività all'interno del gruppo \leaf, nell'ambito del progetto \textbf{CLIPS}.
In particolare, gli obiettivi di tale documento sono:
	\begin{itemize}
	\item analizzare e gestire gli eventuali rischi;
	\item preventivare l'impiego delle risorse;
	\item fornire un consuntivo delle risorse durante lo svolgimento del progetto;
	\item presentare la pianificazione delle attività da svolgere.
	\end{itemize}
	
	\subsection{Scopo del prodotto}
	Lo scopo del prodotto\g\ è implementare un metodo di navigazione indoor\g\ che sia funzionale alla tecnologia BLE\g .
	Il prodotto\g\ comprenderà un prototipo software\g\ che permetta la navigazione all'interno di un'area predefinita, basandosi sui concetti di IPS\g\ e smart places\g.

	\subsection{Glossario}
		Allo scopo di rendere più semplice e chiara la comprensione dei documenti
viene allegato il \glossariov\ nel quale verranno raccolte le spiegazioni di
terminologia tecnica o ambigua, abbreviazioni ed acronimi. Per evidenziare
un termine presente in tale documento, esso verrà marcato con il pedice \g.


	\subsection{Riferimenti utili}

		\subsubsection{Riferimenti normativi}
		\begin{itemize}
		\item Capitolato d'appalto C2: CLIPS: Communication \& Localization with Indoor Positioning Systems. Reperibile all'indirizzo: \par
			\url{http://www.math.unipd.it/~tullio/IS-1/2015/Progetto/C2.pdf};
		\item Norme di Progetto: \normediprogettov.
		\end{itemize}

		\subsubsection{Riferimenti informativi}
		\begin{itemize}
		\item Software Engineering - Ian Sommerville - 9th Edition 2010:  Part 4: Software Management;
		\item Regolamento di Organigramma reperibile all'indirizzo: \par
			\url{http://www.math.unipd.it/~tullio/IS-1/2015/Progetto/PD01b.html};
		\item Materiale del corso di Ingegneria del software - Gestione di progetto: \par
			\url{http://www.math.unipd.it/~tullio/IS-1/2015/Dispense/L04.pdf};
		\item Analisi dei requisiti: \analisideirequisitiv ;
		\item Piano di qualifica: \pianodiqualificav ;
		\item Studio di fattibilità: \studiodifattibilitav ;
		\end{itemize}
			
\end{document}
