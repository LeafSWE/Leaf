\documentclass[../PianoProgetto.tex]{subfiles}

\begin{document}
\section{Introduzione}
	\subsection{Scopo del documento}
	Questo documento espone l'organizzazione delle attività all'interno del gruppo Leaf, nell'ambito del progetto \textbf{CLIPS}.
In particolare, gli obiettivi di tale documento sono:
	\begin{itemize}
	\item Analizzare e gestire gli eventuali rischi;
	\item Preventivare l'impiego delle risorse;
	\item Fornire un consuntivo delle risorse durante lo svolgimento del progetto;
	\item Presentare la pianificazione delle attività da svolgere.
	\end{itemize}
	
	\subsection{Scopo del prodotto}
	Lo scopo del prodotto è implementare un metodo di navigazione indoor che sia funzionale alla tecnologia BLE.
	Il prodotto comprenderà un prototipo software che permetta la navigazione all’interno di un’area predefinita, basandosi sui concetti di IPS e smart places.

	\subsection{Scadenze}
		Le scadenze che il gruppo Leaf  ha deciso di rispettare sono le seguenti:
		\begin{itemize}
		\item Revisione dei Requisiti: 2016-02-16;
		\item Revisione di Progettazione: 2016-04-18;
		\item Revisione di Qualifica: 2016-05-23;
		\item Revisione di Accettazione: 2016-06-17.
		\end{itemize}


	\subsection{Glossario}
		Per garantire una maggiore chiarezza espositiva ed evitare qualsiasi ambiguità, i termini utilizzati nei vari documenti formali relativi al progetto sono stati raccolti nell’allegato Glossario v1.00. 
		Un termine reperibile nel Glossario si riconosce perché, nei documenti ufficiali, è riportato in corsivo ed accompagnato dal simbolo |g| a pendice.


	\subsection{Riferimenti utili}

		\subsubsection{Riferimenti normativi}
		\begin{itemize}
		\item Capitolato d’appalto C2: CLIPS: Communication \& Localisation with Indoor Positioning Systems. Reperibile all’indirizzo: \url{http://www.math.unipd.it/~tullio/IS-1/2015/Progetto/C2.pdf};
		\item Regolamento di Organigramma reperibile all’indirizzo: \url{http://www.math.unipd.it/~tullio/IS-1/2015/Progetto/PD01b.html};
		\item Norme di Progetto: \normediprogettov.
		\end{itemize}

		\subsubsection{Riferimenti informativi}
		\begin{itemize}
		\item Analisi dei Requisiti: ;
		\item Piano di Qualifica: ;
		\item Studio di Fattibilità: ;
		\item Software Engineering - Ian Sommerville - 9th Edition 2010:  Part 4: Software Management;
		\end{itemize}
			
\end{document}
