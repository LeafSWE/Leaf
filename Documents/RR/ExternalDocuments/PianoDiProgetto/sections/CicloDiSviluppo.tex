\documentclass[../PianoProgetto.tex]{subfiles}

\begin{document}

\section{Ciclo di sviluppo}

	Il modello di ciclo di sviluppo scelto per il prodotto è il modello incrementale: il progetto viene suddiviso in fasi ed il completamento di ogni fase è indicato da una milestone.
	Il proponente, al termine di ogni fase, può valutare il sistema prodotto fino a quel momento e fornire un feedback prezioso.
	Per agevolare il coinvolgimento del proponente, il progetto sarà suddiviso in fasi di breve durata.
	\begin{description}
	
	\item[Fase A - Analisi:] Questa fase prevede 5 sottofasi:
		\begin{itemize}
		\item individuazione degli strumenti necessari al lavoro collaborativo;
		\item individuazione degli strumenti adatti alla redazione della documentazione;
		\item individuazione del progetto da sviluppare;
		\item analisi dei requisiti del progetto che si intende sviluppare.
		\end{itemize}
		Questa fase si conclude con la “Revisione dei Requisiti” che consente di avere un riscontro sulle intenzioni del proponente.


	\item[Fase AD - Analisi di Dettaglio:] In questa fase si procede al consolidamento dei requisiti, individuati nella Fase A, attraverso una nuova analisi.
		Eventuali requisiti individuati dagli analisti in questa fase andranno ad aggiungersi ai requisiti individuati precedentemente. 
		Verranno apportate delle modifiche ai documenti che non rispecchiano le richieste del proponente, mentre agli altri verrà apportato un incremento.


	\item[Fase PDROB - Progettazione di Dettaglio e codifica dei] \ \\
		\textbf{ Requisiti Obbligatori:}
		Questa fase termina con una milestone rappresentata dall’approvazione, da parte del proponente, di un software che soddisfi i requisiti obbligatori.
		Verrà apportato un incremento ai documenti prodotti nelle fasi precedenti.
		Alla Revisione di Progettazione si prevede di consegnare il documento “Definizione di Prodotto”.
  
  
	\item[Fase PDRD - Progettazione di Dettaglio e codifica dei Requisiti] \ \\
		\textbf{Desiderabili:}
		Fase che segue immediatamente la Fase PDROB. Questa fase termina con una milestone rappresentata dall’approvazione, da parte del proponente, di un software che soddisfi i requisiti obbligatori e i requisiti desiderabili.
		Verrà apportato un incremento ai documenti prodotti nelle fasi precedenti.


	\item[Fase PDROP - Progettazione di Dettaglio e codifica dei ] \ \\
		\textbf{Requisiti Opzionali:}
		Come la fase precedente, segue immediatamente la Fase PDRD. Questa fase termina con la “Revisione di Qualifica”, nella quale verrà presentato un software che soddisfi i requisiti obbligatori, i requisiti desiderabili e i requisiti opzionali.
		Verrà apportato un incremento ai documenti prodotti nelle fasi precedenti.


	\item[Fase V - Validazione:] Come la fase precedente segue immediatamente la Fase PDROP. Il progetto si conclude in questa fase. Viene eseguita la validazione del software e, successivamente, il collaudo dello stesso.
		Questa fase termina con la “Revisione di Accettazione”. 
		
	\end{description}
	
		Nel caso in cui il soddisfacimento dei requisiti obbligatori richieda più tempo del previsto, la Fase PDRD e la Fase PDROP non verranno avviate.
		Le fasi possono essere facilmente suddivise in sottofasi meno onerose, questo permette un maggior controllo sull’avanzamento del progetto e dà la possibilità di applicare il PDCA frequentemente.

\end{document}
