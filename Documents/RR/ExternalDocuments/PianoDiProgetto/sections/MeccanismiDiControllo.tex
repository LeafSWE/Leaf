\documentclass[../PianoProgetto.tex]{subfiles}

\section{Meccanismi di Controllo e Rendicontazione}

\begin{document}
Per controllare e valutare lo stato di avanzamento del lavoro e delle attività previste dal progetto, si è scelto di utilizzare i seguenti strumenti:
	\begin{itemize}
	\item Teamwork: lo strumento mette a disposizione un calendario interno, sincronizzabile con Google Calendar. Sarà compito del \responsabilediprogetto\ mantenerlo aggiornato con tutte le milestone, scadenze, incontri, date importanti ed eventuali indisponibilità dei membri del gruppo;
	\item Diagrammi, tabelle e grafici: per rendere più efficace la visualizzazione della pianificazione sono stati realizzati diagrammi di Gantt, tabelle e grafici riassuntivi;
	\item Sistema di Ticketing: per avere sempre sotto controllo lo stato di avanzamento dei lavori e le assegnazioni ai vari componenti del gruppo viene utilizzato il sistema di Ticketing messo a disposizione da GitHub e di assegnazione dei task messo a disposizione da Teamwork. Per un corretto utilizzo dei due strumenti, si rimanda al documento \normediprogetto ;
	\item Rendicontazione delle ore di lavoro: Teamwork dispone di un meccanismo per la rendicontazione delle ore di lavoro. In questo modo, il Responsabile di Progetto può controllare l’avanzamento del lavoro ed eventualmente ridistribuire il carico lavorativo in caso di distribuzione sbilanciata. Questo strumento facilita la stesura del Consuntivo;
	\item Riunioni: per un confronto diretto, valutare lo stato di avanzamento dei lavori e prevedere migliorie o variazioni a quanto già pianificato vengono fatti incontri periodici.
	\end{itemize}
	
\end{document}
