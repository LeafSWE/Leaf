\documentclass[../PianoProgetto.tex]{subfiles}

\section{Meccanismi di Controllo e Rendicontazione}

\begin{document}
Per controllare e valutare lo stato di avanzamento del lavoro e delle attività previste dal progetto e facilitare lo svolgimento del ruolo di \responsabilediprogetto\ si è scelto di utilizzare i seguenti strumenti:
	\begin{itemize}
	\item \textbf{Teamwork}: lo strumento mette a disposizione un calendario interno, sincronizzabile con Google Calendar\g\ . Sarà compito del \responsabilediprogetto\ mantenerlo aggiornato con tutte le milestone\g, scadenze, incontri, date importanti ed eventuali indisponibilità dei membri del gruppo.
	\item \textbf{Diagrammi, tabelle e grafici}: per rendere più efficace la visualizzazione della pianificazione sono stati realizzati diagrammi di Gantt, tabelle e diagrammi riassuntivi.
	\item \textbf{Sistema di ticketing}: per avere sempre sotto controllo lo stato di avanzamento dei lavori e le assegnazioni ai vari componenti del gruppo viene utilizzato il sistema di ticketing messo a disposizione da GitHub\g\ e di assegnazione dei task\g\ messo a disposizione da Teamwork\g. Per un corretto utilizzo dei due strumenti, si rimanda al documento \normediprogettov .
	\item \textbf{Rendicontazione delle ore di lavoro}: Teamwork\g\ dispone di un meccanismo per la rendicontazione delle ore di lavoro. In questo modo, il \responsabilediprogetto\ può controllare l'avanzamento del lavoro ed eventualmente ridistribuire il carico lavorativo in caso di distribuzione sbilanciata. Questo strumento inoltre facilita la stesura del \hyperref[sec:consuntivo]{Consuntivo}.
	\item \textbf{Riunioni interne}: tenute per avere un confronto diretto, per valutare lo stato di avanzamento dei lavori e per prevedere migliorie o variazioni a quanto già pianificato. Le riunioni interne saranno convocate dal \responsabilediprogetto . Per ulteriori approfondimenti si rimanda al documento \normediprogettov .
	\end{itemize}
	
\end{document}
