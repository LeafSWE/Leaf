\documentclass[../PianoProgetto.tex]{subfiles}

\begin{document}

\section{Consuntivo di periodo}
\label{sec:consuntivo}

	Verranno indicate di seguito le spese effettivamente sostenute, sia per ruolo che per persona.
	 
	Il bilancio risultante potrà essere: 
	\begin{itemize}
		\item \textbf{positivo}: se il preventivo supera il consuntivo;
		\item \textbf{in pari}: se consuntivo e preventivo sono equivalenti;
		\item \textbf{negativo}: se il consuntivo supera il preventivo;
	\end{itemize}

	\subsection{Fase A}
		\subsubsection{Consuntivo}
		Le ore di lavoro sostenute in questa fase sono da considerarsi come ore di approfondimento personale, in quanto il gruppo \leaf{} non è ancora stato scelto come fornitore ufficiale per il progetto \progetto.
		
		Tali dati riguardano quindi le ore non rendicontate.

		
\begin{table}[h]
		\centering
		\begin{tabular}{l * {2}{c}}
			\toprule
			\textbf{Ruolo} & \textbf{Ore} & \textbf{Costo (\euro{})} \\
			\midrule
			Responsabile &	33 (+7) & 990,00 (+210,00)\\
			%\midrule
			Amministratore & 87 (+12) & 1.740,00 (+240,00)\\
			%\midrule
			Progettista & 0 & 0,00 \\
			%\midrule
			Analista & 86 (+3) & 2,150,00 (+75,00)\\
			%\midrule
			Programmatore & 0 & 0,00 \\
			%\midrule
			Verificatore & 74 (-14) & 1.110,00 (-210,00)\\
			\midrule
			\textbf{Totale Preventivo} & 280
 & 5.990,00
 \\		
			\textbf{Totale Consuntivo} & 288 & 6.305,00
 \\
			\midrule
			\textbf{Differenza} & +8 & +315,00 \\
			\bottomrule
		\end{tabular}
		
		\caption{Fase A - Consuntivo}
		\label{tab:consuntivoA}
		
	\end{table}		
		
		\subsubsection{Conclusioni}	
		Come si evince dalla tabella \ref{tab:consuntivoA}, che presenta i dati relativi al consuntivo della fase A, è stato necessario investire più tempo del previsto nei ruoli di \responsabilediprogetto{}, \amministratore{} e \analista.
		
		L'attività del \responsabilediprogetto{} ha richiesto più tempo del previsto a causa dell'inesperienza nell'ambito della pianificazione e della mancanza di progetti conosciuti sui quali basare la preventivazione dei costi.		
		
		L'attività degli \amministratori{} ha richiesto più tempo del previsto in quanto è stato necessario apportare modifiche non banali al software adottato per il tracciamento dei requisiti.
		
		L'attività degli \analisti{} ha richiesto più tempo del previsto in quanto il capitolato scelto richiede una buona dose di innovazione e ricerca che, in questa fase, ha impattato sulla specifica dei casi d'uso e dei requisiti.

	
\end{document}
