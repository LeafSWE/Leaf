%al posto di template.tex va messo il nome del file del livello superiore
\documentclass[../Riunione16-01-13.tex]{subfiles}


\section{Verbale della riunione}

Dal confronto con il proponente è emerso:
\begin{itemize}
	\item le informazioni ricavate dai beacon sono poche. Anche con diversi protocolli BLE (ad es. l'Edison sviluppato recentemente da Google) la situazione non cambia. Il proponente consiglia di reperire maggiori informazioni nella documentazione della libreria Altbeacon per Android ed è disposto ad offrirci un incontro dove verranno brevemente illustrate nella pratica alcune funzionalità e modalità per gestire i beacon.
	\item nella documentazione richiesta il proponente desidera sapere nel dettaglio tutto ciò che è stato fatto e come è stato fatto al fine di poter ripetere il risultato ottenuto dalle prove senza la necessità della partecipazione di membri del gruppo \leaf ;
	\item sui dubbi riguardo la user experience il proponente ha spiegato che per user experience intende l'usabilità dell'applicazione, principalmente: l'applicazione deve essere intuitiva e avere,se possibile, tempi di risposta ridotti.
	\item il proponente desidera un prototipo che implementi la navigazione punto a punto o, se questa fallisce, l'esperienza scaturita. Qualsiasi funzionalità in più è ritenuta superflua e opzionale. Se il prototipo dovesse soddisfare la prima condizione il proponente si mette a disposizione per aiutarci a sviluppare anche le parti di contorno, come l'integrazione con i social e una grafica più accattivante;
	\item il collegamento ad un server o piattaforma (Ubiika) non è richiesto. Il proponente esorta a concentrare gli sforzi per l'implementazione della navigazione punto punto e fissare dei livelli di qualità;
	\item è emersa l'idea di estendere alla navigazione la disposizione delle immagini (o simboli identificabili) per aiutare meglio l'utente a seguire il percorso. Si è giunti alla conclusione che ciò sarebbe apprezzato dal proponente anche se rimane sempre nel campo del desiderabile. Questa implementazione però comporta la difficoltà di gestire un considerevole numero di immagini.
	\item sono stati dati alcuni consigli sulla gestione della planimetria e della mappa da rappresentare come un grafo. Questi sono di appoggiarsi a librerie esterne per la gestione dei grafi e mantenere il meno complessa possibile la mappa;
	\item per quanto riguarda i sensori: l'attivazione di GPS e BlueTooth è richiesta fin da subito pena il non utilizzo della tecnologia beacon. Altri accessi e autorizzazioni rimangono da definire;
	\item per quanto riguarda l'hardware il proponente ha espresso la possibilità di procurare al team due telefoni cellulari, un Nexus 5X e un Samsung S4 mini. Ciò per permetterci di variare il più possibile l'hardware su cui testare il prototipo. Il proponente inoltre osserva che le differenze saranno molto visibili e da documentare. Da questo è emerso anche che è richiesto che il prototipo sia sviluppato per Android versione 4.4 e superiore. Per la versione 6 saranno necessarie alcuni accorgimenti in più. 
\end{itemize}
