\documentclass{article}
\usepackage[T1]{fontenc}	% codifica dei font in uscita
\usepackage[utf8]{inputenc} % codifica dei font in uscita
\usepackage[italian]{babel} % lingua del documento
\begin{document}
\title{Casi d'uso}
\author{Leaf}
\maketitle
\section{Casi d'uso principali}
	I casi d'uso principali individuati sono:
	\begin{itemize}
		\item UC1: conferma utilizzo sensori;
		\item UC2: ricevere informazioni;
		\item UC3: navigazione;
		\item UC4: impostare la tipologia di utente;
	\end{itemize}
	
	\subsection{Caso d'uso UC1 - Conferma utilizzo sensori}
	Sotto-casi d'uso
	\begin{itemize}
		\item UC1.1: nel caso i sensori di Localizzazione e Bluetooth siano già accesi la notifica all'utente non si presenta;
		\item UC1.2: nel caso i sensori di Localizzazione e Bluetooth non siano già accesi comparirà sul telefono una notifica all'utente con la richiesta di accensione dei suddetti sensori;
	\end{itemize}
	
	\subsubsection{Caso d'uso UC1.2}
	Sotto-casi d'uso:
	\begin{itemize}
		\item UC1.2.1: permesso negato dall'utente;
		\item UC1.2.2: permesso concesso dall'utente;
	\end{itemize}
	
	\subsection{Caso d'uso UC2 - ricevere informazioni}
	Sotto-casi d'uso:
	\begin{itemize}
		\item UC2.1: ricevere le informazioni riguardanti la posizione dei beacon rilevati;
		\item UC2.2: ricevere le informazioni riguardanti le posizioni di tutti i beacon presenti nella struttura mappata;
	\end{itemize}
	
	\subsection{Caso d'uso UC3 - navigazione}
	Sotto-casi d'uso:
	\begin{itemize}
		\item UC3.1: scegliere la destinazione che si vuole raggiungere;
		\item UC3.2: avviare la navigazione;
		\item UC3.3: fermare la navigazione;
	\end{itemize}
	
	\subsubsection{Caso d'uso UC3.1}
	Sotto-casi d'uso:
	\begin{itemize}
		\item UC3.1.1: scegliere la categoria della destinazione;
		\item UC3.1.2: ricercare la propria destinazione tramite una casella di ricerca;
	\end{itemize}
	Esempio:
	\begin{itemize}
		\item Categoria: Aule
		\item Destinazione: Aula 2BC30
	\end{itemize}
	
	\paragraph{Caso d'uso UC3.1.1}
	Sotto-casi d'uso:
	\begin{itemize}
		\item l'utente trova la destinazione cercata e ci clicca;
	\end{itemize}
	
	\paragraph{Caso d'uso UC3.1.2}
	Sotto-casi d'uso:
	\begin{itemize}
		\item l'utente trova la destinazione cercata e ci clicca;
		\item l'utente non trova la destinazione cercata e gli viene notificato un messaggio d'errore;
	\end{itemize}

	
	\subsection{Caso d'uso UC4 - impostare la tipologia di utente}
	Sotto-casi d'uso:
	\begin{itemize}
		\item UC4.1: impostare per la prima volta la tipologia di utente;
		\item UC4.2: modificare la tipologia di utente;
	\end{itemize}

\end{document}