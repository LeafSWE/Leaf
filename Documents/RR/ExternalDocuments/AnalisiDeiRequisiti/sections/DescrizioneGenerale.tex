\documentclass[../AnalisiDeiRequisiti.tex]{subfiles}

\begin{document}
\section{Descrizione generale}
	\subsection{Obiettivo del prodotto}
	Il prodotto\g\ vuole essere un prototipo utile alla sperimentazione dei risultati conseguiti dal gruppo nella ricerca di metodi di navigazione indoor\g\ alternativi al tradizionale GPS.
Il prototipo dovrà avvalersi della tecnologia beacon\g\ per implementare una navigazione di tipo semantico.

	\subsection{Funzioni del prodotto}
	Il prodotto\g\ permette la navigazione all'interno di un edificio: l'utente può scegliere come destinazione un'area dell'edifico in cui si trova, sarà quindi compito del prodotto\g\ 
fornire informazioni utili perché l'utente possa raggiungere tale destinazione.
L'utente potrà inoltre consultare informazioni di varia natura: sia relative all'edificio nel suo complesso che alle singole aree presenti nell'edifico stesso.
		
	\subsection{Caratteristiche degli utenti} 
	Il prodotto\g\ si rivolge principalmente a due categorie di utenti:
	\begin{itemize}
		\item L'utente finale: utilizza le funzionalità esposte dal prototipo. Agli utenti appartenenti a questa categoria non sono richieste conoscenze tecniche specifiche, è sufficiente che siano in grado di utilizzare uno smartphone Android\g;
		\item L'utente sviluppatore: utilizza funzionalità avanzate del sistema, principalmente per testare l'interazione tra il prototipo ed il sistema che utilizza i beacon\g\ entro il quale il prototipo opera. Gli utenti di questa categoria devono conoscere la teoria alla base della tecnologia beacon\g.
	\end{itemize}
	
	\subsection{Piattaforma di esecuzione}
	Per utilizzare il prototipo è necessario possedere un dispositivo con sistema operativo Android\g\ 4.4 o superiore provvisto di tecnologia BLE\g.
	
	\subsection{Vincoli generali}
	Le funzionalità principali del prototipo sono accessibili solo da dispositivi con Bluetooth\g\ attivato e all'interno di edifici mappati tramite l'opportuno posizionamento di sensori beacon\g.
\end{document}

