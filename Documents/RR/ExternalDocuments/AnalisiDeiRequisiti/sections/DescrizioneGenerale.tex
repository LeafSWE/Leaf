\documentclass[../AnalisiDeiRequisiti.tex]{subfiles}

\begin{document}
\section{Descrizione generale}
	\subsection{Obiettivo del prodotto}
	Il prodotto\g\ vuole essere un prototipo utile alla sperimentazione dei risultati conseguiti dal gruppo nella ricerca di metodi di navigazione indoor\g\ alternativi al tradizionale GPS(Global Positioning System).
Il prototipo dovrà avvalersi della tecnologia beacon\g\ per implementare una navigazione di tipo semantico.

	\subsection{Funzioni del prodotto}
	Il prodotto\g\ permette la navigazione all'interno di un edificio: l'utente può scegliere come destinazione un'area dell'edifico in cui si trova, sarà quindi compito del prodotto\g\ fornire indicazioni utili perché l'utente possa raggiungere tale destinazione, a partire dal punto in cui esso si trova. L'utente potrà inoltre consultare informazioni di varia natura: sia relative all'edificio nel suo complesso che alle singole aree presenti nell'edifico stesso. Queste informazioni possono comprendere eventuali eventi che sono in corso oppure che avverrano in una certa area: ad esempio, quali lezioni saranno tenute in una data aula e i relativi orari; ma queste informazioni possono essere anche di carattere generale sull'area, come il suo nome, un eventuale responsabile addetto o le caratteristiche dell'area: ad esempio, data un aula le sue caratteristiche possono essere il numero di posti, la presenza di un proiettore o meno, ecc. Il prodotto\g\ inoltre potrà fornire a degli utenti autorizzati delle informazioni riguardanti i beacon\g\ rilevati: queste informazioni possono essere utili sia per lo sviluppo di applicazioni che utilizzano questa tecnologia, sia per mappare un edificio e quindi creare le condizioni necessarie affinchè sia possibile utilizzare il prodotto stesso, sia per la manutenzione di tale sistema. Queste informazioni riguardano UUID, Major, Minor, livello di potenza del segnale, livello della batteria, distanza approssimativa dai beacon\g\ rilevati, formato dei beacon\g\ e l'area coperta da un beacon\g.
		
	\subsection{Caratteristiche degli utenti} 
	Il prodotto\g\ si rivolge principalmente a due categorie di utenti:
	\begin{itemize}
		\item L'utente finale: utilizza le funzionalità esposte dal prototipo. Agli utenti appartenenti a questa categoria non sono richieste conoscenze tecniche specifiche, è sufficiente che siano in grado di utilizzare uno smartphone Android\g;
		\item L'utente sviluppatore: utilizza funzionalità avanzate del sistema, principalmente per testare l'interazione tra il prototipo ed il sistema che utilizza i beacon\g\ entro il quale il prototipo opera. Gli utenti di questa categoria devono conoscere la teoria alla base della tecnologia beacon\g.
	\end{itemize}
	
	\subsection{Piattaforma di esecuzione}
	Per utilizzare il prototipo è necessario possedere un dispositivo con sistema operativo Android\g\ 4.4 o superiore provvisto di tecnologia BLE\g.
	
	\subsection{Vincoli generali}
	Le funzionalità principali del prototipo sono accessibili solo da dispositivi con Bluetooth\g\ attivato e all'interno di edifici mappati tramite l'opportuno posizionamento di sensori beacon\g.
\end{document}

