\documentclass[../AnalisiDeiRequisiti.tex]{subfiles}

\begin{document}
\section{Introduzione}
	\subsection{Scopo del documento}
	Questo documento analizza, specifica e classifica i requisiti individuati per il prodotto.\\
	I requisiti sono stati individuati dapprima analizzando il capitolato d’appalto e, successivamente, attraverso un incontro con il proponente.\\
	Il presente documento sancisce un vincolo tra il fornitore ed il proponente: il primo si impegna nello sviluppo di un prodotto avente le caratteristiche riportate di seguito, mentre il secondo 				riconosce che tali caratteristiche sono quelle da esso ricercate.

	\subsection{Scopo del prodotto}
	Lo scopo del prodotto è implementare un metodo di navigazione indoor che sia funzionale alla tecnologia BLE. Il prodotto comprenderà un prototipo software che permetta la navigazione all’interno di un’area predefinita, basandosi sui concetti di IPS e smart places.
	
	\subsection{Glossario} 
	Allo scopo di rendere più semplice e chiara la comprensione dei documenti viene allegato il \glossariov\ nel quale verranno raccolte le spiegazioni di  terminologia tecnica o  ambigua,
abbreviazioni ed acronimi. Per evidenziare un termine presente in tale documento, esso verrà marcato con il pedice \g.
	
	\subsection{Riferimenti utili}
		\subsubsection{Riferimenti normativi}
		\begin{itemize}
			\item Capitolati\g\ d'appalto: \\\url{http://www.math.unipd.it/~tullio/IS-1/2015/Progetto/Capitolati.html}
		\end{itemize}
		\subsubsection{Riferimenti informativi}
		\begin{itemize}
			\item Materiale di riferimento del corso di Ingegneria del Software: \\\url{http://www.math.unipd.it/~tullio/IS-1/2015};
			\item Portable Document Format:  \\\url{http://en.wikipedia.org/wiki/Portable_Document_Format};
			\item Rappresentazione dei numeri: \\\url{https://en.wikipedia.org/wiki/ISO_31-0#Numbers};
			\item Rappresentazione di date e orari: \\\url{https://en.wikipedia.org/wiki/ISO_8601};
			\item Unicode\g: \\\url{http://en.wikipedia.org/wiki/Unicode};
			\item Bluetooth\g\ low energy: \\\url{https://en.wikipedia.org/wiki/Bluetooth_low_energy};
			\item Tracy\g: \\\url{https://GitHub.com/dontpanic-swe/tracy}.
		\end{itemize}
\end{document}

