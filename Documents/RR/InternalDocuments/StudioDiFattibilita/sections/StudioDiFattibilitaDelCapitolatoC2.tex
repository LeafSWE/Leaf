\documentclass[../StudioDiFattibilita.tex]{subfiles}
\begin{document}
\section{Studio di fattibilit� del capitolato C2}
	\subsection{Descrizione del capitolato}
	Il capitolato scelto � denominato CLIPS, acronimo di "Communication \& Localization with Indoor Positioning Systems", ed � stato presentato da Miriade S.p.A.
	Il progetto ha come obiettivo la ricerca e la sperimentazione di nuovi scenari per l'implementazione della navigazione indoor applicata a pi� ambiti.
	In questo senso, il proponente non desidera esplorare uno scenario di proximity marketing, gi� largamente diffuso, ma � interessato all'esplorazione di nuove possibilit� di interazione tra un ambiente "opportunamente cablato" e la popolazione di tale ambiente, attraverso un software installato sul sistema operativo Android o iOS (scelta lasciata al gruppo).
	\subsection{Studio del dominio}
	Per sviluppare il capitolato in esame occorre comprendere l'ambito in cui l'applicazione verr� utilizzata e le tecnologie che bisogner� utilizzare per realizzarla. Si descrivono di seguito il dominio applicativo e quello tecnologico.
		\subsubsection{Dominio applicativo}
		Il software permetter� all'utente di spostarsi all'interno di una struttura a lui sconosciuta senza problemi guidandolo fino alla sua destinazione, a patto che la suddetta struttura sia mappata da sensori beacon; ci� permetter� all'utente di interagire con un edificio in maniera totalmente innovativa e render� l'edificio stesso pi� accessibile a chi non lo ha mai visitato in precedenza.
		\subsubsection{Dominio tecnologico}
		Per l'implementazione del prodotto richiesto, il gruppo andr� ad utilizzare le seguenti tecnologie:
			\begin{itemize}
				\item tecnologia Beacon;
				\item tecnologia BLE - \textit{Bluetooth Low Energy} (definita nelle Specifiche Bluetooth 4.0);
				\item il sistema operativo Android per la creazione di un'applicazione mobile;
				\item linguaggio Java.
			\end{itemize}
	\subsection{Valutazione del capitolato}
		\subsubsection{Motivi della scelta}
		Il capitolato C2 � stato scelto perch� le tecnologie usate e il dominio di applicazione risultavano molto interessanti al gruppo.
		Inoltre, il team ha ritenuto positiva l'acquisizione di conoscenze riguardanti la tecnologia beacon: tecnologia nata nel 2011, in continuo sviluppo e con molto potenziale a livello di mercato.
		\subsubsection{Potenziali criticit�}
		Le criticit� sono state rilevate soprattutto nel campo della navigazione, attualmente non implementata per problemi relativi alla tecnologia beacon ed al suo uso, e sono le seguenti:
			\begin{itemize}
				\item variabilit� del segnale;
				\item i tempi di aggiornamento del beacon possono richiedere anche 30 secondi;
				\item interferenze tra beacon vicini;
				\item problemi derivati dalla struttura in cui sono posizionati (Esempio: muri troppo spessi, beacon nella stessa posizione ma su piani differenti che rischiano di essere confusi tra loro);
				\item problemi derivanti dal sovraffollamento del luogo in cui � posizionato il beacon;
				\item inaffidabilit� della posizione segnalata;
				\item necessit� di avere sul dispositivo il sensore di localizzazione e il sensore Bluetooth sempre accesi.
			\end{itemize}
		\subsubsection{Individuazione dei rischi}
		I rischi rilevati sono stati trattati nel \textit{Piano di Progetto v1.00}.
		\subsubsection{Aspetti di mercato}
		Il prodotto ha l'obiettivo di rivolgersi ad un ampio numero di utenti ed attualmente manca sul mercato, quindi ha alta probabilit� di avere successo e di offrire un contributo importante alla societ�.
	\subsection{Stima di fattibilit�}
	Il gruppo \textit{Leaf} in base allo studio effettuato si prefigge l'obiettivo di portare a termine il prodotto entro le scadenze prefissate e i costi stimati.
	Inoltre, il gruppo non ha mai avuto l'opportunit� di fare esperienza nel campo delle tecnologie trattate ma ritiene di possedere le conoscenze necessarie per riuscire a comprendere le principali problematiche ed intende approfondire ed ampliare le proprie conoscenze.
\end{document}