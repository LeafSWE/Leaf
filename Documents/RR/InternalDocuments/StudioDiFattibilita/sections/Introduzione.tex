\documentclass[../StudioDiFattibilita.tex]{subfiles}
\begin{document}
\section{Introduzione}
	\subsection{Scopo del documento}
	Il presente documento nasce con l'intento di motivare la scelta del gruppo \leaf in favore del capitolato d'appalto C2. Per ognuno dei capitolati proposti in data \frmdate{05}{11}{2015} viene fornito uno studio di fattibilità, analizzando i rischi e i benefici derivanti dalla loro eventuale esecuzione.
	\subsection{Glossario} \label{sec:Glossario}
	Allo scopo di rendere più semplice e chiara la comprensione dei documenti viene allegato il \glossariov\ nel quale verranno raccolte le spiegazioni di  terminologia tecnica o  ambigua,
	abbreviazioni ed acronimi. Per evidenziare un termine presente in tale documento, esso verrà marcato con il pedice \g.
	\subsection{Riferimenti utili}
		\subsubsection{Riferimenti normativi}
		\begin{itemize}
			\item \textbf{Norme di Progetto:} \normediprogettov;
			\item \textbf{Capitolato d'appalto C1:} Actorbase: a NoSQL DB based on the Actor model \\\url{http://www.math.unipd.it/~tullio/IS-1/2015/Progetto/C1.pdf};
			\item \textbf{Capitolato d'appalto C2:} CLIPS: Communication \& Localization with Indoor Positioning Systems \\\url{http://www.math.unipd.it/~tullio/IS-1/2015/Progetto/C2.pdf};
			\item \textbf{Capitolato d'appalto C3:} UMAP: un motore per l'analisi predittiva in ambiente Internet of Things \\\url{http://www.math.unipd.it/~tullio/IS-1/2015/Progetto/C3.pdf};
			\item \textbf{Capitolato d'appalto C4:} MaaS: MongoDB as an admin Service \\\url{http://www.math.unipd.it/~tullio/IS-1/2015/Progetto/C4.pdf};
			\item \textbf{Capitolato d'appalto C5:} Quizzipedia: software per la gestione di questionari \\\url{http://www.math.unipd.it/~tullio/IS-1/2015/Progetto/C5.pdf};
			\item \textbf{Capitolato d'appalto C6:} SiVoDiM: Sintesi Vocale per Dispositivi Mobili \\\url{http://www.math.unipd.it/~tullio/IS-1/2015/Progetto/C6.pdf}.
		\end{itemize}
		\subsubsection{Riferimenti informativi}
		\begin{itemize}
			\item \textbf{Piano di Progetto:} \pianodiprogettov;
			\item \textbf{Glossario:} \glossariov;
			\item \textbf{Materiale del corso di Ingegneria del Software:}\\\url{http://www.math.unipd.it/~tullio/IS-1/2015};
			\item \textbf{Specifiche Bluetooth 4.0:}\\\url{https://www.bluetooth.com/specifications/adopted-specifications};
			\item \textbf{Tecnologia Beacon:} \url{http://developer.estimote.com};
			\item \textbf{Protocollo iBeacon:} \url{https://developer.apple.com/ibeacon}.
		\end{itemize}
\end{document}