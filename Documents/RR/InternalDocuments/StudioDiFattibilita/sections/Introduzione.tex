\documentclass[../StudioDiFattibilita.tex]{subfiles}
\begin{document}
\section{Introduzione}
	\subsection{Scopo del documento}
	Il presente documento nasce con l'intento di motivare la scelta del gruppo \textit{Leaf} in favore del capitolato d'appalto C2. Per ognuno dei capitolati proposti in data 2015-11-05 viene fornito uno studio di fattibilit�, analizzando i rischi e i benefici derivanti dalla loro eventuale esecuzione.
	\subsection{Glossario} \label{sec:Glossario}
	Per evitare ambiguit�e aiutare la comprensione del documento si � redatto un apposito glossario (\textit{Glossario v1.00}) che contiene la spiegazione degli acronimi e delle terminologie tecniche utilizzate. Per facilitare la lettura, i vocaboli in questione all'interno del presente documento sono marcati da una "G" maiuscola a pedice.
	\subsection{Riferimenti utili}
		\subsubsection{Riferimenti normativi}
		\begin{itemize}
			\item \textbf{Norme di Progetto:} \textit{Norme di Progetto v1.00};
			\item \textbf{Capitolato d'appalto C1:} Actorbase: a NoSQL DB based on the Actor model \\\url{http://www.math.unipd.it/~tullio/IS-1/2015/Progetto/C1.pdf};
			\item \textbf{Capitolato d'appalto C2:} CLIPS: Communication \& Localization with Indoor Positioning Systems \\\url{http://www.math.unipd.it/~tullio/IS-1/2015/Progetto/C2.pdf};
			\item \textbf{Capitolato d'appalto C3:} UMAP: un motore per l'analisi predittiva in ambiente Internet of Things \\\url{http://www.math.unipd.it/~tullio/IS-1/2015/Progetto/C3.pdf};
			\item \textbf{Capitolato d'appalto C4:} MaaS: MongoDB as an admin Service \\\url{http://www.math.unipd.it/~tullio/IS-1/2015/Progetto/C4.pdf};
			\item \textbf{Capitolato d'appalto C5:} Quizzipedia: software per la gestione di questionari \\\url{http://www.math.unipd.it/~tullio/IS-1/2015/Progetto/C5.pdf};
			\item \textbf{Capitolato d'appalto C6:} SiVoDiM: Sintesi Vocale per Dispositivi Mobili \\\url{http://www.math.unipd.it/~tullio/IS-1/2015/Progetto/C6.pdf}.
		\end{itemize}
		\subsubsection{Riferimenti informativi}
		\begin{itemize}
			\item \textbf{Piano di Progetto:} \textit{Piano di Progetto v1.00};
			\item \textbf{Glossario:} \textit{Glossario v1.00};
			\item \textbf{Materiale del corso di Ingegneria del Software:}\\\url{http://www.math.unipd.it/~tullio/IS-1/2015};
			\item \textbf{Specifiche Bluetooth 4.0:}\\\url{https://www.bluetooth.com/specifications/adopted-specifications};
			\item \textbf{Tecnologia Beacon:} \url{http://developer.estimote.com};
			\item \textbf{Protocollo iBeacon:} \url{https://developer.apple.com/ibeacon}.
		\end{itemize}
\end{document}