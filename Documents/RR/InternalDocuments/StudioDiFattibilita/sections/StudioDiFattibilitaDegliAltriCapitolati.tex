\documentclass[../StudioDiFattibilita.tex]{subfiles}
\begin{document}
\section{Studio di fattibilit� degli altri capitolati}
	\subsection{Capitolato C1 - Actorbase: a NoSQL DB based on the Actor model}
		\subsubsection{Descrizione del capitolato}
		Il capitolato d'appalto C1 richiede di implementare in linguaggio Scala (preferito) o Java un modello di database NoSQL di tipo Key-value utilizzando il modello ad attori. Per l'implementazione degli attori si richiede l'utilizzo della libreria Akka.
		\subsubsection{Aspetti positivi e negativi}
		Aspetti ritenuti positivi:
			\begin{itemize}
				\item l'idea proposta e le sue specifiche sono molto chiare ed esaustive;
				\item le tecnologie che verrebbero usate risultano interessanti ai membri del gruppo.
			\end{itemize}
		Aspetti ritenuti negativi:
			\begin{itemize}
				\item poco entusiasmo riscontrato da parte del gruppo riguardo al capitolato;
				\item progetto unicamente a scopo di ricerca, senza risvolti a livello di mercato.
			\end{itemize}
		\subsubsection{Individuazione dei rischi}
		La maggior parte dei membri del gruppo � poco interessata al capitolato in esame, oltre ad essere in possesso di poca della conoscenza necessaria per lo svolgimento dello stesso.		
	\subsection{Capitolato C3 - UMAP: un motore per l'analisi predittiva in ambiente Internet of Things}
		\subsubsection{Descrizione del capitolato}
		Il capitolato d'appalto C3 prevede di creare un algoritmo predittivo in grado analizzare i dati provenienti da oggetti collegati in rete (\textit{Internet of Things}), inseriti in diversi contesti, e fornire delle previsioni su possibili guasti, interazioni con nuovi utenti ed identificare dei pattern di comportamento degli utenti per prevedere le azioni degli stessi su altri oggetti o altri contesti.
		\subsubsection{Aspetti positivi e negativi}
		Aspetti ritenuti positivi:
			\begin{itemize}
				\item il gruppo la ritiene un'interessante sfida;
				\item possibilit� di lavorare con tecnologie che saranno la realt� del nostro futuro.
			\end{itemize}
		Aspetti ritenuti negativi:
			\begin{itemize}
				\item le specifiche sono abbastanza generiche;
				\item sarebbe richiesto al gruppo di imparare un elevato numero di tecnologie.
			\end{itemize}
		\subsubsection{Individuazione dei rischi}
		Il gruppo ha individuato un sistema molto complesso da implementare e un elevato numero di tecnologie da imparare, oltre al fatto che non tutti i membri del team risultano dimostrare molto interesse per il capitolato in esame.
	\subsection{Capitolato C4 - MaaS: MongoDB as an admin Service}
		\subsubsection{Descrizione del capitolato}
		Il capitolato d'appalto C4 mira a costruire un servizio web che incorpora la piattaforma MaaP (\textit{MongoDB as an admin Platform}) gi� esistente e la rende disponibile direttamente via web a pi� compagnie: questo servizio web viene denominato MaaS (\textit{MongoDB as an admin Service}).
		\subsubsection{Aspetti positivi e negativi}
		Aspetti ritenuti positivi:
			\begin{itemize}
				\item l'idea proposta e le sue specifiche sono molto chiare ed esaustive;
				\item l'esperienza pu� fornire ampie conoscenze nell'ambito dei database non relazionali.
			\end{itemize}
		Aspetti ritenuti negativi:
			\begin{itemize}
				\item poco entusiasmo riscontrato da parte del gruppo riguardo al capitolato;
				\item vi � una forte dipendenza da MaaP per ovvi motivi e, di conseguenza, poca libert� nello sviluppo.
			\end{itemize}
		\subsubsection{Individuazione dei rischi}
		Come per il capitolato C1, la maggior parte dei membri del gruppo � poco interessata al capitolato in esame, oltre ad essere in possesso di poca della conoscenza necessaria per lo svolgimento dello stesso.
	\subsection{Capitolato C5 - Quizzipedia: software per la gestione di questionari}
		\subsubsection{Descrizione del capitolato}
		Il capitolato d'appalto C5 richiede di costruire un sistema che gestisca la creazione e la manipolazione di questionari: il sistema in questione sar� composto da un archivio di domande e da un sistema di test che pescando da tale archivio somministrer� all'utente dei questionari specifici per l'argomento scelto.
		\subsubsection{Aspetti positivi e negativi}
		Aspetti ritenuti positivi:
			\begin{itemize}
				\item l'idea proposta e le sue specifiche sono molto chiare ed esaustive;
				\item il progetto nella sua complessit� non sembra essere eccessivamente impegnativo;
				\item il gruppo ritiene interessante lavorare con un linguaggio di markup.
			\end{itemize}
		Aspetti ritenuti negativi:
			\begin{itemize}
				\item il gruppo ritiene il progetto poco stimolante sotto molti punti di vista;
				\item il gruppo ritiene il progetto poco istruttivo nell'ottica di sviluppo di un software rispetto agli altri capitolati.
			\end{itemize}
		\subsubsection{Individuazione dei rischi}
		Come per il capitolato C1 e C3, la maggior parte dei membri del gruppo � poco interessata al progetto; oltre a questo, � forte il rischio di un'elevata concorrenza date le specifiche chiare e la non eccessiva difficolt� del capitolato.
	\subsection{Capitolato C6 - SiVoDiM: Sintesi Vocale per Dispositivi Mobili}
		\subsubsection{Descrizione del capitolato}
		Il capitolato d'appalto C6 prevede la realizzazione di un'applicazione per dispositivi mobili (smartphone e tablet) che sfrutti appieno le potenzialit� offerte dal motore di sintesi open source "Flexible and Adaptive Text To Speech" (FA-TTS).
		\subsubsection{Aspetti positivi e negativi}
		Aspetti ritenuti positivi:
			\begin{itemize}
				\item l'esperienza pu� essere divertente oltre che istruttiva;
				\item possibilit� di apprendere nuove conoscenze che in futuro possono tornare utili dato l'enorme diffusione degli assistenti vocali su dispositivi mobili;
				\item molto interesse riscontrato nel gruppo per la simulazione delle emozioni, argomento molto intrigante e mai trattato prima da nessun membro del gruppo;
				\item risvolto sociale del prodotto finale.
			\end{itemize}
		Aspetti ritenuti negativi:
			\begin{itemize}
				\item rischio di sviluppare un prodotto che non trova un utilizzo concreto dopo lo sviluppo;
				\item immaturit� delle tecnologie del capitolato in esame;
				\item ampia libert� lasciata dal committente nella ricerca dell'idea e nello sviluppo del progetto, con conseguenti requisiti molto generali.
			\end{itemize}
		\subsubsection{Individuazione dei rischi}
		All'interno del capitolato viene concessa molta libert� su quanto debba essere fatto. Tale cosa � a prima vista positiva, tuttavia si ritiene che essa possa comportare una maggiore difficolt� nell'individuare gli obiettivi desiderati dal proponente. L'\textit{Analisi dei Requisiti} pu� dunque risultare molto difficile, con conseguente rischio di non soddisfare tutte le aspettative del proponente.
\end{document}