%al posto di template.tex va messo il nome del file del livello superiore
\documentclass[../Riunione2015-12-21.tex]{subfiles}

\begin{document}
\section{Verbale della riunione}
Nella prima parte della riunione Davide Castello e Andrea Tombolato hanno esposto le nostre idee riguardanti il capitolato. Dalla discussione successiva ne è emerso che la parte riguardante la navigazione è più interessante per il proponente. Abbiamo chiesto informazioni sui limiti della tecnologia Bacon, sui beacon forniti e sulla possibilità di utilizzare un loro database. 
Tra i problemi rilevati, attinenti alla navigazione, alcuni riguardano la frammentazione del sistema operativo Android, invece l'utilizzo della bussola può essere una buona idea. \\ Il suggerimento è di sviluppare un'applicazione nativa rispetto un applicazione sviluppata tramite HTML5 per il supporto ai sensori e lato client rispetto lato server per la complessità di sviluppo.
Per il posizionamento dei beacon, sul quale abbiamo l'obbligo di indicare altezze e posizioni decise nella documentazione del progetto, con relativa motivazione, ci è stato suggerito di pensare ad una ubicazione ad un'altezza superiore rispetto l'altezza media delle persone. \\
Per quanto concerne l'interfaccia utente dell'applicazione bisogna prestare particolare attenzione all'usabilità, rispetto alla grafica e alla bellezza estetica, poiché non concerne in alcun modo il progetto. \\
Infine per quanto riguarda lo scopo dell'applicazione ci è stato detto che se ci concentriamo sulla navigazione è necessario trovare un modo che, utilizzando la navigazione semantica, permetta di dare indicazione ad un utente portandolo ad una meta desiderata. Per fare questo è necessario trovare un metodo differente dal GPS, poiché la precisione ottenibile è minore.
