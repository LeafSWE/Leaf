%al posto di template.tex va messo il nome del file del livello superiore
\documentclass[../Riunione16-01-13.tex]{subfiles}


\begin{document}
\section{Verbale della riunione}

Dal confronto con il committente è emerso:
\begin{itemize}
	\item le informazioni ricavate dai beacon sono poche. Anche con diversi protocolli BLE (ad es. l'Edison sviluppato recentemente da Google) la situazione non cambia. Il committente consiglia di reperire maggiori informazioni nella documentazione della libreria Altbeacon per Android ed è disposto ad offrirci un incontro dove verranno brevemente illustrate nella pratica alcune funzionalità e modalità per gestire i beacon.
	\item nella documentazione richiesta il committente desidera sapere nel dettaglio tutto ciò che è stato fatto e come è stato fatto al fine di poter ripetere il risultato ottenuto dalle prove senza la necessità della partecipazione di membri del gruppo \leaf ;
	\item il committente desidera un prototipo che implementi la navigazione punto a punto o, se questa fallisce, l'esperienza scaturita. Qualsiasi funzionalità in più è ritenuta superflua e opzionale. Se il prototipo dovesse soddisfare la prima condizione il committente si mette a disposizione per aiutarci a sviluppare anche le parti di contorno, come l'integrazione con i social e una grafica più accattivante;
	\item il collegamento ad un server o piattaforma (Ubiika) non è richiesto. Il committente esorta a concentrare gli sforzi per l'implementazione della navigazione punto punto e fissare dei livelli di qualità;
	\item sono stati dati alcuni consigli sulla gestione della planimetria e della mappa da rappresentare come un grafo. Questi sono di appoggiarsi a librerie esterne per la gestione dei grafi e mantenere il meno complessa possibile la mappa;
	\item per quanto riguarda i sensori: l'attivazione di GPS e BlueTooth è richiesta fin da subito pena il non utilizzo della tecnologia beacon. Altri accessi e autorizzazioni rimangono da definire;
	\item per quanto riguarda l'hardware il committente ha espresso la possibilità di procurare al team due telefoni cellulari, un Nexus 5X e un Samsung S4 mini. Ciò per permetterci di variare il più possibile l'hardware su cui testare il prototipo. Il committente inoltre osserva che le differenze saranno molto visibili e da documentare.
\end{itemize}

\end{document}
